\documentclass[12pt, a4paper, twoside]{report}
\usepackage[utf8]{inputenc}
\usepackage{titlesec}
\makeatletter

\usepackage[citestyle=numeric, giveninits=true, backend=biber, doi=true, url=true, block=ragged, maxnames=6]{biblatex} % backend=bibtex

\usepackage{microtype}
\addbibresource{references.bib}

\usepackage[ngerman]{babel}
\usepackage{csquotes}
\usepackage{subcaption}
\usepackage[hidelinks]{hyperref}

\usepackage{acronym}
\usepackage{graphicx}
\graphicspath{ {./images/} }
 
\usepackage{glossaries}
\makeglossaries
\loadglsentries{glossar}
\usepackage{fancyhdr}
\setlength{\headheight}{15pt}

% NEXT

\usepackage{blindtext}
% \usepackage[english]{babel}
% \usepackage[utf8]{inputenc}
\usepackage{xfrac} % Brüche im Fließtext
\usepackage{graphicx} % Grafiken
\usepackage{xcolor}
\usepackage{float} % Gleitumgebungen (u.a. [H] für Bilder)
\usepackage{amsmath}
\usepackage{listings}
\usepackage{listliketab}
\usepackage{colortbl}
\usepackage{pdfpages} % für Unterschrift

% Tabellen
\usepackage{booktabs}
\usepackage{tabularx} % Tabellen mit Autobreite
\usepackage{longtable} % Tabellen über mehrere Seiten
\usepackage{ltablex} % longtable + tabularx
\usepackage{array} % Spaltenformate
\usepackage{multirow} % Zellen über mehrere Spalten
\usepackage{adjustbox} % Verkleinerung/Vergrößerung
\newcolumntype{P}[1]{>{\let\newline\\\arraybackslash\hspace{0pt}}p{#1}}
\newcolumntype{L}[1]{>{\raggedright\let\newline\\\arraybackslash\hspace{0pt}}m{#1}}
\newcolumntype{C}[1]{>{\centering\let\newline\\\arraybackslash\hspace{0pt}}m{#1}}
\newcolumntype{R}[1]{>{\raggedleft\let\newline\\\arraybackslash\hspace{0pt}}m{#1}}
\newcolumntype{Y}{>{\raggedright\arraybackslash}X}
\newcommand{\vertical}[1]{\begin{tabular}{@{}c@{}}\rotatebox[origin=c]{90}{#1}\end{tabular}}
\usepackage{makecell} % https://tex.stackexchange.com/a/176780

% Aufzählungs-Listen
\usepackage{enumitem} % https://tex.stackexchange.com/a/109608
\setlist{noitemsep} % https://stackoverflow.com/a/1073140
\setlist[2]{nosep}
\setlist[3]{nosep}
\setlist[4]{nosep}
\setlist[5]{nosep}

\usepackage{textgreek} % https://texblog.org/2012/03/15/greek-letters-in-text-without-changing-to-math-mode/

% \usepackage[acronym]{glossaries}
\DefineBibliographyStrings{ngerman}{
   andothers = {{et\,al\adddot}},
}

\hypersetup{
    pdftitle={Masterthesis - Generierung von Feedback-Regeln fuer Modelle aus annotierten Musterloesungen},
    pdfauthor={Michael Mboni},
    pdfsubject={Masterthesis}
}


% Quelltext
\usepackage{listings}
\usepackage{inconsolata}
\lstset{
	basicstyle=\small\ttfamily,
	frame=single,
	captionpos=b,
	% numbers=none,
	numbers=left,
	numberstyle=\scriptsize,
	% tabsize=2,
	% gobble=2,
	inputpath={ {./content/code/} },
	upquote=true,
	showspaces=false,
	showstringspaces=false
}
% https://en.wikibooks.org/wiki/LaTeX/Source_Code_Listings#Encoding_issue
\lstset{literate=%
	{ä}{{\"a}}1 {ë}{{\"e}}1 {ï}{{\"i}}1 {ö}{{\"o}}1 {ü}{{\"u}}1
	{Ä}{{\"A}}1 {Ë}{{\"E}}1 {Ï}{{\"I}}1 {Ö}{{\"O}}1 {Ü}{{\"U}}1
	{ß}{{\ss}}1
}
\renewcommand{\lstlistingname}{Quelltext}
\renewcommand{\lstlistlistingname}{Quelltextverzeichnis}

% Andere Abkürzungen
% https://texwelt.de/fragen/15497/leerzeichen-nach-punkt-z-b-nach-z-b-oder-eg-oder-vs-oder
% https://tex.stackexchange.com/questions/22561/what-is-the-proper-use-of-i-e-backslash-at
% https://texwelt.de/fragen/1154/was-ist-french-spacing-was-macht-frenchspacing
\newcommand{\sczb}{z.\,B. }
\newcommand{\sczbb}{z.\,B.}
\newcommand{\scdh}{d.\,h. }
\newcommand{\scua}{u.\,a. }
\newcommand{\scuaa}{u.\,a.}
\newcommand{\scog}{o.\,g. }
\newcommand{\scggf}{ggf. }
\newcommand{\scvgl}{vgl. }
\newcommand{\scbzw}{bzw. }
\newcommand{\scso}{(s.\,o.) }
\newcommand{\scsoo}{(s.\,o.)}
\newcommand{\scsu}{(s.\,u.) }
\newcommand{\scsuu}{(s.\,u.)}
\newcommand{\scs}{s. }
\newcommand{\scbspw}{bspw. }
\newcommand{\etal}{et~al. }
\newcommand{\scvs}{vs. }
\newcommand{\sczt}{z.\,T. }

\newcommand{\mono}[1]{\texttt{#1}} %Monofont
\newcommand{\footurl}[1]{\footnote{\url{#1}, zuletzt aufgerufen am \today}}
\newcommand{\noemph}[1]{{\let\emph\relax #1}}

\newcommand{\bolditem}[2]{\item \textsf{\textbf{#1}}\newline #2}

% Zum Kürzen des Texts:
% \renewcommand{\toprule}{\hline}
% \renewcommand{\midrule}{\hline}
% \renewcommand{\bottomrule}{\hline}
\aboverulesep = 0mm
\belowrulesep = 0mm



% Stop Imports
%\linespread{0.4}

\def\whatIsIt{Masterarbeit}
\def\title{Generierung von Feedback-Regeln für Modelle aus annotierten Musterlösungen}
\def\fakultaet{Wirtschaftswissenschaften}

\def\author{Michael Vivian Mboni Saha}
\def\addrLineEins{michael.mboni-saha@stud.uni-due.de}
\def\matrikelNr{3154361}

\def\location{Campus Essen}
\def\date{30.03.2024}

\def\betreuer{Dr. Michael Striewe}
\def\ersterGutachter{Prof. Dr. Michael Goedicke}
\def\fach{M.Sc. Software and Network Engineering}
\def\semester{Wintersemester 2023/2024}


% Change Font
\usepackage{helvet}
\renewcommand{\familydefault}{\sfdefault} % Set the default font family to Helvetica

\usepackage{tocloft}
% to update compile time remove this
\renewcommand{\numberline}[1]{\@cftbsnum #1.\@cftasnum \space} \makeatother

  

\titleformat{\chapter}[block]
  {\normalfont\Huge\bfseries}
  {\thechapter. }
  {0pt}
  {\Huge}

\begin{document}
	% !TeX root = main.tex

\begin{titlepage}
	\vspace*{-6em}
	\hspace{-3.5em}
	\includegraphics[scale=0.2]{icon/ude.jpg}
	\vspace*{\fill}
			
	\begin{center}
		{\huge\bfseries \title} \\
		\vspace*{\fill}
Vorgelegt der Fakultät für Informatik der Universität Duisburg-Essen\\[1cm]
		von\\[1cm]
		\textbf{\author} \\
		\addrLineEins\\
            \url{https://michael-mboni.dev/}\\
		Matrikelnummer: \matrikelNr \\

            \vspace*{\fill}
            {\Large\bfseries \whatIsIt}\\[0.5cm]
		zur Erlangung des akademischen Grades\\[0.5cm]
            {\Large\bfseries{Master of Science (M.Sc.)}}\\
		\vspace*{\fill}
  
 \end{center}
	
	\begin{tabular}{ll}
		Betreuer: & \betreuer\\
		Erstgutachter: & \ersterGutachter\\
		Zweitgutachter: & \zweiterGutachter\\
		Studiengang: & \fach \\
		Studiensemester: & \semester \\
		Datum: & \date
	\end{tabular}
	
\end{titlepage}

	%\frontmatter
	\tableofcontents
        \newpage
        \listoffigures
        \newpage
        \listoftables

        \printglossaries
        \chapter*{Abkürzungsverzeichnis}
\begin{acronym}[Bash]
    \acro{OPAL}{Open Academic Learning Platform}
    \acro{UML}{Unified Modeling Language}
    \acro{UDE}{Universität Duisburg Essen}
    \acro{GReQL}{ Graph Repository Query Language}
    \acro{XMI}{XML Metadata Interchange}
    \acro{UXF} {UML eXchange Format}
    \acro{NLP} {Natural Language Processing }
\end{acronym}
	\chapter*{Abstract - Englisch}

This master's thesis focuses on the development of the GReQL Converter to simplify the process of generating feedback
rules for UML models in the context of E-Assessment systems. The tool is designed to provide (semi-)automated support to
teachers in creating GReQL code required for evaluating student submissions related to UML class diagram tasks on the
JACK platform. The GReQL Converter is designed to address the challenges that teachers face when creating GReQL code,
such as the need for expertise in GReQL syntax and the time-consuming nature of the process. The tool aims to streamline
the process of generating feedback rules for UML models, thereby improving the efficiency and effectiveness of
E-Assessment systems. The results of this work demonstrate the feasibility and effectiveness of the GReQL Converter
in simplifying the process of generating feedback rules for UML models.


\chapter*{Abstract - Deutsch}
Diese Masterarbeit konzentriert sich auf die Entwicklung des GReQL Converters, um den Prozess der Generierung
von Feedback-Regeln für UML-Modelle im Kontext von E-Assessment-Systemen zu vereinfachen. Das Tool soll Lehrern
(halb-)automatisierte Unterstützung bei der Erstellung von GReQL-Code bieten, der für die Bewertung von
Studenteneinreichungen bezüglich UML-Klassendiagramm-Aufgabe auf der JACK-Plattform erforderlich ist.
Der GReQL Converter ist darauf ausgelegt, die Herausforderungen zu bewältigen, mit denen Lehrer bei der
Erstellung von GReQL-Code konfrontiert sind, wie z.B. die Notwendigkeit von Expertise in GReQL-Syntax und der
zeitaufwändige Charakter des Prozesses. Das Tool soll den Prozess der Generierung von Feedback-Regeln für
UML-Modelle vereinfachen und damit die Effizienz und Effektivität von E-Assessment-Systemen verbessern. Die Ergebnisse
dieser Arbeit zeigen die Machbarkeit und Wirksamkeit des GReQL Converter bei der Vereinfachung des Prozesses der
Generierung von Feedback-Regeln für UML-Modelle.

	

	%\mainmatter
	\lhead{}
	\chead{}
	\pagestyle{fancy}

	\chapter{Einleitung}

Das erste Kapitel dieser Arbeit befasst sich mit der grundlegenden Motivation und definiert dann das allgemeine Problem, das gelöst werden soll. Es folgt eine Erläuterung der Zielsetzung und zum Schluss wird der Aufbau der Arbeit im Detail beschrieben.

\section{Motivation}

Nach Abschluss der Prüfungsphasen in Sekundar- und Hochschuleinrichtungen stellt sich regelmäßig die Herausforderung
einer umfangreichen Anzahl an Korrekturaufgaben, die von Lehrkräften und Dozenten bewältigt werden müssen. Diese Aufgabe
kann mitunter mühsam und zeitaufwendig sein, insbesondere im Falle einer hohen Anzahl an Prüfungen~\cite{aufwendig}.
Die Evaluation von Prüfungsleistungen von 20 Prüflingen mag noch als vertretbar erscheinen, doch wie verhält es sich in
Lehrveranstaltungen, in denen sich 400 oder gar mehr Studierende beteiligen? Bereits in den 1960er Jahren wurden
Bestrebungen unternommen, solche Probleme durch die Entwicklung von \gls{E-Assessment-Systemen} zu lösen~\cite{Hollingsworth}.

Heute ist die automatisierte und computerbasierte Bewertung von akademischen Arbeiten weit verbreitet im
Hochschulbereich. Diverse Plattformen wie Dynexite~\cite{Dynexite}, LPLUS~\cite{LPLUS}, Questionmark~\cite{Questionmark},
Turnitin~\cite{Turnitin}, Leapsome~\cite{leapsome} wurden konzipiert, um den Prüfungsprozess zu rationalisieren und vollständig zu automatisieren.
Diese Plattformen finden insbesondere in der Beurteilung geschlossener Fragen wie Multiple-Choice-Aufgaben Anwendung,
bei denen die Antworten klar und die Bewertung einfach automatisiert werden kann. Die Frage, die sich jedoch stellt,
ist: Wie gestaltet sich die Evaluation offener Fragenstellungen in Übungen, wie beispielsweise die Erstellung
konzeptioneller Modelle, bei denen Lösungsansätze variieren können, jedoch dennoch korrekt sind?

In der allgemeinen Informatikbildung ist es üblich, Modelle zur Beschreibung der Architektur eines Systems oder
abstrakterer Konzepte heranzuziehen. Diese Modelle können in Form von Diagrammen wie \ac{UML}-Klassendiagrammen, S
equenzdiagrammen oder Zustandsdiagrammen präsentiert werden. Bei einer Bewertungsaufgabe, bei der ein Studierender ein
Modell auf Grundlage eines gegebenen Textes erstellen soll, sieht sich der Evaluierende oft mit der Herausforderung
konfrontiert, das Ergebnis des Studierenden mit der erwarteten Lösung abzugleichen. Diese Form der Beurteilung kann
jedoch diverse Schwierigkeiten aufwerfen:


\begin{enumerate}
    \item In der Modellierung existiert keine eindeutige Lösungsstrategie. Die Resultate jedes Studierenden müssen mit
    der Musterlösung verglichen werden, um Vollständigkeit sicherzustellen.

    \item Richtige Modelle können von der erwarteten Lösung abweichen und Studierende benachteiligen, was Originalität
    hemmen könnte.

    \item Die Beurteilung kann subjektiv sein. Unterschiedliche Auslegungen könnten zu ungleichen Bewertungen führen,
    besonders wenn mehrere Personen bewerten.

    \item Das Evaluationsmodell ist zeitaufwendig, besonders bei vielen Studierenden. Der Abgleich der Studierendenarbeiten
    mit der Musterlösung erfordert Zeit und kann zu Verzögerungen führen.

\end{enumerate}
    

In diesem Zusammenhang könnte die Implementierung einer automatisierten Bewertung auf Basis klar definierter und spezifischer Kriterien, die in jeder Lösung berücksichtigt sein müssen, eine optimale Lösung darstellen. Eine automatisierte Evaluierung durch eine computerbasierte Anwendung könnte sämtliche dieser Probleme lösen, indem sie eine rasche, objektive und vorurteilsfreie Bewertung ermöglicht. Die automatische Evaluierung von \gls{konzeptuellen Modellen} ist kein neues Forschungsthema. Umfangreiche Untersuchungen in dieser Richtung wurden bereits durchgeführt, und verschiedene Ansätze wurden entwickelt, um dieses Ziel zu realisieren.

\section{Zielsetzung und Abgrenzung}

Im Rahmen dieser Masterarbeit wird das Ziel verfolgt, ein Verfahren zur (halb-)automatischen Erstellung von Bewertungsregeln auf Basis annotierter Musterlösungen zu entwickeln und in Form einer Softwareanwendung zu prototypisieren. Dieses Verfahren soll dazu beitragen, den Prozess der Regeldefinition zu vereinfachen und zu optimieren.

In der Fakultät für Informatik an der \ac{UDE} wird ein E-Assessment-System namens ``\gls{JACK}'' \cite{jack}  verwendet, um Studierende automatisch bei bestimmten Prüfungen und Übungen zu bewerten. \gls{JACK} ist in der Lage, verschiedene Arten von Aufgaben, sowohl geschlossene als auch offene Fragen, zu bewerten. Unter den verschiedenen Aufgabentypen fallen auch Aufgaben zur Erstellung von \ac{UML}-Diagrammen. \gls{JACK} wurde entwickelt, um die Lösungen der Studierenden zu bewerten und Noten zu vergeben. Bei der Bewertung von \ac{UML}-Diagrammen verwendet \gls{JACK} regelbasierte Ansätze. Die Lehrenden müssen die Regeln definieren, die das Diagramm des Studierenden validieren und ihm eine Note zuweisen. Dieser Prozess der Regeldefinition ist jedoch zeitaufwändig und birgt das Risiko, dass kleine, aber bedeutende Unaufmerksamkeitsfehler und Flüchtigkeitsfehler auftreten.

Das angestrebte Verfahren soll daher Lehrkräften dabei unterstützen, solche Bewertungsregeln effizienter und fehlerminimiert zu erstellen, indem es auf vorhandene annotierte Musterlösungen zurückgreift. Durch die Entwicklung einer Softwareanwendung, die diesen Prozess (halb-)automatisch durchführt, wird das Ziel verfolgt, eine zeit- und ressourcensparende Lösung bereitzustellen. Damit kann die Qualität der automatischen Bewertung von UML-Diagrammen in \gls{JACK} verbessert und der Aufwand für die Lehrkräfte reduziert werden.

\section{Aufbau der Arbeit}

Die vorliegende Arbeit gliedert sich in sechs Hauptkapitel, die jeweils einen spezifischen Aspekt des Forschungsgebiets beleuchten.

Im zweiten Kapitel werden die grundlegenden Konzepte und Informationen eingeführt, die für das Verständnis der Arbeit erforderlich sind. Dies umfasst eine Erklärung der \gls{E-Assessment-Systemen}, konzeptioneller Modelle sowie der automatisierten Bewertung. Darüber hinaus werden verschiedene Ansätze zur automatisierten Bewertung im Detail betrachtet. Ein Überblick über den aktuellen Stand der Forschung auf diesem Gebiet wird ebenfalls gegeben, gefolgt von einer Einführung in das E-Assessment-System \gls{JACK}.

Das dritte Kapitel widmet sich der detaillierten Analyse des zugrunde liegenden Problems. Hier wird der Prozess der Erstellung und Einreichung von Übungen sowie deren Bewertung durch die \gls{JACK}-Plattform erläutert. Die spezifischen Herausforderungen und Probleme, die in dieser Arbeit adressiert werden, werden identifiziert und diskutiert.

Anschließend folgt das Kapitel vier, welches hauptsächlich das Konzept zur Bewältigung der zuvor im vorhergehenden Kapitel definierten Problematik behandelt. In diesem Kapitel wird eine theoretische Herangehensweise zur Problemlösung vorgestellt, und es wird ein methodischer Ansatz zur Problembewältigung skizziert. Die praktische Umsetzbarkeit dieser Lösungsstrategie wird im darauf folgenden Abschnitt ausführlich erörtert.

Das Kapitel fünf beschäftigt sich mit der praktischen Umsetzung der entwickelten Lösung. Hier werden die verwendeten Technologien und der Entwicklungsprozess vorgestellt. Die Gründe für die Auswahl bestimmter Technologien, Herangehensweisen und Implementierungsmethoden werden erläutert. Die Gesamtarchitektur des entwickelten Prototyps wird präsentiert und begründet.

Das sechste Kapitel widmet sich der Bewertung der entwickelten Lösung. Es werden Kriterien definiert, anhand derer der Prototyp bewertet wird. Der Evaluationsprozess wird erläutert, einschließlich der durchgeführten User Studies. Im Kapitel sieben werden die Ergebnisse der Arbeit kritisch diskutiert und interpretiert. Hierbei werden die Implikationen der Ergebnisse auf das Gesamtforschungsgebiet erörtert, eventuelle Limitationen der Arbeit aufgezeigt und mögliche Ansätze für zukünftige Forschung identifiziert.

Abschließend bietet Kapitel acht eine Zusammenfassung der wichtigsten Erkenntnisse und Ergebnisse dieser Masterarbeit. Es werden auch Perspektiven für die Weiterentwicklung der vorgestellten Lösung sowie die Bereitstellung der vollständigen Dokumentation des entwickelten Tools dargelegt.
        \chapter{Hintergrund}

Im vorliegenden Kapitel wird eine grundlegende Einführung in die zentralen Konzepte, Methoden und Ansätze gegeben, die
für das Verständnis und die weiterführende Behandlung dieser Masterarbeit von Bedeutung sind. Zunächst erfolgt eine
Erläuterung und Definition des Begriffs ``E-Assessment-Systeme'', um eine klare Basis für die nachfolgende Diskussion
zu schaffen. Im Anschluss wird die Thematik der ``Konzeptionellen Modelle'' ausführlich behandelt, wobei der Fokus auf
den Modellen liegt, die in der Informatik allgemein angewandt werden. Darüber hinaus erfolgt eine umfassende Einführung
in das Konzept der ``Automatisierten Bewertung'', wobei insbesondere die verschiedenen Methoden und Ansätze, die in
diesem Bereich von Relevanz sind, beleuchtet werden.

Des Weiteren wird ein Überblick über die verschiedene Ansätze für die automatisierte Bewertung von UML-Modellen gegeben.
Schließlich erfolgt eine generelle Vorstellung des Tools ``JACK'', welches im Rahmen dieser Arbeit eine herausragende
Rolle spielt und in späteren Abschnitten ausführlicher behandelt wird. Dieses Kapitel dient somit als Grundlage und
Orientierungshilfe, um den Leser in die Thematik einzuführen und die notwendigen Begrifflichkeiten und Zusammenhänge zu
vermitteln, die im weiteren Verlauf der Masterarbeit von zentraler Bedeutung sein werden.

\section{E-Assessment-Systeme}
    Nach Eilers et al. sind E-Assessment-Systeme computergestützte Bildungstechnologien, die entwickelt wurden, um den Prozess der Bewertung von Lernleistungen in Bildungseinrichtungen zu automatisieren, zu verbessern und zu erweitern \cite{eilers2008konzeption}. Diese Systeme ermöglichen die Erfassung, Bewertung und Analyse von Schüler- oder Studentenleistungen in einem digitalen Umfeld. E-Assessment-Systeme verwenden verschiedene Arten von Aufgaben und Prüfungen, darunter Multiple-Choice-Fragen, Essays, Simulationen, interaktive Aufgaben und mehr, um das Wissen, die Fähigkeiten und die Kompetenzen der Lernenden zu bewerten.
    
\subsection{Definition und Vorteile}

Die Nutzung von E-Assessment-Systemen bietet eine Reihe von Vorteilen, die sich aus ihrer Fähigkeit zur Automatisierung ergeben. Diese Systeme verwenden vordefinierte Algorithmen und Kriterien, um die Leistung der Lernenden objektiv zu bewerten, was eine schnellere und effizientere Verarbeitung der Ergebnisse ermöglicht. Dies ermöglicht es Lehrern, mehr Zeit für pädagogische Aktivitäten aufzuwenden, anstatt sich mit manuellen Prüfungen und Aufgabenbewertungen zu befassen. Die Art der Verwendung eines E-Assessment-Systems hängt von den angestrebten Zielen ab \cite{review-e}. Es handelt sich nicht nur um ein Werkzeug zur Bewertung von Schülern am Ende eines Semesters (summative), sondern sie können auch am Lernprozess der Schüler teilnehmen (formative).

Die \gls{summative Bewertung} ist eine Art von Bewertung, die am Ende eines definierten Lernzeitraums durchgeführt wird, wie z.B. am Ende eines Kurses, eines Moduls oder eines Schuljahres. Ihr Hauptziel besteht darin, die Gesamtleistung der Schüler zu messen und festzustellen, inwieweit sie die zuvor festgelegten Lernziele erreicht haben \cite{review-e}. Diese Art der Bewertung wird in der Regel nach Abschluss des Lernens durchgeführt, oft in Form einer Abschlussprüfung oder eines Abschlussprojekts. Die summative Bewertung wird hauptsächlich verwendet, um eine Note zu vergeben oder Entscheidungen wie die Promotion oder den Erwerb eines Abschlusses zu treffen. Ihr Fokus liegt weniger auf detailliertem Feedback an die Schüler als vielmehr auf der Bewertung ihres Kompetenzniveaus \cite{review-e}.

Im Gegensatz zur summative Bewertung handelt es sich bei der formativen Bewertung um einen kontinuierlichen Prozess, der während des Lernens stattfindet. Das Hauptziel ist es, den Fortschritt der Schüler zu verfolgen, ihre Bedürfnisse zu identifizieren und ihnen bei der Verbesserung ihrer Leistung zu helfen \cite{gruttmann2009formatives}. Formative Bewertungen werden regelmäßig während des Lernprozesses durchgeführt, oft in Form von Quiz, Übungen oder Klassendiskussionen. Sie bieten den Schülern konstruktives Feedback, das sie dazu ermutigt, ihr eigenes Lernen zu verstehen, ihre Schwächen zu identifizieren und sich entsprechend zu verbessern. Darüber hinaus leitet die \gls{formative Bewertung} die Lehrer an und zeigt ihnen notwendige Anpassungen in ihrer Unterrichtsgestaltung auf, um den spezifischen Bedürfnissen der Schüler gerecht zu werden. Dies fördert effektiveres und zielgerichtetes Lernen \cite{gruttmann2009formatives}.

Angesichts dieser Unterschiede kann man je nach Art der Übung verschiedene Vorteile der Verwendung von E-Assessment-Systemen nutzen:

\begin{enumerate}
    \item \textbf{Effizienz und Zeitersparnis:} Online-Bewertungssysteme ermöglichen eine automatische Bewertung von Prüfungen und Aufgaben, was den Zeitaufwand für Lehrer erheblich reduziert. Lehrer können so mehr Zeit für die Entwicklung von Lehrinhalten und die Unterstützung der Lernenden aufwenden \cite{alruwais2018advantages}.

    \item \textbf{Skalierbarkeit:} Diese Systeme sind äußerst skalierbar und können gleichzeitig eine große Anzahl von Prüfungen und Aufgaben verwalten. Dies ist besonders nützlich für Bildungseinrichtungen, die viele Schüler oder Studenten betreuen \cite{gruttmann2009formatives}  \cite{alruwais2018advantages}.

    \item \textbf{Schnelles Feedback:} Online-Bewertungssysteme bieten den Lernenden sofortiges Feedback. Dies trägt zur Beschleunigung des Lernprozesses bei und ermöglicht es den Studenten, ihre Leistung sofort zu überprüfen und zu verbessern \cite{gruttmann2009formatives} \cite{alruwais2018advantages}.

    \item \textbf{Individualisierung:} Einige E-Assessment-Systeme ermöglichen es, Prüfungen und Aufgaben an die individuellen Bedürfnisse und Ziele der Lernenden anzupassen. Dies fördert personalisiertes Lernen und ermöglicht es den Studenten, in ihrem eigenen Tempo zu arbeiten \cite{alruwais2018advantages}.

    \item \textbf{Umfassende Datenanalyse:} Diese Systeme sammeln umfangreiche Daten zur Leistung der Lernenden. Durch die Analyse dieser Daten können Bildungseinrichtungen Trends identifizieren, Schwächen und Stärken erkennen und das Lehrplanangebot entsprechend anpassen \cite{alruwais2018advantages}.

    \item \textbf{Reduzierung von Betrug und Plagiat:} Online-Bewertungssysteme verfügen über Sicherheitsmechanismen, die Betrug und Plagiat bei Prüfungen und Aufgaben minimieren. Dies trägt zur Integrität des Bewertungsprozesses bei \cite{alruwais2018advantages}.

    \item \textbf{Flexibilität und Zugänglichkeit:} Online-Bewertungssysteme ermöglichen die Durchführung von Prüfungen und Aufgaben in verschiedenen Umgebungen, einschließlich Online- und Hybrid-Lernumgebungen. Dies bietet den Studenten Flexibilität und Zugänglichkeit zu Bildungsbewertungen \cite{gruttmann2009formatives} \cite{alruwais2018advantages}.
\end{enumerate}

Zusammenfassend kann festgehalten werden, dass E-Assessment-Systeme als eine effiziente und vielseitige Methode zur Bewertung von Lernenden betrachtet werden können, die sowohl formative als auch summative Bewertungen ermöglicht. Zahlreiche Vorteile werden durch sie geboten. Mit diesem Verständnis werden im nächsten Kapitel die möglichen Aufgabentypen in E-Assessment-Systemen nähere Einblicke gewonnen.

\subsection{Mögliche Aufgabentypen}

In der Erstellung von Übungen zur Bewertung von Schülern und Studierenden lassen sich im Allgemeinen zwei primäre Aufgabentypen identifizieren, die konsequent zu zwei verschiedenen Arten von Übungsszenarien führen: \textbf{Geschlossene Fragen} und \textbf{Offene Fragen} \cite{review-e}. Diese Klassifikationen sind jeweils durch spezifische Charakteristika gekennzeichnet und bieten Vorzüge, die auf präzise Beurteilungsziele ausgerichtet sind \cite{kocdar2018cheating}.

\subsubsection{\gls{Geschlossene Fragen} (Closed questions)}

Die Kategorie der Geschlossenen Fragen repräsentiert eine geläufige Form von Aufgaben in E-Assessment-Systemen.
Diese Fragen zeichnen sich durch ihre Fähigkeit aus, eine begrenzte Auswahl vorab definierter Antwortmöglichkeiten
anzubieten, aus denen die Lernenden die adäquate Antwort auswählen müssen\cite{gruttmann2009formatives}. In der Regel
umfasst diese Kategorie Multiple-Choice-Fragen, Wahr-Falsch-Fragen und ähnliche Formate. Die Vorzüge dieses Aufgabentyps
sind vielschichtig. Primär sind sie als effektive Instrumente zur Beurteilung des Verständnisses spezifischer Konzepte
und der Retention von Wissen zu betrachten. Darüber hinaus ermöglichen sie eine prompte und automatisierte Evaluierung.
Allerdings sollte in Betracht gezogen werden, dass geschlossene Fragen in ihrer Fähigkeit, kritisches Denken,
Kreativität und eigenständige Problemlösungsfähigkeiten zu bewerten, eingeschränkt sein können. Infolgedessen eignen
sie sich vorwiegend für Beurteilungsziele, die auf die Prüfung grundlegender Kenntnisse und konzeptioneller Fertigkeiten
abzielen~\cite{gruttmann2009formatives}. Einige Beispiele für eine typische Übung mit geschlossener Frage:

\begin{enumerate}
    \item Multiple-Choice-Fragen: Bei Multiple-Choice-Fragen wird den Lernenden eine Frage gestellt, und sie müssen aus einer Liste von vorgegebenen Antwortmöglichkeiten die richtige auswählen. Diese Art von Frage eignet sich gut, um das Verständnis von Fakten, Konzepten und Definitionen zu überprüfen~\cite{azevedo2019assessment}~\cite{azevedo2015assessment}.

    \item Wahr/Falsch-Fragen: Wahr/Falsch-Fragen erfordern, dass die Lernenden entscheiden, ob eine gegebene Aussage wahr oder falsch ist. Diese Fragen sind besonders nützlich, um das Verständnis von Sachverhalten zu überprüfen \cite{khdour2020semantic}.

    \item Zuordnungsaufgaben (Matching Questions): Bei Zuordnungsaufgaben müssen die Lernenden Elemente aus zwei verschiedenen Listen miteinander in Beziehung setzen. Dies kann verwendet werden, um das Verständnis von Zusammenhängen und Beziehungen zwischen Konzepten zu prüfen \cite{gruttmann2009formatives}.

    \item Lückentexte (Fill-in-the-Blanks): Bei Lückentexten müssen die Lernenden fehlende Wörter oder Phrasen in einem
    Satz oder Text ergänzen. In solchen Übungen wird sehr häufig eine Wortliste bereitgestellt, die von den
    Teilnehmenden zur Lösung der Übungen verwendet werden soll. Diese Art von Aufgabe kann verwendet werden, um das
    Verständnis von Kontext und Details zu überprüfen~\cite{gruttmann2009formatives}.
    
\end{enumerate}

Der Vorteil von geschlossenen Fragen in E-Assessment-Systemen liegt in ihrer klaren Struktur und ihrer objektiven Bewertbarkeit. Sie ermöglichen eine schnelle Auswertung und sind besonders geeignet, um das Wissen über Fakten und Grundlagen zu überprüfen. Darüber hinaus können sie automatisch bewertet werden, was die Effizienz bei der Beurteilung von Lernenden in großen Gruppen erhöht.

\subsubsection{\gls{Offene Fragen} (Open-Ended questions)}
Dem gegenüber bieten Offene Fragen, auch als ``open-ended questions''  bezeichnet, einen anpassungsfähigeren und nuancierteren Ansatz zur Evaluation. Im Kontrast zu geschlossenen Fragen präsentieren sie keine vordefinierten Antworten und gestatten den Lernenden, ihre Gedanken selbstständig zu artikulieren \cite{review-e}. Offene Fragen können in verschiedenen Formen gestellt werden, wie schriftliche Antworten, Lösungen von Problemen, argumentative Erläuterungen oder ähnliche Formate. Einer der Hauptvorteile dieser Fragestellungen liegt darin, dass sie die Beurteilung von kritischem Denken, Kreativität, Synthese- sowie schriftlichen oder mündlichen Ausdrucksfertigkeiten ermöglichen \cite{review-e}. Sie bieten zudem detailliertere Einblicke in das Verständnis und die Fertigkeiten der Lernenden. Es ist jedoch bedeutend zu betonen, dass die Bewertung der Antworten auf diese Fragen häufig komplexer und subjektiver ist, eine menschliche Bewertung erfordert und länger dauern kann als die automatisierte Auswertung \cite{gruttmann2009formatives}. Des Weiteren könnten offene Fragen aufgrund ihres ressourcenintensiven Charakters unter Umständen weniger geeignet sein für umfangreiche Beurteilungen. Einige Beispiele für eine typische Übung mit offener Frage:

\begin{enumerate}
    \item Essay-Fragen: Essay-Fragen sind offene Fragen, bei denen die Lernenden in ausführlichen schriftlichen Antworten ihr Wissen, ihre Analysefähigkeiten und ihre Argumentationsfähigkeiten darlegen müssen. Diese Art von Frage eignet sich gut, um komplexe Konzepte zu vertiefen und kritisches Denken zu fördern \cite{gruttmann2009formatives}.

    \item Fallstudien und Szenario-basierte Fragen: Bei diesen Fragen werden den Lernenden reale oder fiktive Szenarien oder Fallstudien vorgelegt, die sie analysieren und Lösungen oder Empfehlungen entwickeln müssen. Dies fördert die Anwendung von Wissen auf komplexe Probleme \cite{gruttmann2009formatives}.

    \item Reflexionsfragen: Reflexionsfragen ermutigen die Lernenden dazu, über ihr eigenes Lernen, ihre Erfahrungen und ihre Entwicklung nachzudenken. Diese Art von Frage ist besonders nützlich, um metakognitive Fähigkeiten zu fördern und das Bewusstsein für den Lernprozess zu schärfen \cite{gruttmann2009formatives}.

    \item Problemstellungen und Aufgaben mit freier Lösung: Bei dieser Art von Fragen werden den Lernenden komplexe Probleme oder Aufgaben gestellt, für die es keine festen Lösungen gibt. Die Lernenden müssen ihre eigenen Lösungen entwickeln und ihre Entscheidungen begründen \cite{gruttmann2009formatives}.
\end{enumerate}

Offene Fragen bieten den Lernenden die Möglichkeit, ihr Verständnis und ihre Fähigkeiten auf eine tiefere Weise zu zeigen, die über reine Faktenkenntnisse hinausgeht. Sie fördern kritisches Denken, Problemlösungs-fähigkeiten und die Fähigkeit zur Kommunikation komplexer Ideen. Allerdings erfordert die Bewertung von offenen Fragen in der Regel mehr Zeit und Aufwand von Lehrkräften oder Experten, da die Antworten vielfältig und subjektiver Natur sein können.


Übungen, bei denen aus einem Text ein \ac{UML}-Diagramm erstellt werden soll, gehören in der Regel zu den offenen Fragen in E-Assessment-Systemen. Dies liegt daran, dass sie von den Lernenden verlangen, nicht nur Faktenwissen anzuwenden, sondern auch kreativ denken und die Informationen aus dem Text analysieren müssen, um ein geeignetes \ac{UML}-Diagramm zu erstellen.  Diese Übungen erfordern von den Lernenden, dass sie ein tieferes Verständnis für das gegebene Thema entwickeln und die Informationen aus dem Text in einen visuellen Kontext übertragen können \cite{ullrich2021automated}.

Zusammenfassend sind geschlossene Fragen und offene Fragen zwei unterschiedliche Ansätze zur Bewertung in E-Assessment-Systemen. Geschlossene Fragen eignen sich gut zur Bewertung von Grundkenntnissen und zur automatischen Bewertung, während offene Fragen eine erhöhte Flexibilität bieten, um komplexe Fähigkeiten zu bewerten, obwohl sie möglicherweise eine intensivere Bewertung erfordern. Die Auswahl zwischen diesen beiden Arten von Aufgaben hängt von den spezifischen Bewertungszielen, der Art der zu bewertenden Fähigkeiten sowie den verfügbaren Ressourcen für die Bewertung und Auswertung ab. In Kombination tragen diese beiden Fragekategorien wesentlich zu einer umfassenden und ausgewogenen Bewertung der Lernenden im Kontext des E-Assessment bei.


\section{Konzeptionelle Modelle}
Ein konzeptionelles Modell ist eine abstrakte Darstellung oder eine konzeptionelle Struktur, die darauf abzielt, Ideen, Beziehungen, Konzepte oder Entitäten eines bestimmten Bereichs zu beschreiben, ohne in konkrete Details oder Implementierungsdetails einzugehen. Solche Modelle werden häufig in verschiedenen Bereichen wie Informatik, Wissenschaft, Ingenieurwissenschaften, Management, Philosophie usw. verwendet, um das Verständnis eines komplexen Themas zu klären, die Kommunikation und Diskussion zu erleichtern und als Grundlage für die Gestaltung oder Analyse konkreterer Systeme zu dienen \cite{abramowicz2013business}.

Konzeptionelle Modelle können verschiedene Formen annehmen, darunter Diagramme, Schemata, grafische Darstellungen, textuelle Beschreibungen sowie mathematische Repräsentationen. Diese dienen häufig als initialer Schritt innerhalb des Modellierungs- oder Problemlösungsprozesses, um die grundlegenden Konzepte und Zusammenhänge zu erfassen, bevor man sich in die detaillierten Facetten vertieft \cite{abramowicz2013business}. In der Fachdisziplin der Informatik, zum Beispiel, wird ein konzeptionelles Modell genutzt, um Schlüsselparameter und Verknüpfungen innerhalb einer Datenbank zu definieren, ohne Einzelheiten zur Datenspeicherung oder -abfrage preiszugeben \cite{abramowicz2013business}. Es kann auch verwendet werden, um die logische Architektur eines Systems zu beschreiben, wobei die Beziehung zwischen Objekten, Klassen und verschiedenen Entitäten hervorgehoben wird.

Im Allgemeinen nutzen Modellierungssprachen in den betreffenden Fachgebieten Notationen, die graphische Symbole einschließen und in zweidimensionalen visuellen Darstellungen resultieren \cite{moody2009physics}. Solche Darstellungen sind gebräuchlicherweise als Diagramme bekannt, und dies findet oft seinen Ausdruck in den Namen spezifischer Modelltypen wie ``Entity-Relationship-Diagramm'' oder ``\ac{UML}-Klassendiagramm''. Falls grafische Symbole innerhalb eines Modells eingesetzt werden, erfolgt ihre Annotation üblicherweise durch textliche Kennzeichnungen, um ihre Relevanz im Kontext des modellierten Objekts zu präzisieren. Die Modellierungssprachen, die im Rahmen der initialen Forschung herausragen (wie \ac{ERD}, UML, EPC,\ac{BPMN} und Petri-Netze), repräsentieren beispielhafte Instanzen solcher graphischen Modellierungssprachen \cite{ullrich2021automated}.

Im Fachgebiet der Informatik gibt es verschiedene Arten von konzeptionellen Modellen, die je nach ihrem Anwendungsbereich und Ziel unterschiedliche Formen und Eigenschaften aufweisen. Hier sind einige häufig vorkommende Arten von konzeptionellen Modellen in der Informatik:

\begin{enumerate}
    \item \textbf{Entity-Relationship-Modelle (ER-Modelle):} Diese Modelle werden verwendet, um die Struktur von Datenbanken zu beschreiben, indem sie Entitäten (Objekte oder Konzepte) und deren Beziehungen zueinander darstellen. ER-Modelle verwenden typischerweise Diagramme, um Entitäten, Attribute und Beziehungen grafisch darzustellen \cite{gregersen1999temporal}.
    
    \item \textbf{UML-Diagramme (Unified Modeling Language):} UML ist eine weit verbreitete Modellierungssprache in der Softwareentwicklung. Sie umfasst verschiedene Diagrammtypen, die zur Modellierung von Softwarearchitekturen, Prozessen und Verhaltensweisen verwendet werden \cite{reggio2013used}. 
    
    \item \textbf{\ac{DFD}:} DFDs werden verwendet, um den Datenfluss und die Datenverarbeitung in Informationssystemen darzustellen. Sie zeigen, wie Daten zwischen Prozessen, Datenlagern und externen Entitäten fließen \cite{li2009data}.
    
    \item \textbf{BPMN-Diagramme (Business Process Model and Notation):} BPMN ist eine Modellierungssprache, die sich auf die Darstellung von Geschäfts-prozessen konzentriert. Mit BPMN-Diagrammen können Abläufe, Aktivitäten und Entscheidungen innerhalb eines Unternehmensmodells visualisiert werden \cite{white2004introduction}.
    
    \item \textbf{Petri-Netze:} Petri-Netze sind mathematische Modelle, die zur Modellierung und Analyse von parallelen und verteilten Systemen verwendet werden. Sie sind besonders nützlich bei der Modellierung von Prozessen in der Softwareentwicklung und der Kommunikation zwischen Komponenten \cite{petri2008petri}.
    \item \textbf{Systemarchitekturmodelle:} Diese Modelle bieten eine Übersicht über die Architektur eines Software- oder Informationssystems und zeigen die Hauptkomponenten und ihre Interaktionen.
\end{enumerate}

Diese Arten von konzeptionellen Modellen dienen dazu, komplexe Systeme, Prozesse und Datenstrukturen in der Informatik zu erfassen, zu analysieren und zu kommunizieren. Die Wahl des geeigneten Modells hängt von den spezifischen Anforderungen und Zielen eines Projekts ab. 

Das Thema dieser Masterarbeit hat als Anwendungsfall Aufgabe mit offenen Fragen, die sich mit der Modellierung mit UML-Diagrammen befassen und bewertet werden sollen. Die Unified Modeling Language (UML) ist eine standardisierte und visuelle Modellierungssprache, die in der Softwareentwicklung und Systemmodellierung weit verbreitet ist. UML dient dazu, komplexe Systeme, insbesondere Softwareanwendungen, zu beschreiben, zu analysieren, zu entwerfen und zu dokumentieren. Sie wurde erstmals in den 1990er Jahren von Grady Booch, James Rumbaugh und Ivar Jacobson entwickelt und hat sich seitdem zu einem Industriestandard für die Modellierung von Software- und Systemarchitekturen entwickelt \cite{UML-History}.

UML bietet eine breite Palette von Diagrammtypen, darunter Klassendiagramme, Aktivitätsdiagramme, Sequenzdiagramme, Zustandsdiagramme und viele mehr. Jeder Diagrammtyp konzentriert sich auf bestimmte Aspekte eines Systems und ermöglicht es den Entwicklern und Ingenieuren, die verschiedenen Elemente und deren Beziehungen in einer klaren und leicht verständlichen visuellen Darstellung festzuhalten \cite{UML-History}. Dies fördert eine bessere Kommunikation und Zusammenarbeit zwischen den Mitgliedern eines Entwicklungsteams sowie zwischen den verschiedenen Interessengruppen eines Projekts.

Die Verwendung von UML in der Softwareentwicklung bietet eine Reihe von Vorteilen, darunter die Möglichkeit, Systeme zu abstrahieren, zu modularisieren und zu dokumentieren, was die Entwicklung, Wartung und Erweiterung von Software erleichtert \cite{UML-History}. Darüber hinaus unterstützt UML die frühzeitige Fehlererkennung und das systematische Design von Softwarelösungen, was zu einer höheren Qualität und Zuverlässigkeit von Anwendungen führt. Aufgrund seiner weitverbreiteten Akzeptanz und seiner Fähigkeit, komplexe Ideen in leicht verständlichen Diagrammen darzustellen, spielt UML eine entscheidende Rolle in der modernen Softwareentwicklung und Systemmodellierung und bildet die Grundlage für die Erstellung und den Austausch von Modellen und Entwurfsmustern in der Industrie \cite{UML-History}.



\section{Automatisierte Bewertung}

\subsection{Definition und Abgrenzung}

Die automatisierte, computergestützte Bewertung von akademischen Einreichungen, wird in der Hochschulbildung häufig verwendet. Informatikbasierte automatisierte Bewertungssysteme existieren bereits seit den 1960er Jahren~\cite{ullrich2021automated}. Diese Systeme werden eingesetzt, um sowohl offene Aufgaben zu bewältigen, als auch geschlossene Fragen.  Bei einer Bewertung, in der ein Student ein Modell aus einem gegebenen Text erstellen soll, ist der Korrektor oft dazu verpflichtet, einen Vergleich zwischen dem vom Studenten vorgelegten Ergebnis und der erwarteten Lösung durchzuführen. Allerdings kann dieses Korrekturmodell mehrere Probleme aufwerfen:

\begin{enumerate}
    \item Zunächst einmal, da es im Bereich der Modellierung keine eindeutige Lösung gibt, wird diese Aufgabe weniger offensichtlich, da die Ergebnisse jedes Studenten mehrmals mit der vorgeschlagenen Lösung verglichen werden müssen, um sicherzustellen, dass die Arbeit des Studenten alle wesentlichen Komponenten der Lösung berücksichtigt~\cite{geer1988open}.
    \item Das zweite Hindernis, das aus diesem Korrekturmodell resultiert, liegt in der inhärenten Subjektivität dieser Methode. Einige Studenten könnten Modelle entwickeln, die vollkommen funktionsfähig sind, aber von der erwarteten Lösung abweichen, und für diese Abweichung benachteiligt werden. Darüber hinaus kann dies zu einer gewissen Starrheit in der Lehre führen und die Studenten daran hindern, originelle Ansätze zu erkunden~\cite{mccann2010factors}~\cite{hancock1995implementing}.
    \item Eine weitere wichtige Herausforderung dieses Korrekturmodells besteht in der potenziellen Subjektivität der Korrektoren. Jeder Korrektor kann seine eigene Interpretation dessen haben, was eine richtige Antwort ausmacht, was zu Inkonsistenzen bei den Bewertungen und einer ungleichmäßigen Verteilung der Noten zwischen den Studentenarbeiten führen kann. Dies kann besonders problematisch werden, wenn mehrere Korrektoren dieselben Arbeiten bewerten, was zu erheblichen Unterschieden in den vergebenen Noten führen kann~\cite{mccann2010factors}.
    \item Schließlich kann dieses Korrekturmodell auch zeitaufwändig sein, insbesondere in Kursen mit einer großen Anzahl von Studenten. Der detaillierte Vergleich zwischen den Studentenarbeiten und der Referenzlösung erfordert Zeit und kann zu Verzögerungen bei der Mitteilung der Ergebnisse an die Studenten führen und die Arbeitsbelastung der Korrektoren erhöhen.
\end{enumerate}

In diesem Zusammenhang könnte die Einführung einer automatisierten Korrektur auf der Grundlage klar definierter spezifischer Kriterien, die in jeder Lösung zwingend enthalten sein müssen, eine optimale Lösung darstellen. Eine automatisierte Korrektur durch ein Computerprogramm könnte jedes dieser Probleme lösen, indem sie eine schnelle, faire und vorurteilsfreie Bewertung ermöglicht.

\subsection{Automatisierte Bewertungsmethoden für UML}

Die Bewertung von UML-Diagrammen ist in verschiedenen Bildungskontexten von entscheidender Bedeutung, sei es in Informatiklehrgängen an Universitäten oder in der beruflichen Weiterbildung. Dabei kann die manuelle Bewertung von UML-Diagrammen zeitaufwändig und subjektiv sein, insbesondere wenn es sich um eine große Anzahl von Diagrammen handelt. Um diese Herausforderungen zu bewältigen und eine effiziente und objektive Bewertung zu gewährleisten, werden verschiedene Methoden und Ansätze zur automatisierten Bewertung entwickelt und angewendet:

\begin{enumerate}
    \item \textbf{Methoden zur Modellvergleich:}  Diese Methoden beinhalten den Vergleich des bewerteten Modells mit
    einem oder mehreren \\ Lösungsmodellen. Dieser Vergleich kann mithilfe von Ähnlichkeits-maßen, wie
    Ähnlichkeitsmaßen oder Graphenabgleich, durchgeführt werden. Dabei werden Ähnlichkeiten und Unterschiede zwischen dem bewerteten
    Modell und den Lösungsmodellen ermittelt~\cite{ullrich2021automated}~\cite{fauzan2021different}.

    \item \textbf{Regelbasierte Ansätze:} Regelbasierte Ansätze verwenden vordefinierte Regeln oder Kriterien zur Bewertung des bewerteten Modells. Diese Regeln können verschiedene Formen annehmen, darunter Graphabfragen, Eigenschaften, Metriken, Mängel oder Suchmuster innerhalb des bewerteten Modells. Die Bewertung erfolgt anhand der Einhaltung oder Nichteinhaltung dieser Regeln \cite{ullrich2021automated} \cite{striewe2011automated}.

    \item \textbf{Constraints-basierte Ansätze:} Diese Ansätze sind eng mit regelbasierten Ansätzen verwandt und werden häufig in intelligenten Tutorensystemen eingesetzt. Sie beinhalten die Anwendung von Einschränkungen oder Regeln zur Bewertung des bewerteten Modells. Diese Einschränkungen können sich auf die Struktur, die Semantik oder andere Aspekte des Modells beziehen \cite{ullrich2021automated} \cite{holland2011effects}.

    \item \textbf{Methoden des maschinellen Lernens:}  Einige aktuelle ForschungsArtikel präsentieren Ansätze, die auf Methoden des maschinellen Lernens setzen, um die automatisierte Bewertung durchzuführen. Hierbei werden maschinelle Lernmodelle trainiert, um Modelle auf Grundlage vorheriger manueller Bewertungen zu bewerten. Dies beinhaltet das Training von Modellen zur Identifizierung von Mustern, Anomalien und potenziellen Problemen in UML-Diagrammen \cite{ullrich2021automated} \cite{boubekeur2020automatic} \cite{ml}.

    \item \textbf{Andere Techniken:}  Neben den genannten Ansätzen gibt es verschiedene andere Techniken, wie die Simulation des bewerteten Modells, Teststrategien, die Gruppierung von Modellen und die Ausrichtung des bewerteten Modells mit einer annotierten textuellen Beschreibung \cite{ullrich2021automated} .

    \begin{enumerate}
        \item \textbf{Werkzeuge zur Modelltransformation und Codegenerierung:}  Diese Werkzeuge ermöglichen die automatische Generierung von Code aus UML-Diagrammen. Sie helfen dabei, den manuellen Aufwand zur Übersetzung eines UML-Designs in ausführbaren Code zu reduzieren und die Korrektheit des Diagramms zu überprüfen. Dies geschieht durch die automatisierte Erzeugung von ausführbarem Code aus den Elementen und Strukturen des UML-Diagramms und ermöglicht die anschließende Überprüfung der Korrektheit des generierten Codes \cite{sturm2002generating}.
    \end{enumerate}
\end{enumerate}

Diese verschiedenen Ansätze und Techniken tragen dazu bei, die automatisierte Bewertung von UML-Diagrammen in der Bildung und anderen Anwendungsbereichen effektiver und vielfältiger zu gestalten. Sie ermöglichen eine präzise und umfassende Bewertung von Modellen und tragen zur Verbesserung der Qualität von UML-Diagrammen bei.

\subsection{Herausforderung der automatisierten Bewertung}

Das vorrangige Ziel der automatisierten Bewertung besteht in der Verleihung von Bewertungen, was als eine Klassifizierungsaufgabe innerhalb des Bewertungskontexts betrachtet werden kann~\cite{laumer2009assessment}. In Fällen von offenen Aufgaben erfolgt die Bewertung anhand der relativen Position der eingereichten Lösung innerhalb des umfassenden Lösungsraums. Dieser Lösungsraum beinhaltet eine Bandbreite von Antwortmöglichkeiten, die von vollständigen Lösungen über partielle Lösungen bis hin zu ungültigen oder nicht akzeptablen Lösungen reicht. Dieser methodische Ansatz zur automatisierten Bewertung, der auf der Unterscheidung und Beurteilung der Qualität von Antworten basiert, findet in verschiedenen Bereichen breite Anwendung. Beispiele hierfür sind die automatisierte Bewertung von Essays \cite{valenti2003overview}, die Beurteilung von Programmieraufgaben \cite{gross2012feedback} sowie die Modellierung und Bewertung von Modellen in diversen Fachdisziplinen \cite{sousa2015structural}. Dieser Ansatz ermöglicht eine effiziente, objektive und skalierbare Bewertung von eingereichten Arbeiten, und er trägt wesentlich dazu bei, den Bewertungsprozess zu rationalisieren und zu standardisieren.


Im Kontext der Erstellung eines Diagramms (UML, Sequenzdiagramm, Entity-Relationship-Diagramm oder andere) aus einem Text, der ein System beschreibt, wird diese Aufgabe in der Regel als offen angesehen. Die Studierenden müssen die im Text beschriebenen Konzepte, Entitäten und Beziehungen abstrakt interpretieren und darstellen, was bedeutet, dass es normalerweise keine eindeutige Antwort oder vordefinierte Lösung gibt. Verschiedene Studierende können leicht unterschiedliche Diagramme erstellen, um die gleiche textuelle Beschreibung darzustellen.

Da diese Aufgabe offen und subjektiv ist, müssen die Prüfer die Kreativität und das Verständnis jedes Studierenden bewerten. Ihr Ziel ist es zu bestimmen, ob das erstellte Diagramm die im Ausgangstext beschriebenen Konzepte und Beziehungen angemessen erfasst~\cite{fellmann2016evaluation}. Diese Komplexität macht den Ansatz auf Grundlage von Regeln für die Bewertung dieser Aufgabe geeignet, da er es ermöglicht, vordefinierte Kriterien anzuwenden und spezifisches Feedback entsprechend dieser Kriterien bereitzustellen.


\section{Automatisierte Bewertung von UML-Modellen: Verschiedene Ansätze}\label{sec:automatisierte-bewertung-von-uml-modellen:-verschiedene-ansatze}

In diesem Kapitel wird die Diskussion auf bestimmte, mehr oder weniger aktuelle Technologien ausgedehnt, die im Kontext
des E-Assessments verschiedene Methoden aus dem vorherigen Kapitel verwenden.

\subsection{Machine learning zur Bewertung von UML}

Der Artikel mit dem Titel ``Automatic Assessment of Students Software Models Using a Simple Heuristic and Machine
Learning''~\cite{boubekeur2020automatic} stammt von den Autoren Younes Boubekeur, Gunter Mussbacher und Shane McIntosh.
In ihrem Artikel stellen sie einen innovativen Ansatz zur Bewertung von Studienarbeiten in Modellierungskursen vor. Ihr
Hauptziel besteht darin, den zeitaufwändigen und subjektiven Bewertungsprozess von UML-Diagrammen zu bewältigen, der in
der Softwaretechnikausbildung weit verbreitet ist. Die vorgeschlagene Methodik kombiniert einen einfachen heuristischen
Algorithmus mit fortgeschrittenen maschinellen Lernverfahren, um nicht nur hochwertige Studienarbeiten zu
identifizieren, sondern auch ungefähre Noten vorherzusagen.

stellen Die Autoren stellen ihren innovativen Ansatz vor, der einen einfachen heuristischen Algorithmus verwendet, um
die Einreichungen der Studierenden mit einer Idealvorlage zu vergleichen~\cite{boubekeur2020automatic}. Dieser
heuristische Algorithmus basiert auf dem Prinzip, die Unterschiede zwischen der Einreichung des Studierenden und der
Idealvorlage zu quantifizieren und anschließend eine Punktzahl aufgrund dieser Unterschiede zuzuweisen~\cite{huyck1993efficient}.
Darüber hinaus setzen die Autoren maschinelle Lernverfahren ein, um ungefähre Noten vorherzusagen, basierend auf den von
dem heuristischen Algorithmus generierten Punktzahlen (siehe Abbildung~\ref{fig:ml-approach}).

\begin{figure}
	\centering
	\includegraphics[width=14cm]{images/ml-approach}
	\caption{Ansatz der Studie \cite{boubekeur2020automatic}.}
	\label{fig:ml-approach}
\end{figure}

Um die Effektivität ihres vorgeschlagenen Ansatzes zu validieren, führten die Autoren eine empirische Studie mit 50
Studierenden durch, die an einem Softwaretechnikkurs teilnahmen. Die Studierenden wurden gebeten, UML-Diagramme für ein
gegebenes Problem einzureichen, und die Autoren wandten ihren innovativen Ansatz zur Bewertung dieser Einreichungen an.
Um die Zuverlässigkeit der Bewertung sicherzustellen, wurden auch zwei menschliche Prüfer beauftragt, dieselben
Einreichungen unabhängig voneinander zu bewerten.

Die Ergebnisse der Studie zeigen, dass der vorgeschlagene Ansatz mit der menschlichen Bewertung hinsichtlich der
Identifizierung hochwertiger Einreichungen vergleichbar ist und erstaunlich präzise ungefähre Noten vorhersagen kann.
Darüber hinaus verglichen die Autoren ihren Ansatz mit einer komplexen regelbasierten Technik und stellten fest, dass
ihre Methode in Bezug auf Effizienz und Wartbarkeit überlegen ist~\cite{boubekeur2020automatic}.

Zusammenfassend bietet der innovative Ansatz eine Lösung zur Bewertung von UML-Diagrammen im Bereich
der Softwaretechnikausbildung. Durch die Kombination eines einfachen heuristischen Algorithmus mit fortgeschrittenen
maschinellen Lernverfahren identifiziert ihr Ansatz nicht nur hochwertige Einreichungen, sondern bietet auch eine
Möglichkeit zur Vorhersage ungefährer Noten. Die Autoren betonen das potenzielle Zeitersparnis und die Reduzierung der
Subjektivität im Bewertungsprozess und schlagen vor, dass ihr Ansatz auf andere Bereiche erweitert werden kann,
sofern es eine klare Definition einer Idealvorlage gibt und eine Methode existiert, um die Einreichungen der
Studierenden mit dieser Idealvorlage zu vergleichen. Diese Herangehensweise weist jedoch einige zu berücksichtigende
Schwächen auf:

\begin{enumerate}
    \item Es muss ein Zwischenmodell generiert werden, von dem aus die Diagramme der Studierenden bewertet werden,
    was zusätzliche Arbeit für die Lehrkräfte bedeutet~\cite{boubekeur2020automatic}.
    \item Die Herangehensweise funktioniert für allgemeine Diagrammfälle, aber ihre Effektivität nimmt ab, wenn die
    Modellierung komplexer wird~\cite{boubekeur2020automatic}.
    \item Die Autoren sprechen davon, dass die Herangehensweise mit einem begrenzten Datensatz getestet wurde und
    eine größere Menge an annotierten Daten benötigt, um sie zu verbessern, was für das
    Entwicklungsteam zeitintensiv sein kann~\cite{boubekeur2020automatic}.
    \item Das Trainieren und Evaluieren einer künstlichen Intelligenz ist eine mühsame Arbeit, die Jahre dauern kann~\cite{boubekeur2020automatic}.
\end{enumerate}

Die in dieser Masterarbeit behandelte Herangehensweise, wie sie im Konzeptkapitel~\ref{sec:konzept} dargelegt wird,
ermöglicht es hingegen, von einer bereits vorhandenen Musterlösung des Lehrers auszugehen, Regeln zu generieren und
Feedback zu erhalten, um einen Student schnell und ohne Zwischenkomplexität zu bewerten. Diese Musterlösung sollte im
Voraus annotiert werden, um zusätzliche Details hinzuzufügen, die zur Erstellung dieser Regeln erforderlich sind. Dieser
Ansatz ist daher schneller.


\subsection{Bewertung auf Basis von Ähnlichkeitsmaßen}

Die Autoren des Artikels ``A Different Approach on Automated Use Case Diagram Semantic
Assessment''~\cite{fauzan2021different} sind Daniel Siahaan, Siti Rochimah, Reza Fauzan und Evi Triandini. In diesem
Artikel stellen die Autoren eine innovative Methode zur automatischen semantischen Bewertung von
Anwendungsfall-Diagrammen im Rahmen von UML-Diagrammen vor. Das Hauptziel dieser Methode besteht darin, die
Herausforderungen anzugehen, vor denen Lehrkräfte bei der Bewertung von Anwendungsfall-Diagrammen stehen, sowohl in
zwischenmenschlicher als auch intrapersonaler Hinsicht.

Die Autoren verwenden einen semantischen Bewertungsansatz, der in zwei Hauptdimensionen unterteilt ist: Eigenschaft
(property) und Beziehung (relationship). Alle in dieser Bewertung verwendeten Informationen stammen aus Beschriftungen,
die aus einem \ac{XMI}-Dokument~\cite{xmi} übersetzt wurden. Zur Messung der Ähnlichkeit zwischen den Beschriftungen
verwenden die Autoren die Kosinus-Ähnlichkeit und nutzen WuPalmer zur Unterstützung von
WordNet-Suchen~\cite{fauzan2021different,al2017matching}.

Für ihre empirische Studie sammelten die Autoren einen Datensatz aus drei unterschiedlichen Projekten: Outlay, QuickBill
und dem Restaurant Management System (RMS). Diese Projekte umfassen die Finanzaufzeichnung, den Point of Sale und das
Restaurant-Bestellsystem. Den Studierenden wurden Demonstrationen zur Funktionsweise jedes Projekts gezeigt, und anschließend wurden sie
aufgefordert, Anwendungsfall-Diagramme auf der Grundlage ihres Verständnisses der Projekte und der zugehörigen
Terminologie zu erstellen. Insgesamt sammelten die Autoren 36 Anwendungsfall-Diagramme von den Studierenden, wobei
jedes Projekt ein Antwortschlüssel-Diagramm hatte. Somit umfasst der Datensatz insgesamt 39
Anwendungsfall-Diagramme~\cite{fauzan2021different}.

Bevor sie ihre vorgeschlagene Methode bewerteten, etablierten die Autoren einen Goldstandard als Referenz für die
Bewertung. Dieser Goldstandard repräsentiert das durchschnittliche Ergebnis der Expertenbewertungen von
Studentenantworten auf der Grundlage des Antwortschlüssels. Insgesamt beteiligten sich 21 Experten an dieser
Bewertung, die von zwölf verschiedenen Universitäten stammten. Die Mindestanforderung für eine Expertin oder einen
Experten war ein Master-Abschluss in Informatik.

Zur Bewertung ihrer vorgeschlagenen Methode nutzten die Autoren Gwets AC1, ein Maß, das den Grad der
Übereinstimmung zwischen zwei Experten anzeigt. In diesem Fall repräsentiert der erste Experte den Durchschnitt der
Bewertungen der Experten, während der zweite Experte die Bewertung durch die vorgeschlagene Methode darstellt. Die
Autoren übertrugen die Ähnlichkeitswerte der beiden Experten auf eine Skala von 1 bis 5, wobei eine Wertung von eins
ein Bewertungsergebnis von weniger als zwanzig, zwei ein Bewertungsergebnis von zwanzig bis vierzig, drei ein Ergebnis
von vierzig bis sechzig, vier ein Ergebnis von sechzig bis achtzig und fünf ein Ergebnis von mehr als achtzig darstellt.
Anschließend berechneten die Autoren die Übereinstimmung zwischen den beiden Experten unter Verwendung von Gwets
AC1~\cite{fauzan2021different}.

Die Ergebnisse ihrer Bewertung zeigten, dass die vorgeschlagene Methode eine erhebliche Übereinstimmung mit der
Bewertung durch den Experten aufwies. Interessanterweise stellten die Autoren fest, dass ihre vorgeschlagene Methode
eine höhere Übereinstimmung mit dem Experten aufwies als die durchschnittliche Übereinstimmung zwischen den Experten.
Darüber hinaus beobachteten die Autoren, dass Lehrkräfte tendenziell mehr Wert auf Informationen über Eigenschaften
legen als auf Informationen über Beziehungen bei der Bewertung von Anwendungsfall-Diagrammen. Diese Erkenntnis könnte
dazu beitragen, den Bewertungsprozess zu verbessern und eine konsistentere und objektivere Bewertung von
Studentenantworten zu gewährleisten~\cite{fauzan2021different}.

Die semantische Bewertung ist eine Evaluierungsmethode, die sich auf die Bedeutung und das Verständnis von
Informationen konzentriert, anstatt auf ihre Form oder Struktur. Diese Herangehensweise ist für Anwendungsfalldiagramme
geeignet. Dabei wird die semantische Bewertung genutzt, um die Bedeutung und Relevanz der Diagrammelemente wie
Eigenschaften und Beziehungen zu bewerten, anstatt lediglich zu überprüfen, ob sie korrekt beschriftet oder formatiert sind.


Die vom Autorenteam präsentierte Methode wurde bisher noch nicht auf Klassendiagramme angepasst, da Klassendiagramme
eine andere Struktur als Anwendungsfalldiagramme aufweisen. Daher können die relevanten semantischen Informationen für
Klassendiagramme anders sein als für Anwendungsfalldiagramme. Um die Evaluierung von Klassendiagrammen mit dieser
Methode zu ermöglichen, müssten die für die Extraktion semantischer Informationen verwendeten natürlichsprachlichen
Verarbeitungstechniken an die spezifischen semantischen Informationen von Klassendiagrammen angepasst werden~\cite{fauzan2021different}. Dennoch
bedeutet diese Methode zusätzliche Arbeit für den Lehrer. Bei der Erstellung seiner Übung wird er spezifische Wörter
verwenden müssen und eine klar definierte Semantik beherrschen müssen, um dem Evaluierungstool die größtmögliche
Präzision bei der Bewertung der Diagramme zu ermöglichen. Jedoch wurde aus denselben Gründen wie in dem vorherigen
Abschnitt eine schnellere Herangehensweise für die Bewertung von Klassendiagrammen im Rahmen dieser Masterarbeit gewählt.


\subsection{Graph matching zur Bewertung von UML}

Der Artikel mit dem Titel ``New method for summative evaluation of UML class diagrams based on graph similarities''
wurde von Outair Anas, Mohammed Ouadou und Abdelhadi Lotfi verfasst~\cite{anas2021new}. In ihrem Artikel widmen sich die
Autoren der anspruchsvollen Aufgabe der Bewertung von UML-Klassendiagrammen, die von Studierenden erstellt wurden. Diese
Aufgabe ist oft komplex für Lehrende, da UML-Klassendiagramme mehrere gültige Darstellungen haben können. Die Autoren
schlagen ein halbautomatisches System vor, das dieses Problem durch den Einsatz eines Vergleichs von syntaktischen,
strukturellen und semantischen Ähnlichkeiten angeht, um Fehler von Studierenden zu identifizieren und wertvolles
Feedback zu ihrem Lernprozess zu geben. Ihre Methode besteht aus drei Schritten:

\begin{enumerate}
    \item \textbf{UML-Diagramme in Graphen umwandeln:} Die Autoren beginnen ihren Ansatz, indem sie UML-Klassendiagramme
    in Graphdarstellungen umwandeln und das Metamodell durch die Einführung neuer Elemente zur Erleichterung der
    Bewertung verbessern. Sie stellen Klassen als Knoten dar, Attribute als mit Klassen verbundene Knoten und
    Assoziationen als beschriftete Kanten~\cite{auxepaules2015diagram}.

    \item \textbf{Definition von Ähnlichkeitsmaßen:} Die Autoren definieren eine Reihe von Ähnlichkeitsmaßen, die auf
    die transformierten UML-Graphen anwendbar sind. Sie untersuchen verschiedene Techniken zur Graphabstimmung und
    Metriken zur Knotenähnlichkeit, um den Vergleich von Graphen und die Fehlererkennung zu erleichtern~\cite{fauzan2018class}.

    \item \textbf{Abgleich und Vergleich von Graphen:} Unter Verwendung der zuvor definierten Ähnlichkeitsmaße führen
    die Autoren einen Abgleich und Vergleich der von Studierenden erstellten UML-Diagramme und der vom Lehrer
    bereitgestellten Referenzdiagramme durch~\cite{outair2017towards} (Siehe Abbildung~\ref{fig:graph-matching}).
\end{enumerate}


\begin{figure}
	\centering
	\includegraphics[width=15cm]{images/graph-matching}
	\caption{Beispiel für einen Graphenabgleich \cite{anas2021new}.}
	\label{fig:graph-matching}
\end{figure}

Die Leistung des Systems wird anhand eines Datensatzes bewertet, der aus 100 von Studierenden erstellten
Klassendiagrammen besteht, wobei für jedes Diagramm ein entsprechendes Referenzdiagramm zur Bewertung bereitgestellt
wird. Drei Übungen werden ausgewählt, um das System offline zu konfigurieren und zu bewerten, wobei iterative
Verbesserungen an den Ähnlichkeitskriterien und der Systemfunktionalität auf Grundlage der Übungsergebnisse vorgenommen
werden. Die Ergebnisse zeigen, dass das System eine Genauigkeitsrate von 70 \% bei der Erkennung von Fehlern von
Studierenden erreicht und minimalen Eingriff erfordert, um Übereinstimmungen für 80 \% der verarbeiteten Diagramme zu
korrigieren~\cite{anas2021new}. Die Autoren betonen, dass der in das System integrierte formative Bewertungsansatz gut
geeignet ist, um UML-Klassendiagramme zu bewerten, da er den Studierenden Feedback zur Verbesserung ihres Verständnisses
des Lehrstoffes liefert. Darüber hinaus hat das System das Potenzial, die Belastung der Lehrenden zu verringern, indem
es den Bewertungsprozess automatisiert und ihnen ermöglicht, individuelleres Feedback an einzelne Studierende zu geben.

Zusammenfassend bietet die in diesem Artikel vorgestellte Methode eine vielversprechende Möglichkeit zur summativen
Bewertung von UML-Klassendiagrammen unter Berücksichtigung der Möglichkeit mehrerer gültiger Darstellungen. Durch den
umfassenden Vergleich von syntaktischen, strukturellen und semantischen Ähnlichkeiten kann das System Fehler in von
Studierenden erstellten Diagrammen erkennen und zur Verbesserung ihres Lernfortschritts beitragen. Die Autoren erkennen
an, dass weitere Forschung erforderlich ist, um das System zu verfeinern und seine Genauigkeit zu verbessern, aber die
Ergebnisse dieser Studie legen nahe, dass es das Potenzial hat, eine wertvolle Unterstützung sowohl für Lehrende als
auch für Studierende bei der Bewertung von UML-Klassendiagrammen zu sein.

\subsection{Automatisierte Bewertung mit GReQL}\label{subsec:automatisierte-bewertung-mit-greql}

In ihrem Artikel ``Automated checks on UML diagrams'' \cite{striewe2011automated} stellen Michael Striewe und Michael Goedicke eine innovative Technik zur Evaluierung von UML-Klassendiagrammen auf der Grundlage von Graphabfragen vor. Sie betrachten UML-Klassendiagramme als Graphen und formalisieren ihren Ansatz mithilfe einer Graphabfragesprache namens GReQL, die von der JGraLab-Bibliothek bereitgestellt wird.

\ac{GReQL}, ähnlich wie SQL, eignet sich gut für die Implementierung regelbasierter Prüfungen von graphenbasierten Daten \cite{striewe2014automated}. Sie ermöglicht die Abfrage von Elementen bestimmter Typen, die Untersuchung ihrer Verbindungen und die Überprüfung ihrer Attribute. Im Kontext von UML-Klassendiagrammen können diese Abfragen das Vorhandensein von Diagrammelementen wie Klassen, Schnittstellen, Eigenschaften, Operationen, Parametern, Assoziationen und Generalisierungen ermitteln und ihre Beziehungen bewerten \cite{striewe2011automated}.

\begin{lstlisting}[caption={[Codebeispiel] Codebeispiel in GReQL}, label={lst:greql}, float=!ht, language=xml]
  <rule type="presence" points="5">
    <query>
    from x : V{Class},
    y : V{Property}, z : V{PrimitiveType}
           with
              isDefined(x.name) and x.name="A" and
              x --> y and
              isDefined(y.name) and y.name="b" and
              y --> z and
              isDefined(z.name) and z.name="String"
           report 1 end
    </query>
    <feedback>Ein "A" soll ein Attribut für die
    Eigenschaft "b" bereitstellen.</feedback>
  </rule>
\end{lstlisting}

Um ein UML-Diagramm mithilfe dieses Ansatzes zu evaluieren, parsen die Autoren das Diagramm zunächst in eine graphenbasierte Darstellung seiner abstrakten Syntax, in der Regel durch einen XML-Parser. Anschließend verwenden sie \ac{GReQL}-Abfragen, um die Anwesenheit bestimmter Diagrammelemente und ihrer Verbindungen zu überprüfen. Die Autoren stellen eine Reihe von Regeln vor, die als \ac{GReQL}-Abfragen implementiert sind und an die Anforderungen bestimmter Kurse oder Aufgaben angepasst werden können.

In ihrem Bewertungssystem werden einzelnen Regeln individuelle Punktzahlen zugewiesen, wobei unterschiedliche Gewichtungen zur Unterscheidung zwischen Korrektheits- und Qualitätsaspekten verwendet werden. Zum Beispiel könnte eine Regel, die das Vorhandensein einer Klasse mit einem bestimmten Namen sicherstellt, als wichtiger angesehen werden als eine, die das Vorhandensein eines Kommentars zu einer Klasse überprüft. Die Autoren schlagen auch eine Methode zur Berechnung von Gesamtnoten vor, indem die Punktzahlen jeder Regel aggregiert werden \cite{striewe2011automated}.

Um die praktische Anwendung ihres Ansatzes zu zeigen, bieten die Autoren ein Beispiel aus einem UML-Modellierungskurs an, in dem Lösungen von Studierenden automatisch mit ihrer Methodik bewertet wurden. Das Ergebnis dieser Beispielbewertung zeigt die Effektivität ihres Ansatzes bei der Erkennung von Fehlern und der Bereitstellung rechtzeitigen Feedbacks an die Studierenden, wodurch die Arbeitsbelastung der Dozenten reduziert wird \cite{striewe2011automated}.

Über UML hinaus sehen die Autoren die Vielseitigkeit ihres Ansatzes bei der Bewertung anderer Sprachen, die in eine graphenbasierte Darstellung ihrer abstrakten Syntax transformiert werden können, was praktisch alle Programmiersprachen einschließt. Sie sehen seine Anwendung in intelligenten Tutoringsystemen und automatisiertem Tutoring, da es eine sofortige Rückmeldung an Studierende ermöglicht und den Bewertungsprozess für Lehrkräfte vereinfacht.

Mögliche Einschränkungen dieses Ansatzes sind jedoch die Abhängigkeit von der Annahme einer korrekten Umwandlung von UML-Diagrammen in graphenbasierte Darstellungen, die zu Ungenauigkeiten führen kann, wenn der Umwandlungsprozess fehlerhaft ist. Die Autoren schlagen jedoch vor, den Umwandlungsprozess durch eine Reihe von Testfällen zu validieren, um die Zuverlässigkeit sicherzustellen \cite{striewe2011automated}.

Zusätzlich konzentriert sich der Ansatz hauptsächlich darauf, das Vorhandensein von Diagrammelementen und ihren Verbindungen zu überprüfen, bewertet jedoch nicht intrinsisch die Korrektheit des Inhalts innerhalb dieser Elemente. Zum Beispiel kann er überprüfen, ob eine Klasse einen bestimmten Namen hat, aber er überprüft nicht die Richtigkeit der zugehörigen Attribute und Methoden. Die Autoren weisen jedoch darauf hin, dass ihr Ansatz erweitert werden kann, um komplexere Regeln für die Bewertung des Inhalts von Diagrammelementen zu umfassen \cite{striewe2014automated}.

Zusammenfassend bietet der Ansatz von Striewe und Goedicke eine vielversprechende Alternative zur Bewertung von UML-Klassendiagrammen. Indem sie diese Diagramme als Graphen behandeln und \ac{GReQL} für Abfragen nutzen, bieten sie eine flexible und anpassbare Möglichkeit zur Bewertung. Ihre Methodik hat das Potenzial für breitere Anwendungen in verschiedenen Sprachen und Bildungsbereichen und verspricht, das Feedback an Studierende zu verbessern, während sie die Arbeit für Lehrende vereinfacht.


\section{JACK}

Das E-Assessment-System namens ``JACK3'' (dritte Version von JACK)\cite{jack}, entwickelt von der Universität
Duisburg-Essen, stellt eine webbasierte Plattform dar, die Lehrkräfte eine effektive Lösung für die Erstellung und
Durchführung von Online-Übungen bietet. Diese vielseitige Plattform ermöglicht Lehrkräften die Erstellung einer breiten
Palette von Übungen, die nahtlos über eine benutzerfreundliche Oberfläche bereitgestellt werden können. Neben der
Erstellung von Bewertungen verfügt JACK3 über eine Auswahl an Funktionen, die den Bewertungsprozess optimieren sollen.
Diese beinhalten die automatisierte Bewertung, umfassende Berichterstellungswerkzeuge und die Echtzeitverfolgung des
Lernfortschritts der Schüler.

JACK3 bietet eine breite Palette von Übungstypen an, die den unterschiedlichen Lernzielen der Lehrkräfte und den
spezifischen Bedürfnissen ihrer Schüler gerecht werden. Zu den Beispielen gehören Multiple-Choice-Quizfragen,
Wahr/Falsch-Quizfragen, Zuordnungsübungen, Lückentextübungen, Kurzantwortfragen Aufsatzfragen und vieles mehr. Des
Weiteren ermöglicht JACK3 die Erstellung komplexerer Übungen wie Drag-and-Drop-Aufgaben, Hotspot-Übungen, Umfragen zur
Datensammlung und Webquests, die eine tiefgehende Exploration von Themen ermöglichen~\cite{jack}.

Es ist erwähnenswert, dass JACK3 seine Fähigkeiten zur Unterstützung der Bewertung von Übungen zur Unified Modeling
Language (UML) erweitert hat, was seine Anwendbarkeit in der Softwaretechnik-Ausbildung verbessert. Lehrer können das
System nutzen, um UML-Diagramme wie Klassendiagramme, Sequenzdiagramme und Aktivitätsdiagramme automatisch zu bewerten.

Im Bereich der UML-Übungen bietet JACK3 Lehrern die Flexibilität, Übungen zu gestalten, die Klassendiagramme,
Sequenzdiagramme und Aktivitätsdiagramme umfassen. Diese Übungen ermöglichen es den Schülern, die Struktur und das
Verhalten von Softwaresystemen zu visualisieren, die komplexen Beziehungen zwischen Systemkomponenten zu verstehen,
ihre Ideen effektiv an Kommilitonen zu kommunizieren und ihre Fähigkeiten im Bereich Problemlösung und kritisches
Denken zu schärfen~\cite{jack}. JACK3 verwendet die im Abschnitt~\ref{subsec:automatisierte-bewertung-mit-greql}
vorgestellte Methode, um diese verschiedenen Diagrammtypen zu bewerten.


Das Hauptaugenmerk dieser wissenschaftlichen Arbeit liegt auf der Analyse und Evaluierung des Instrumentariums JACK,
insbesondere seiner Kapazität zur kritischen Beurteilung von UML-Diagrammen. Im darauf folgenden Abschnitt wird die
spezifische Problematik in umfassender Weise erörtert und eine breite Palette möglicher Lösungsansätze erörtert.
        \chapter{Problemanalyse}\label{ch:problemanalyse}

In diesem Kapitel erfolgt eine gründliche Analyse der zugrunde liegenden Problematik. Die Identifikation und eingehende Diskussion der spezifischen Herausforderungen und Schwierigkeiten, die im Rahmen dieser Arbeit behandelt werden, stehen dabei im Fokus.

\section{Beschreibung des Bewertungsprozesses}\label{sec:bewertungsprozess}
% Description de l'environnement

Wie bereits in den vorherigen Kapiteln erwähnt, verwendet die Fakultät für Informatik der Universität Duisburg-Essen (UDE) ein elektronisches Bewertungssystem namens ``JACK'' \cite{jack}, um bestimmte Prüfungen und Übungen automatisch zu bewerten. Unter den verschiedenen Arten von Übungen konzentriert sich diese Masterarbeit auf Übungen vom Typ UML.

Eine Übung vom Typ UML-Klassendiagramm besteht darin, die statische Struktur eines Software-Systems mithilfe der UML-Modellierungssprache zu modellieren \cite{reggio2013used}. Den Studierenden wird in der Regel eine Beschreibung des Systems, der wichtigsten Entitäten und ihrer Beziehungen zur Verfügung gestellt, und sie müssen dann ein Klassendiagramm erstellen, das diese Elemente darstellt. In diesem Diagramm sind die Klassen die Hauptobjekte, mit ihren Attributen (Variablen) und Methoden (Funktionen), und die Beziehungen zwischen den Klassen werden durch Assoziationen, Aggregationen oder Kompositionen dargestellt. Die Studierenden müssen besonders auf die Genauigkeit der Namen, Multiplizitäten und Kardinalitäten achten, um die Struktur des Systems korrekt widerzuspiegeln \cite{reggio2013used}. Das Hauptziel dieser Übung ist es, das Verständnis der objektorientierten Modellierungskonzepte zu vertiefen und eine solide Grundlage für die Softwareentwicklung zu schaffen. In Bezug auf das JACK-System kann der Bewertungsprozess in mehrere verschiedene Phasen unterteilt werden:

\vspace{0.5cm}

\textbf{Phase 1: Erstellung der Übung}

In dieser ersten Phase erstellt der Lehrer die Übung, indem er eine schriftliche Beschreibung eines zu modellierenden Systems bereitstellt. Besonderes Augenmerk wird auf die Genauigkeit der Bezeichnungen der Entitäten und die Klarheit und Explizitheit der Beziehungen zwischen ihnen gelegt. Der Lehrer hat dann einen Bereich in der JACK-Anwendung, in den er diesen Text einfügen kann, der von den Studierenden eingesehen wird.

\\~\\
\textbf{Phase 2: Erstellung einer Musterlösung oder Anmerkung (optional)}

In dieser Phase kann der Lehrer entscheiden, ein UML-Diagramm als Musterlösung für die Übung zu erstellen. Dies erleichtert das Verständnis der Übung, ermöglicht die Überprüfung der Kohärenz und Durchführbarkeit des Systems. Alternativ kann der Lehrer die Übung lediglich annotieren, um die Schlüsselbegriffe und -elemente hervorzuheben, die in der späteren Phase nützlich sein werden.

\\~\\
\textbf{Phase 3: Entwicklung des GReQL-Codes}

In dieser Phase verwendet der Lehrer seine Anmerkungen und die Musterlösung, um den GReQL-Code zu entwickeln, der von JACK interpretiert wird, um die von den Studierenden eingereichten Lösungen zu bewerten. JACK verwendet den GReQL-Code sowie verschiedene in dieser Sprache definierte Regeln, um die UML-Diagramme zu bewerten. Wenn ein Student seine Lösung einreicht, wird eine grafische Darstellung dieser Lösung erstellt, und der GReQL-Code führt Abfragen auf dieser Darstellung aus, um eine Note für die Lösung des Studenten zu vergeben \cite{striewe2011automated}.

\\~\\
\textbf{Phase 4: Einreichung der Lösung durch den Studenten}

Nachdem der Student an der Lösung der Übung gearbeitet hat, lädt er ein XML-Dokument im XMI Format auf JACK hoch. Zur Generierung dieses kompatiblen XMI-Dokuments verwenden Studierende Tools wie BOUML \cite{bouml} oder Software Ideas Modeler \cite{sim}, mit denen sie XMI-Code aus einer zuvor erstellten grafischen UML Darstellung ableiten können. Dieses Dokument repräsentiert die von ihm entwickelte Lösung.

\\~\\
\textbf{Phase 5: Bewertung des Diagramms}

Das von den Studierenden eingereichte Diagramm wird anhand des von den Lehrern verfassten GReQL-Codes bewertet. Anschließend wird dem Studenten eine Note zugeteilt.


Diese Phasen veranschaulichen die grundlegende Funktionsweise der JACK-Plattform in Bezug auf die automatisierte Bewertung von Übungen zur Erstellung von UML-Klassendiagrammen.


\section{Untersuchung des Problems}

Im Verlauf der dritten Phase, wie im vorherigen Kapitel beschrieben, sehen sich Lehrende mit der Notwendigkeit konfrontiert, GReQL-Code zu verfassen, der vom JACK-System zur Bewertung der Einreichungen verschiedener Studierender verwendet wird. Diese Phase stellt jedoch bereits auf verschiedenen Ebenen eine Herausforderung für die Lehrenden dar, aus folgenden Gründen:

\vspace{0.5cm}

\textbf{Erforderliche Expertise für die Erstellung von GReQL-Code:}
Das Verfassen von GReQL-Code erfordert eine gewisse Expertise. Auf den ersten Blick mag die Syntax von GReQL nicht schwer zu verstehen sein. Dennoch kann es anspruchsvoll sein, die Feinheiten der Code-Erstellung zu beherrschen. Dies erfordert von den Lehrenden mehrere Stunden, intensives Üben und umfangreiche Tests, um einen Code zu erstellen, der präzise bewertet, insbesondere in Fällen von komplexeren Beziehungen zwischen Entitäten.

\vspace{0.5cm}

\textbf{Hohe Möglichkeit von Fehlern im GReQL-Code:}
Selbst bei Beherrschung der Feinheiten von GReQL sind Lehrende nicht vor möglichen Fehlern gefeit. Diese Fehler können zwar geringfügig erscheinen, haben jedoch das Potenzial, das Bewertungssystem erheblich zu beeinträchtigen.

\vspace{0.5cm}

\textbf{Zeitaufwand für die Erstellung von GReQL-Code:}
Die Bewertung jedes Diagramms erfordert erheblichen Zeitaufwand, um einen GReQL-Bewertungscode zu erstellen, insbesondere für komplexere Beziehungen zwischen verschiedenen Entitäten. Die Zeit, um die erforderliche Expertise zu erlangen, kann stark variieren und erfordert zahlreiche Stunden intensiver Übung.

\vspace{0.5cm}

\textbf{Wartung und Anpassungsfähigkeit des GReQL-Codes:}
Nachdem der GReQL-Code erstellt und für die Bewertung von UML-Diagrammen implementiert wurde, ist kontinuierliche Wartung erforderlich. Mit steigenden Anforderungen an die Bewertung muss der GReQL-Code regelmäßig aktualisiert und angepasst werden. Dies stellt für Lehrende eine fortwährende Herausforderung dar und erfordert Zeit und Aufwand, um sicherzustellen, dass das Bewertungssystem präzise und relevant bleibt.

\vspace{0.5cm}
\textbf{Bedarf an Ressourcen und technischer Unterstützung:}
Lehrende benötigen möglicherweise Zugang zu technischen Ressourcen und angemessener Unterstützung, um die Kunst der effizienten GReQL-Code-Erstellung zu beherrschen. Dies kann Schulungen, Orientierungsdokumente, Diskussionsforen oder andere Formen technischer Unterstützung einschließen. Die Beschaffung dieser Ressourcen kann Zeit in Anspruch nehmen und eine institutionelle Koordination erfordern.

\vspace{0.5cm}
Dies sind die verschiedenen Herausforderungen, die bei der Erstellung von GReQL-Code für die Bewertung von UML-Diagrammen auf der JACK-Plattform auftreten können.


\section{Ableitung und Abgrenzung der Anforderungen}
% Description des choses à faire ainsi qu'une limitation

Das Ziel dieser Masterarbeit besteht nicht darin, Lehrkräfte von der Notwendigkeit zu befreien, GReQL-Code zu schreiben oder zu bearbeiten. Vielmehr geht es darum, den Prozess der Regeldefinition erheblich zu erleichtern. Um diesen Prozess zu vereinfachen, zielt diese Masterarbeit darauf ab, ein Verfahren zu entwickeln und in Form einer Softwareanwendung zu prototypisieren, mit dem (halb-)automatisch Bewertungsregeln aus annotierten Musterlösungen erstellt werden können. Dieses System soll in der Lage sein, folgende Ziele zu erreichen:

\begin{enumerate}[itemsep=8pt, parsep=5pt]
    \item \textbf{Verminderung der Einstiegshürde:} Es strebt danach, den Prozess der Erstellung von GReQL-Regeln zu erleichtern. Selbst jemand, der keine Vorkenntnisse in GReQL hat, sollte unser Tool verwenden können, um Regeln zu erstellen, die bereits eine Bewertung von einfachen Diagrammen und Beziehungen ermöglichen.
    
    \item \textbf{Verbesserung der Präzision und Zuverlässigkeit der GReQL-Regeln:}
    Durch die Nutzung von annotierten Musterlösungen zielt die Anwendung darauf ab, die Präzision und Zuverlässigkeit der generierten Regeln zu verbessern. Dies würde das Risiko von Fehlern und Inkonsistenzen in den Bewertungsregeln verringern und somit eine konsistentere Bewertung der studentischen Arbeiten gewährleisten.
    
    \item \textbf{Optimierung von Zeit und Aufwand der Lehrkräfte:}
    Durch Automatisierung eines Teils des Regelbildungsprozesses zielt das System darauf ab, Zeit und Aufwand der Lehrkräfte zu sparen. Dies würde es ihnen ermöglichen, sich mehr auf das Lehren und Betreuen der Studierenden zu konzentrieren, anstatt auf mühsame administrative Aufgaben.
    
    \item \textbf{Förderung der Skalierbarkeit und Wartbarkeit der Regeln:}
    Durch die Implementierung eines Mechanismus zur Pflege und Aktualisierung der generierten Regeln würde die Anwendung dazu beitragen sicherzustellen, dass die Regeln relevant und anpassungsfähig bleiben und sich den sich ändernden Anforderungen von Lehre und Bewertung anpassen.
    
    \item \textbf{Unterstützung einer breiten Palette von Bewertungsszenarien:}
    Die Anwendung sollte flexibel genug sein, um eine Vielzahl von Bewertungsszenarien zu unterstützen, einschließlich solcher mit komplexen Diagrammen und Beziehungen. Dadurch würde sie eine vielseitige Lösung für Lehrkräfte bieten.
    
\end{enumerate}

Das Ziel dieser Initiative ist es, den Prozess der Bewertung von UML-Diagrammen effizienter, zugänglicher und präziser zu gestalten, während die administrativen Arbeitslasten der Lehrkräfte reduziert werden. Dadurch soll die Qualität von Lehre und Bewertung im Bereich der Softwaremodellierung verbessert werden.

        \chapter{Entwicklung des Konzepts}

In diesem Kapitel wird der Übergang von der Problemanalyse zur kreativen Konzeptentwicklung im Forschungsprozess hervorgehoben. Es markiert die zentrale Phase, in der theoretisches Wissen und praktische Erkenntnisse zusammenfließen, um innovative Lösungsansätze zu formulieren. Wir integrieren theoretische Grundlagen und Modelle aus vorangegangenen Kapiteln und erläutern die angewandten Methoden zur Entwicklung tragfähiger Lösungen. Dieses Kapitel dient als Brücke zwischen Analyse und Umsetzung, betont die kreative Synthese von Ideen und Lösungsansätzen und bietet einen detaillierten Einblick in den Prozess der Konzeptentwicklung.

\section{Verschiedene Herangehensweisen}

Vor der Ausarbeitung des endgültigen Implementierungskonzepts wurden mehrere methodische Ansätze einer eingehenden Untersuchung unterzogen. Das vorrangige Ziel dieser methodischen Herangehensweisen bestand in der Extraktion einer Reihe von Regelwerken, die später in GReQL-Ausdrücke transformiert werden sollten. Diese Regelwerke sollten aus den Kommentaren oder Anmerkungen extrahiert werden, welche von Lehrkräften in Bezug auf eine UML-Übung hinterlassen wurden. Die Wahl des Notenformats ist frei, was viele Möglichkeiten bietet. 

\subsection{YAML-basierten Annotationen}

Der erste untersuchte methodische Ansatz fokussierte auf die Entwicklung eines benutzerfreundlichen Annotationsystems, das eine umfassende Modellierung der Interaktionen innerhalb eines UML-Diagramms ermöglichen sollte. Infolgedessen wurden editierbare Regelobjekte abgeleitet, welche aus diesem Annotationsystem generiert wurden. Sobald der Nutzer sämtliche Regeln, die aus dem Annotationsystem resultierten, verifizierte, konnte er GReQL-Code generieren, welcher anschließend in das JACK-System eingefügt wurde. Die Konzeption dieses schlichten Notationssystems diente dem Zweck der Abbildung von Beziehungen innerhalb eines UML Klassendiagramms. Zu diesem Zweck wurde das YAML-Format aus mehreren Gründen präferiert:

\begin{enumerate}
    \item Erstens, das Schreiben im YAML-Format erweist sich als unkompliziert.
    \item Zweitens, es liefert Daten in einem strukturierten Format, welches leicht manipuliert werden kann.
\end{enumerate}


\begin{figure}
	\centering
	\includegraphics[width=15cm]{images/yaml-uml.png}
	\caption{Beispiel einer Annotation von UML mit YAML}
	\label{fig:yaml-uml}
\end{figure}


Die Notation in YAML mag den Eindruck vermitteln, bereits eine Form der Regeldefinition darzustellen, weil der Lehrende bereits alle Interaktionen zwischen den verschiedenen Objekten im Diagramm definiert. Um jedoch die Klarheit bezüglich dieser Frage zu gewährleisten, könnte es sinnvoll sein, ein weniger formelles Notationssystem zu etablieren. Beispielweise:

\begin{itemize}
    \item Die Klasse A erbt von der Klasse B.
    \item Die Klasse A besitzt drei Attribute:
    \begin{itemize}
        \item Attribut a vom Typ x mit öffentlicher Sichtbarkeit.
        \item Attribut b vom Typ y mit öffentlicher Sichtbarkeit.
        \item Attribut c vom Typ z mit privater Sichtbarkeit.
    \end{itemize}
\end{itemize}

oder ein noch weniger wortreiches System: 

\begin{itemize}
    \item A $\Rightarrow$ B 
    \item A:
    \begin{itemize}
        \item +a:x (+ für öffentlich)
        \item +b:y
        \item -c:z (- für privat)
    \end{itemize}
\end{itemize}


Dieses System ähnelt eher herkömmlichen Kommentaren, jedoch könnte es herausfordernder sein, den darin enthaltenen Text zu analysieren und daraus anwendbare Regelvorschläge zu extrahieren. Trotzdem birgt dieser methodische Ansatz mehrere Limitationen:

\begin{enumerate}
    \item Es existieren bereits seit geraumer Zeit vielfältige Notationssysteme und Datenformate für UML, wie beispielsweise XMI \cite{skogan1999uml}, \ac{UXF} \cite{suzuki1998making}, Textbasierte \cite{washizaki2010tcd} oder JSON-basierte Datenformate \cite{benson2021uml}, um nur einige zu nennen.

    \item Die Entwicklung eines vollständig neuen und umfassenden Notationssystems innerhalb eines begrenzten Zeitraums kann äußerst zeitintensiv sein, insbesondere wenn es die umfassende Darstellung sämtlicher Varianten eines UML-Klassendiagramms anstrebt.

    \item Dieses System verlagert die Aufgabe der Ableitung von Beziehungen zwischen den verschiedenen Objekten lediglich in ein anderes Format, ohne dabei die Arbeitsbelastung für die Lehrkraft zu mindern.
\end{enumerate}


In Anbetracht dieser diversen Überlegungen wurde von der Verfolgung dieses methodischen Ansatzes Abstand genommen, und es erfolgte die Exploration alternativer Vorgehensweisen.


\subsection{Verwendung von Natural Language Processing  Tools}

Die zweite konzeptionelle Idee besteht darin, ein \ac{NLP}-Tool zu verwenden, um die Kommentare oder Notizen des Lehrers zu analysieren. Anschließend kann das NLP-Tool verwendet werden, um Empfehlungen auf der Grundlage einer vorher festgelegten Reihe von Parametern für den Lehrer abzugeben. Diese Parameter könnten eine Reihe von UML-Regeln in einem spezifischen Format sein, so dass die KI in der Lage ist, die richtige Lösung auf der Grundlage des Kommentars zu finden. Der Vorschlag könnte dann in GReQL-Code übersetzt werden. Diese Herangehensweise hat jedoch auch einige Herausforderungen:

\begin{enumerate}
    \item Die von dieser Methode generierten Lösungen neigen dazu, ungenau zu sein, und es sind möglicherweise zahlreiche Anpassungen erforderlich, um das gewünschte Ergebnis zu erzielen \cite{chowdhary2020natural}. Dies könnte möglicherweise die Arbeitsbelastung der Lernenden erhöhen, anstatt sie zu erleichtern.
    \item Es wäre notwendig, die KI darauf zu trainieren, bestimmte Muster zu erkennen und sie mit vordefinierten Regeln in Verbindung zu bringen, um die Zuverlässigkeit zu erhöhen.

\end{enumerate}

In ihrem aktuellen Stand erfordert diese Herangehensweise erhebliche Anstrengungen, um diese Herausforderungen zu bewältigen und eine effektive Implementierung zu erreichen. Aufgrund der potenziellen Schwierigkeiten bei der Implementierung und der möglicherweise nicht zufriedenstellenden Ergebnisse wurde dieser Ansatz auch aufgegeben.

\section{Konzept}\label{sec:konzept}

Bei der Entwicklung des Konzepts wurde ein radikal andersartiger Ansatz als der in dem vorherigen Abschnitt dargelegte verfolgt. Die Herangehensweise, Regeln aus Kommentaren abzuleiten, erweist sich als ein komplexer Prozess und potenziell schwierig umzusetzen. Die Gewährleistung der Zuverlässigkeit der erzielten Ergebnisse stellt ebenfalls eine bedeutende Herausforderung dar. In diesem Zusammenhang wurde eine intuitivere Herangehensweise in Betracht gezogen, nämlich die direkte Ableitung von Regeln aus der Mustervorlage.

Bevor dieses Konzept im Detail beschrieben wird, ist es unerlässlich, das Endziel in Erinnerung zu rufen, nämlich die Unterstützung von Lehrenden bei der Bewertung von UML-Übungsaufgaben (insbesondere Klassendiagrammen). Dies soll durch erhebliche Vereinfachung des Schreibens von GReQL-Code auf der JACK-Plattform erreicht werden, indem Lehrern mit Hilfe eines im Rahmen dieser Masterarbeit zu entwickelnden Tools (halb-)automatische Unterstützung geboten wird. Das Konzept zur Erfüllung dieser Aufgabe kann in drei Wichtige Schritte unterteilt werden:

\subsubsection{Schritt 1: Erstellung einer Musterlösung mit Hilfe eines Modellierungstools}

Wie im \autoref{sec:bewertungsprozess} erwähnt wurde, umfasst Phase 2 die optionalen Schritte zur Erstellung einer Musterlösung. In diesem Konzept ist diese Phase unerlässlich und gewinnt ihre volle Bedeutung. Nachdem ein UML-Klassendiagrammübungsaufgabe entwickelt wurde, erstellen die meisten Lehrer eine Musterlösung, oft in Form eines Klassendiagramms. Diese Lösung wird dann mit den verschiedenen Einreichungen der Studenten verglichen, und Punkte werden gemäß den vom Lehrer festgelegten Kriterien an die einzelnen Studenten vergeben. Da in den meisten Fällen eine Musterlösung in Form eines Klassendiagramms für die Aufgabe vorhanden ist, wäre es sinnvoll, diese zur Ableitung relevanter Regeln zu verwenden. Dies stellt die erste Phase des Ansatzes dar, bei der eine Musterlösung mithilfe einer Modellierungsanwendung erstellt wird, die anschließend die Möglichkeit bietet, das Diagramm in einem leicht programmierbaren Format zu exportieren.

\subsubsection{Schritt 2: Verarbeitung der exportierten Datei}

Nachdem der Lehrer die Musterlösung erfolgreich modelliert hat, ergibt sich die Notwendigkeit, diese Vorlage in ein geeignetes Format zu exportieren, welches eine programmgesteuerte Bearbeitung ermöglicht. Dieser Transformationsprozess kann mittels diverser Datenformate wie JSON, XML oder sogar YAML realisiert werden. Sobald das exportierte Dokument verfügbar ist, eröffnet sich im Kontext eines Klassendiagramms die Möglichkeit, zunächst sämtliche in der Musterlösung enthaltenen Klassen sowie sämtliche Interdependenzen zwischen den verschiedenen Elementen zu extrahieren. Infolge dieses Phasenschritts besteht die Option zur Ableitung von ``Regel''-Objekten, welche auf der Frontend-Oberfläche sichtbar sind und vom Lehrer interaktiv bearbeitet werden können. Die Bereitstellung dieser Funktionalität befähigt den Lehrer dazu, wertvolles Feedback hinzuzufügen, die zugehörigen Punktzahlen zu definieren und sämtliche erforderlichen Informationen für den nächsten Schritt und die anschließende Evaluierung umfassend zu vervollkommnen.

\subsubsection{Schritt 3: Generierung des GReQL-Codes}

Der abschließende Schritt dieses Vorgehens umfasst die Generierung von GReQL-Code aus den Regeln, die in der vorherigen Phase extrahiert und/oder hinzugefügt wurden. Nachdem die Konfiguration abgeschlossen ist, muss der GReQL-Code unter Verwendung verschiedener XMI-Vorlagen generiert werden. Der resultierende Code kann vom Lehrer auf der JACK-Plattform verwendet werden.

\begin{figure}
	\centering
	\includegraphics[width=15cm]{images/concept.png}
	\caption{Repräsentatives schema des konzepts.}
	\label{fig:concept}
\end{figure}

Durch dieses Konzept (siehe \ref{fig:concept}) und seine verschiedenen Schritte ist es möglich, von einer Musterlösung zur Generierung des erforderlichen GReQL-Codes für die Bewertung von Diagrammen durch JACK zu gelangen. Dieser Ansatz hat das Potenzial, den Bewertungsprozess von Diagrammen über die JACK-Plattform erheblich zu vereinfachen, was das zentrale Ziel dieser Masterarbeit ist. Die tatsächliche Umsetzung dieses Konzepts hängt jedoch eng vom gewählten Workflow und der Art und Weise ab, wie das Konzept umgesetzt wird. Dieser Ausblick leitet zum nächsten Kapitel über, das die Implementierung behandelt.
        \chapter{Implementierung}

Das vorliegende Kapitel widmet sich der umfassenden Dokumentation des Implementierungsprozesses des zuvor beschriebenen Konzepts. Es bietet eine detaillierte Aufarbeitung der technischen Umsetzung und des Entwicklungsprozesses, der im Rahmen dieser Masterarbeit durchgeführt wurde. Beginnend mit einer umfassenden Beschreibung der verwendeten Technologien und Werkzeuge sowie einer ausführlichen Begründung für die Wahl dieser spezifischen Technologien, wird dieses Kapitel einen tiefen Einblick in die präzise Umsetzung des Konzepts gewähren. Die Implementierung ist ein entscheidender Schritt zur Realisierung des in den vorherigen Kapiteln skizzierten Ansatzes zur Bewertung von UML-Diagrammen. Durch die Dokumentation dieses Schrittes wird das Verständnis für die technischen Aspekte des Projekts vertieft und ermöglicht eine transparente Darstellung des Entwicklungsprozesses.
        \chapter{Evaluation}

Dieses Kapitel widmet sich einer umfassenden Analyse der Relevanz und Effektivität des GReQL Converters als innovatives
Werkzeug. Spezifisch zielt diese Abschnitt darauf ab, die grundlegende Frage zu beantworten, ob dieses Instrument
tatsächlich im akademischen und pädagogischen Kontext nützlich ist. Um dieses Ziel zu erreichen, ist es von größter
Wichtigkeit, eine systematische Herangehensweise zu verfolgen, die die Anwendung verschiedener Bewertungsmethoden
beinhaltet, um die Vorzüge und Effektivität des GReQL Converters nachzuweisen. Dieses Kapitel behandelt ausführlich die
angewandten Ansätze zur Prüfung des GReQL Converters, die Datensammlungsmethoden und die durchgeführten Analysen zur
Messung seiner Nützlichkeit. Letztendlich geht es darum, empirisch festzustellen, ob dieses Werkzeug konkrete Vorteile
und einen signifikanten Mehrwert für Lehrende und Lernende bietet.

\section{Erreichte Ziele}
todo - write something here

\section{Umfrage zur Bewertung des GReQL Converters}
todo - write something here
        \chapter{Diskussion}

Dieses Kapitel wird in mehrere Abschnitte unterteilt sein. Zunächst wird eine Interpretation der Ergebnisse der
Evaluation vorgenommen, um fundierte Schlussfolgerungen zu ziehen. Anschließend wird eine eingehende Diskussion über
die verschiedenen Herausforderungen geführt, die im Laufe des Entwicklungsprozesses des Tools bewältigt wurden.
Abschließend wird eine umfassende Untersuchung durchgeführt, um die verschiedenen Möglichkeiten zur Verbesserung des
GReQL Converters zu erörtern und sein fortwährendes Potenzial zu bewerten. Diese thematische Struktur zielt darauf ab,
eine ganzheitliche und gründliche Analyse der Leistung, der Herausforderungen und der Verbesserungsaussichten im
Zusammenhang mit diesem Tool bereitzustellen.

\section{Interpretation der Evaluationsergebnisse}
todo - write something here

\section{Herausforderungen während des Entwicklungsprozesses}

Im Verlauf des Entwicklungsprozesses manifestierten sich verschiedene herausfordernde Sachverhalte, die eine gezielte
Entwicklungsdynamik bedingten. Dies wiederum zwang die Notwendigkeit zur Implementierung spezifischer Beschränkungen
oder die Abkehr von bestimmten funktionalen Aspekten.

\subsubsection{PlantText Parser}

Bezüglich des PlantText Parsers sind gewisse Limitationen zu konstatieren. Er ist nicht in der Lage, statische
Attribute, statische Methoden und statische Klassen zu erfassen. Indessen fand rasch eine Lösung in Form eines
Kompromisses Anklang, indem dem Anwender ermöglicht wird, diese Modifikationen manuell im Rahmen des Regel-Editors
vorzunehmen.

\subsubsection{GReQL Engine Optimizer}

Der GReQL Engine Optimizer  verfügt über einen Algorithmus zur Optimierung von Abfragen, um deren Ausführung zu
erleichtern und mögliche Probleme wie die Verwendung undefinierter Variablen, welche eine Abfrage fehlerhaft machen
könnten, zu umgehen. Nichtsdestoweniger kann dieser Optimierer zuweilen Unklarheit in der Abfrageausführung stiften.
Es besteht die Möglichkeit, dass eine Abfrage verfasst wird, die auf den ersten Blick in vollkommen korrektem Einklang
erscheint, jedoch bei der Ausführung vom Optimierer in einer Art und Weise modifiziert wird, welche die Abfrage invalide
werden lässt (Wie es bei einigen Regeln im WIKI der Fall ist~\cite{GReQL-wiki}). Daraus resultiert, dass die GReQL Engine
Fehlermeldungen retourniert. Zur Bewältigung dieser Thematik waren eigens maßgeschneiderte Abfragen erforderlich, welche
verschiedene Prüfungen vor der Ausführung durchführen. In dieser Hinsicht erweist sich die Verwendung des GReQL
Converters als vorteilhaft, indem er ausschließlich valide Abfragen zur Optimierung generiert und dem Nutzer die
Frustration erspart, scheinbar korrekte, aber nicht funktionierende Abfragen manuell zu konzipieren.


\subsubsection{Beschränkung auf BOUML}

Der Kern des GReQL Converters liegt in der Erstellung und Definition von Vorlagen, die für jede Regel festgelegt wurden.
Zur Herstellung dieser Vorlagen war es erforderlich, zunächst ein Diagramm, welches die jeweilige Regel in Anspruch
nimmt, mittels der Software BOUML zu modellieren. Anschließend erfolgte die grafische Darstellung mithilfe des GReQL
Engine, um abschließend die Regel aus der grafischen Darstellung abzuleiten. Diese Vorgehensweise impliziert, dass die
Mehrzahl der in Gebrauch genommenen Regeln ihren Ursprung in einer bildlichen Repräsentation eines Diagramms haben,
welches mithilfe von BOUML erstellt wurde. Dies stellt ein substantielles Problem dar, da die XMI-Repräsentationen der
Diagramme abhängig vom verwendeten Tool variieren. Als Beispiel generiert der Enterprise Architect offensichtlich eine
XMI-Datei, die sich von derjenigen generiert durch BOUML zu unterscheiden scheint. Dies hätte zur Konsequenz, dass die
Mehrzahl der durch den GReQL Converter generierten Regeln ungültig würde, sofern das zu beurteilende Diagramm mittels
eines alternativen Tools geschaffen wurde. Das bedeutet, dass die Auswahl des Tools, das für die Generierung der
Lösungen zur Beurteilung eingesetzt wird, von entscheidender Relevanz ist, was wiederum den GReQL Converter auf eine
spezifische Werkzeugauswahl oder auf die Nutzung von BOUML für die Gestaltung der zu beurteilenden Diagramme beschränkt.

\subsubsection{Primitive Datentypen}

Im Zusammenhang mit den von BOUML generierten Diagrammen ist zu berücksichtigen, dass sie einem ausgewiesenen Standard
entsprechen, nämlich dem UML-Standard 2.3~\cite{OMG_UML_23_Infrastructure}, der von BOUML in Gebrauch genommen wird.
Dieser Standard erkennt jedoch lediglich vier primitive Datentypen, nämlich int, bool, string und
UnlimitedNatural~\cite{OMG_UML_23_Infrastructure}. Diese Beschränkung führt dazu, dass Typen, die im Grundsatz als
primitiv erachtet werden könnten, wie double, float, char und dergleichen, schlichtweg nicht berücksichtigt werden.
Dieses Problem hat zur Folge, dass Typen in GReQL-Abfragen nicht überprüft werden können, sofern sie nicht den Kriterien
des UML 2.3-Standards genügen. In praktischer Konsequenz mussten gewisse Funktionen aufgegeben werden, etwa die
Überprüfung des Rückgabetyps einer Methode oder des Typs einer Variablen (sofern diese nicht gemäß UML 2.3 als primitiv
gelten), da die Repräsentation, die durch die GReQL Engine erzeugt wird (basierend auf dem XMI von BOUML), diese Typen
nicht erkennt und daher nicht darstellen kann. Infolgedessen können derlei Abfragen nicht ausgeführt werden.
\\~\\
Diese genannten Beschränkungen stellen zweifelsohne vielversprechende Ansatzpunkte für eine substantielle Verbesserung
des GReQL Converters dar. Daher wird in dem folgenden Abschnitt eine Diskussion darüber eingeleitet, wie der GReQL
Converter möglicherweise verbessert werden kann, um einige dieser inhärenten Einschränkungen zu überwinden.


\section{Potenziale für Weiterentwicklungen}

Der GReQL Converter, obwohl er vielversprechend ist, verwehrt sich der Illusion der Vollkommenheit. In diversen Domänen
sind signifikante Verbesserungen realisierbar, um seine Effektivität bei der Bewältigung spezifischer Herausforderungen
zu optimieren.

\subsubsection{Hinzufügen neuer Regeln}

Eine solche Möglichkeit zur Verbesserung manifestiert sich in der Erweiterung des Regelkatalogs. Obwohl die Entwicklung
des GReQL Converters bereits eine umfassende Berücksichtigung der Regeln, die der Modellierung von UML-Klassendiagrammen
zugrunde liegen, einschloss, bleiben einige subtile Nuancen unvollständig berücksichtigt. Zum Beispiel wurden keine
Regeln für Assoziationen mit spezifischer Richtung oder für nicht-ausgerichtete Beziehungen integriert. Während die
Assoziation zwischen zwei Klassen betrachtet wird, sofern eine Beziehung zwischen ihnen besteht, erfolgt keine explizite
Erfassung der Ausrichtung dieser Assoziation. Ebenso bleiben die mit Assoziationen verknüpften Rollennamen
unberücksichtigt. Diese und andere Feinheiten könnten zukünftige Erweiterungen des GReQL Converters sein, um das Tool
in Bezug auf die Generierung präziserer GReQL-Regeln zu bereichern.

\subsubsection{Erweiterung bestehender Regeln}

Mehrere Regeln könnten von Verbesserungen profitieren, um mit einer breiteren Palette von Entwurfsvarianten kompatibel
zu sein. Hierfür ist es unerlässlich, den GReQL Converter auf einer signifikanten Anzahl von Diagrammen zu testen, um
seine Grenzen zu ermitteln und sie umfassend anzugehen.


\subsubsection{Erweiterung der Kompatibilität des GReQL Converter}

Wie zuvor erwähnt, ist der von GReQL Converter generierte GReQL-Code derzeit zu 100\% kompatibel mit Lösungsdiagrammen,
die mit BOUML erstellt wurden. Der GReQL Converter wurde jedoch mit Blick auf die Erweiterbarkeit zu anderen
Technologien entwickelt, die die Modellierung von UML-Diagrammen und die Generierung von XMI-Dateien ermöglichen. Die
Erweiterung auf andere Diagramm-Modellierungstools sollte hauptsächlich das Hinzufügen von regelbasierten Vorlagen für
diese Tools und die Möglichkeit zur Auswahl des bevorzugten Tools über die grafische Benutzeroberfläche des GReQL
Converter einschließen. Glücklicherweise sollte diese Aufgabe nicht besonders komplex sein.

\subsubsection{Erweiterung des PlantText-Syntaxparsers zur Erkennung von Syntaxfehlern}
Während des Interviews mit Dr. Michael Striewe wurde ein Verhalten bemerkt. Als er seinen PlantText-Code auf die
Plattform kopierte, konnten nur die Klassendefinitionen Regeln (``Class Definition'') generiert werden, und die
Assoziationsregeln wurden nicht erkannt. Dies lag daran, dass die Klassennamen in den Assoziationen in
Anführungszeichen standen, was der Parser nicht erkennen konnte. Dieses Problem lässt sich dadurch erklären, dass der
Parser solche Ausdrücke nicht als Fehler erkennt. Es gibt mehrere Arten von gültigem PlantText-Code, der jedoch nicht
zwangsläufig vom Parser verarbeitet werden kann. Aus diesem Grund sollte man sich an die Dokumentation halten, um die
Regeln zu schreiben, und auf der Entwicklungsseite einen Weg finden, diese Fehler zu identifizieren und insbesondere den
Benutzer darauf aufmerksam zu machen, dass etwas nicht wie erwartet funktioniert. Das gilt auch für den Fehler, den er
bezüglich des Enums erwähnt hat, der eine spezifische Syntax im GReQL Converter erfordert.

\\~\\

Bei der Entwicklung des GReQL Converter war eine der entscheidenden Überlegungen, ein Tool zu schaffen, das sich im
Laufe der Zeit weiterentwickeln kann. Aus diesem Grund wurde der Code unter Einhaltung von Designprinzipien und
bewährten Praktiken geschrieben, um zukünftigen Entwicklern, die an dem Tool arbeiten, die Orientierung zu erleichtern
und ihre Entwicklererfahrung erheblich zu vereinfachen~\cite{mcconnell2006software}.

        \chapter{Zusammenfassung und Ausblick}

Dieses Kapitel markiert den Abschluss dieser Masterarbeit. In erster Linie wird eine kurze Zusammenfassung der
erreichten Ergebnisse in dieser Arbeit präsentiert. Anschließend wird die Diskussion über potenzielle Ansätze und
Alternativen eröffnet, die zur Lösung der in diesem Kontext aufgeworfenen Problematik erkundet werden könnten.

\section{Zusammenfassung}

In dieser Masterarbeit wurde die Generierung von Feedback-Regeln für UML-Modelle im Kontext von E-Assessment-Systemen
untersucht. Die Untersuchung umfasste die Analyse der Herausforderungen, die im Zusammenhang mit der manuellen
Erstellung von GReQL-Code für die Bewertung von UML-Modellen auftreten können, sowie die Vorstellung des GReQL
Converters als Lösungsansatz. Im dritten Kapitel erfolgte eine detaillierte Analyse der Problematiken, die im
Rahmen der manuellen Erstellung von GReQL-Code auftreten können. Diese Probleme umfassen die notwendige Expertise für
die Erstellung von GReQL-Code, die Komplexität bei der Formulierung von Regelwerken sowie den zeitlichen Aufwand,
der für die manuelle Erstellung von Feedback-Regeln erforderlich ist. Die erwähnten Schwierigkeiten können zu einer
zeitaufwändigen und fehleranfälligen  manuellen Erstellung von Feedback-Regeln führen.

Der GReQL Converter präsentiert sich als Lösung für diese Problematiken. Durch den Einsatz dieses Konverters können
Feedback-Regeln automatisch generiert werden, ohne dass umfassende Kenntnisse im Bereich GReQL-Code erforderlich sind.
Zusätzlich dazu ist der Konverter in der Lage, komplexe Regeln zu formulieren und den zeitlichen Aufwand für die
manuelle Erstellung von Feedback-Regeln zu reduzieren. Die Ergebnisse dieser Untersuchung legen nahe, dass der GReQL
Converter als ein nützliches Instrument für die automatisierte Bewertung von UML-Modellen betrachtet werden kann. Er
trägt dazu bei, den Prozess der Generierung von Feedback-Regeln zu vereinfachen.

In der Gesamtschau der vorliegenden Masterarbeit wird aufgezeigt, dass die Nutzung des GReQL Converters eine wirksame
Lösung für die Herausforderungen bei der manuellen Erstellung von Feedback-Regeln für UML-Modelle darstellt. Die hier
erzielten Erkenntnisse haben das Potenzial, den Einsatz von E-Assessment-Systemen zu optimieren und den Lernprozess für
Lehrende und Lernende zu verbessern.

\section{Ausblick}

Für die Entwicklung des GReQL Converters wurde ursprünglich PlantText aufgrund seiner Fähigkeit gewählt, mithilfe des
PlantUML Parser Beziehungen und Entitäten aus einem UML-Diagramm zu extrahieren und sie in einem leicht verwertbaren
JSON-Format zu exportieren. Es gibt jedoch mehrere andere, komplexere Strategien, die in Erwägung gezogen werden können.
Aktuell erfordert die Verwendung des GReQL Converters die Anwendung von PlantText, während für die Erstellung des zu
bewertenden Diagramms auf der JACK-Seite die Generierung einer XMI-Datei mit BOUML erforderlich ist. Dies führt zu einer
gewissen Heterogenität im Arbeitsablauf. Die Erkundung der Möglichkeit, Regeln direkt aus einer XMI-Datei zu generieren,
könnte ein interessanter Entwicklungsbereich für den GReQL Converter sein. Dies wäre jedoch aufgrund der Entwicklung
eines angepassten Parsers, die bereits eine recht komplexe Aufgabe darstellt, anspruchsvoll. Dennoch würde dies
zweifellos einen Mehrwert für den Entwicklungsprozess bieten und es ermöglichen, aus einer Lösung, die beispielsweise
in BOUML erstellt wurde, nicht nur GReQL-Regeln zu generieren, sondern sie auch direkt zu überprüfen. Der Benutzer
müsste keine zwei unterschiedlichen Technologien erlernen, um diese Aufgabe zu erfüllen.

Ein weiterer Entwicklungsbereich, der von Anfang an aufgegeben wurde, aber dennoch von Interesse sein könnte, ist die
Integration eines KI-Algorithmus. Ein solcher Algorithmus könnte dazu befähigt sein, ein zuvor annotiertes
UML-Übungsszenario zu lesen und daraus abzuleiten, welche Regeln generiert werden sollten. Eine ähnliche Aufgabe wurde
bereits in einem anderen Kontext von Mohammed Amraouy et al.\cite{amraouy2023sentiment} realisiert, bei dem mithilfe
von Studentenkommentaren auf einer E-Assessment-Plattform eine Analyse und Klassifizierung der Emotionen durchgeführt
wurde, um das Engagement der Studenten in einem bestimmten Kurs zu bewerten. In einem Kontext, der dem dieser
Masterarbeit ähnlicher ist, ist auch die Arbeit von Aggarwal et al.\cite{aggarwal2018machine} relevant, die verschiedene
Algorithmen des maschinellen Lernens verwenden, um verschiedene Entitäten auf der Grundlage von Text in Kategorien zu
klassifizieren.

Da das Tool nicht vollständig automatisiert werden kann, ist nach wie vor das Eingreifen eines Benutzers erforderlich,
um die Richtigkeit und Konsistenz der generierten Regeln zu überprüfen.
        %\chapter{Spielwiese TODO wieder entfernen!} \label{spielwiese}

Zwölf Boxkämpfer jagen Viktor quer über den großen Sylter ``Deich''.

Lorem  ipsum dolor sit amet, consectetur adipisici elit, sed eiusmod tempor incidunt ut labore et dolore magna aliqua. Ut enim ad minim veniam, quis nostrud exercitation ullamco laboris nisi ut aliquid ex ea commodi consequat. Quis aute iure reprehenderit in voluptate velit esse cillum dolore eu fugiat nulla pariatur. Excepteur sint obcaecat cupiditat non proident\footnote{this is an url \url{https://wikipedia.de}}, sunt in culpa qui officia deserunt mollit anim id est laborum. 

\section{Abschnitt Eins - Gleichungen}

\begin{equation} \label{eq:relativity}
e = m \cdot c^{2}
\end{equation}

Lorem ipsum dolor sit amet, \autoref{eq:relativity} consectetur adipisici elit, sed eiusmod tempor incidunt ut labore et dolore magna aliqua. Ut enim ad minim veniam, quis nostrud exercitation ullamco $m=\frac{1 \cdot n}{2k!}$ laboris nisi ut aliquid ex ea commodi consequat. Quis aute iure reprehenderit in voluptate velit esse cillum dolore eu fugiat nulla pariatur. Excepteur sint obcaecat cupiditat non proident, sunt in culpa qui officia deserunt mollit anim id est laborum.

$$ e=\frac{a}{b}$$

\subsection{Unterabschnitt Eins}
Lorem ipsum dolor sit amet, consectetur adipisici elit, sed eiusmod tempor incidunt ut labore et dolore magna aliqua. Ut enim ad minim veniam, quis nostrud exercitation ullamco laboris nisi ut aliquid ex ea commodi consequat. Quis aute iure reprehenderit in voluptate velit esse cillum dolore eu fugiat nulla pariatur. Excepteur sint obcaecat cupiditat non proident, sunt in culpa qui officia deserunt mollit anim id est laborum.

\section{Abschnitt Zwei}

\begin{figure}
	\centering
	\includegraphics[width=8cm]{Plot}
	\caption[Kurzbeschreibung]{Die Graphik zeigt gedöns.}
	\label{fig:labelForPlot}
\end{figure}


Lorem ipsum dolor sit amet, consectetur adipisici elit, sed eiusmod tempor incidunt ut labore et dolore magna aliqua. Ut enim ad minim veniam, quis nostrud exercitation ullamco laboris nisi ut aliquid ex ea commodi consequat. Quis aute iure reprehenderit in voluptate velit esse cillum dolore eu fugiat nulla pariatur. Excepteur sint obcaecat cupiditat non proident, sunt in culpa qui officia deserunt mollit anim id est laborum. \autoref{fig:subPlot1} 

Lorem ipsum dolor sit amet, consectetur adipisici elit, sed eiusmod tempor incidunt ut labore et dolore magna aliqua. Ut enim ad minim veniam, quis nostrud exercitation ullamco laboris nisi ut aliquid ex ea commodi consequat. Quis aute iure reprehenderit in voluptate velit esse cillum dolore eu fugiat nulla pariatur. Excepteur sint obcaecat cupiditat non proident, sunt in culpa qui officia deserunt mollit anim id est laborum. \autoref{fig:labelForPlot}.

\subsection{Unterabschnitt Eins}

Lorem ipsum dolor sit amet, consectetur adipisici elit, sed eiusmod tempor incidunt ut labore et dolore magna aliqua. Ut enim ad minim veniam, quis nostrud exercitation ullamco laboris nisi ut aliquid ex ea commodi consequat. Quis aute iure reprehenderit in voluptate velit esse cillum dolore eu fugiat nulla pariatur. Excepteur sint obcaecat cupiditat non proident, sunt in culpa qui officia deserunt mollit anim id est laborum. \autoref{fig:subPlot2}

Lorem ipsum dolor sit amet, consectetur adipisici elit, sed eiusmod tempor incidunt ut labore et dolore magna aliqua. Ut enim ad minim veniam, quis nostrud exercitation ullamco laboris nisi ut aliquid ex ea commodi consequat. Quis aute iure reprehenderit in voluptate velit esse cillum dolore eu fugiat nulla pariatur. Excepteur sint obcaecat cupiditat non proident, sunt in culpa qui officia deserunt mollit anim id est laborum. \autoref{fig:subPlotAll}

\begin{figure}
	\centering
	\begin{subfigure}{.49\textwidth}
		\includegraphics[width=\textwidth]{Plot}
		\caption[Kurzbeschreibung]{Die Graphik zeigt gedöns a.}
		\label{fig:subPlot1}
	\end{subfigure}
	\hfill
	\begin{subfigure}{.49\textwidth}
		\includegraphics[width=\textwidth]{Plot}
		\caption[Kurzbeschreibung]{Die Graphik zeigt gedöns b.}
		\label{fig:subPlot2}
	\end{subfigure}
	\caption[Kurzbeschreibung Gedöns All]{Die Graphik zeigt gedöns insgesammt.}
	\label{fig:subPlotAll}
\end{figure}

\section[Kurztitel 3]{Abschnitt Drei - Abkürzungen}
Lorem  ipsum dolor sit amet, consectetur adipisici elit, sed eiusmod  tempor incidunt ut labore et dolore magna aliqua. Ut enim ad minim veniam, quis nostrud exercitation ullamco laboris nisi ut aliquid ex ea commodi consequat. Quis aute iure reprehenderit in voluptate velit esse cillum dolore eu fugiat nulla pariatur. Excepteur sint obcaecat cupiditat non proident,  sunt in culpa qui officia deserunt mollit anim id est laborum. 


Dieser Teil kann genutzt werden, um Bild-, Tabellen- und Code-Elemente herauszukopieren.
Generell für bessere Schreib/ Debug Übersicht immer nur einen Satz pro angefangene Zeile schreiben (\mono{Enter} drücken nach jedem Punkt).
\LaTeX~rendert das trotzdem als Fließtext. \todo{Das ist eine ToDo-Nachricht}

\vspace{3cm}

% Normalen Absatz einfach mit leeren Absätzen trennen.
Lorem ipsum dolor sit amet, consetetur sadipscing elitr, sed diam nonumy eirmod tempor invidunt ut labore et dolore magna aliquyam erat, sed diam voluptua. 

At vero eos et accusam et justo duo dolores et ea rebum. Stet clita kasd gubergren, no sea takimata sanctus est Lorem ipsum dolor sit amet.


Abschnitte referenzieren können wir wie hier: \autoref{spielwiese} oder mit dem Namen \nameref{spielwiese}.
\enquote{Anführungszeichen}.
Direktes Zitat: \enquote{The toaster is the greatest invention since sliced bread} %\parencite[14]{einstein}.
Blocktext:

\blockquote{
Lorem ipsum dolor sit amet, consetetur sadipscing elitr, sed diam nonumy eirmod tempor invidunt ut labore et dolore magna aliquyam erat, sed diam voluptua.
At vero eos et accusam et justo duo dolores et ea rebum. Stet clita kasd gubergren, no sea takimata sanctus est Lorem ipsum dolor sit amet.
}

Wenn man Abkürzungen wie verwendet, werden diese automatisch im Verzeichnis gelistet, verlinkt und beim zweiten Mal wird nur noch kurz geschrieben. 
Fußnote\footnote{Hier steht Text. \url{https://www.google.com/}, aufgerufen am \today.}.
Url ohne Extras als Fußnote\footurl{https://example.com}.

Liste:
\begin{enumerate}
    \item Item
    \item Item
\end{enumerate}

Unnummerierte Liste:
\begin{itemize}
    \item Item
    \item Item
\end{itemize}

\section{Quellcode}

Im Fließtext \mono{wie hier}.
Als Block wie in \autoref{spielwiese:code}.

% https://de.overleaf.com/learn/latex/Code_listing
\begin{lstlisting}[caption={[Codebeispiel]Codebeispiel mit Hello World in Java}, label=spielwiese:code, float=!ht, language=java]
public class HelloWorld {
    public static void main (String[] args) {
        System.out.println("Hello World!");
    }
}
\end{lstlisting}

\section{Formeln}

Dies ist eine inline-Formel: $a^2+b^2=c^2$ es geht aber auch eine Formel in der ganzen Zeile:
%
$$a^2+b^2=c^2$$
%
Alternativ kann auch die \mono{align}-Umgebung verwendet werden.
Hier wird am Gleichheitszeichen ausgerichtet: % https://de.overleaf.com/learn/latex/Aligning_equations_with_amsmath
%
\begin{align*} % ohne Stern falls nummeriert
2x - 5y &=  8 \\ 
3x + 9y + z &=  -12 + 14
\end{align*}
%
Oder mit Nummerierung und unausgerichtet wie in \ref{spielwiese:formel-gather2}:
%
\begin{gather} % man beachte, dass gather nicht ausrichtet. gather* nutzen für unnummeriert
2x - 5y = 8 \label{spielwiese:formel-gather1} \\ 
3x^2 + 9y = 3a + c \notag \\ % \notag führt dazu, dass hier keine Zahl steht
8x = 8 \label{spielwiese:formel-gather2}
\end{gather}
%
Es geht auch mit Sub-Nummerierung wie folgt:
\begin{subequations} \label{spielwiese:formel}
	\begin{gather}
		a^2+b^2=c^2 \label{spielwiese:formel:a} \\
		e=mc^2 \label{spielwiese:formel:b}
	\end{gather}
\end{subequations}
%
Referenzieren kann man diese einfach normal: \autoref{spielwiese:formel} oder \ref{spielwiese:formel:b}.

\section{Bilder}

Wie wir hier sehen, können wir Abbildungen referenzieren (\scvgl \autoref{spielwiese:testbild}).
Wir können auch nur die Nummer angeben, \sczb \ref{spielwiese:testbild}.
Es ist auch möglich, Grafiken nebeneinanderzustellen, wie \sczb in \autoref{spielwiese:fragment}, die in \autoref{spielwiese:fragment1} und \autoref{spielwiese:fragment2} aufgeteilt ist.
Nur die \ref{spielwiese:fragment} erscheint dabei im Verzeichnis.
Zwei gleichwertige Grafiken nebeneinander sind in \autoref{spielwiese:vollwertig1} und \autoref{spielwiese:vollwertig2} zu finden.
Beide Grafiken erscheinen dann als \ref{spielwiese:vollwertig1} und \ref{spielwiese:vollwertig2} im Verzeichnis.

% https://de.overleaf.com/learn/latex/Positioning_of_Figures
% h = here, t = top, b = bottom, ! = überschreibe Parameter für "guten Stil"
\begin{figure}[!ht]
	\centering
    % z.B. "width=0.5\textwidth" für halbe Seitenbreite, "width=\textwidth" für gesamte Seitenbreite
    % "height=3cm" wäre z.B. ebenso möglich
	\includegraphics[width=\textwidth]{images/test.jpg}  % 
    % in eckigen Klammern der Text, wie er im Verzeichnis erscheint, in geschweiften Klammern der Text, wie er in der Unterschrift erscheint
	\caption[Testbild]{Dies ist ein Testbild.}
	\label{spielwiese:testbild}
\end{figure}

% https://texfragen.de/bilder_nebeneinander
% h = here, t = top, b = bottom, ! = überschreibe Parameter für "guten Stil"
\begin{figure}[!ht]
	\centering
    % Insgesamt sollten die addierten Breiten 0.95 nicht überschreiten.
	\begin{subfigure}{.45\textwidth} % Breite 1
		\includegraphics[width=\textwidth]{images/test.jpg} % hier volle Größe angeben, wird durch subfigure-Befehl gesteuert
		\caption{Erstes Fragment}
		\label{spielwiese:fragment1}
	\end{subfigure}
	\hspace{.05\textwidth} % kann zusätzlich eingefügt werden, um Platz aufzufüllen
	\begin{subfigure}{.45\textwidth} % Breite 2
		\includegraphics[width=\textwidth]{images/test.jpg} % hier volle Größe angeben, wird durch subfigure-Befehl gesteuert
		\caption{Zweites Fragment}
		\label{spielwiese:fragment2}
	\end{subfigure}
	\caption[Bild mit Fragmenten]{Dies ist eine Grafik mit zwei Fragmenten.}
	\label{spielwiese:fragment}
\end{figure}

% Erklärung siehe oben
\begin{figure}[!ht]
    \centering
    \begin{minipage}[b]{.45\textwidth} % [b] => Ausrichtung an \caption
        \includegraphics[width=\textwidth]{images/test.jpg}
        \caption[Vollwertige Grafiken (1)]{Erste Grafik}
        \label{spielwiese:vollwertig1}
    \end{minipage}
    \hspace{.05\textwidth}
    \begin{minipage}[b]{.45\textwidth} % [b] => Ausrichtung an \caption
        \includegraphics[width=\textwidth]{images/test.jpg}
        \caption[Vollwertige Grafiken (2)]{Zweite Grafik}
        \label{spielwiese:vollwertig2}
    \end{minipage}
\end{figure}

\clearpage

\section{Tabellen}

Im Folgenden einfache Tabellen, wie \sczb \autoref{spielwiese:tabelle1}:

\begin{table}[!ht]
	\centering
	\caption{Kleine Tabelle}
	\label{spielwiese:tabelle1}
	\begin{tabular}{|l|c|l|} % | = Linie, l = linksbündig, c = zentriert, r = rechtsbündig
		\hline Text & Text & Text Text \\\hline
		Text & Text Text & Text \\\hline
	\end{tabular}
\end{table}

\begin{table}[!ht]
	\centering
	\caption{Kleine Tabelle mit fester Spaltenbreite}
	\label{spielwiese:tabelle2}
	\begin{tabular}{|l|C{3cm}|R{5cm}|} % L, C und R für linksbündig, zentriert, rechtsbündig mit fester Spaltenbreite
		\hline Text & Text & Text Text \\\hline
		Text & Text Text & Text \\\hline
	\end{tabular}
\end{table}

\begin{table}[!ht]
	\centering
	\caption{Kleine Tabelle mit Blocksatz}
	\label{spielwiese:tabelle3}
	\begin{tabular}{|l|L{3cm}|P{5cm}|} % P = Blocksatz, Text wird umgebrochen
		\hline Text & Text & Text Text \\\hline
		Text Text & Text Text & Dieser lange Text wird umgebrochen. Dieser lange Text wird umgebrochen. \\\hline
	\end{tabular}
\end{table}

\begin{tabularx}{\textwidth}{|l|X|} % X = Spalte bis zum Rand auffüllen
	\caption{Tabelle mit aufgefüllter Spalte} \label{spielwiese:tabelle4} \\\hline \endhead
    \hline Text & Text \\ \hline
    Text Text & Dieser lange Text wird umgebrochen und aufgefüllt.
    Dieser lange Text wird umgebrochen und aufgefüllt.
    Dieser lange Text wird umgebrochen und aufgefüllt. \\ \hline
\end{tabularx}

\begin{table}[!t]
    \centering
    \caption{Tabelle mit Multi-Columns}
    \label{spielwiese:multicoltabelle}
    \begin{tabular}{p{1cm}llllcp{2cm}}
        \toprule%
        \multirow{2}{*}{$P_{ID}$} & \multicolumn{4}{c}{Order of Application}\\%
        \cmidrule{2-5}%
        {} & 1 & 2 & 3 & 4\\%
        \midrule%
        \addlinespace
        0 & a & b & c & d\\%
        1 & b & c & d & a\\%
        2 & c & d & a & b\\%
        3 & $x_{1}$ & $x_{2}$ & $x_{3}$ & $x_{4}$\\%
        \bottomrule%
    \end{tabular}
\end{table}


\autoref{spielwiese:langetabelle} erstreckt sich sogar auf mehrere Seiten:

% \begin{longtblr}[
% 	caption= {Lange Überschrift für eine mehrseitige Tabelle. Lange Überschrift für eine mehrseitige Tabelle. Lange Überschrift für eine mehrseitige Tabelle.},
%     entry = {Mehrseitige Tabelle},
% 	label = {spielwiese:langetabelle},
% 	note{a} = {Fußnote, hier geht die Linie nicht ganz durch}
% ]{width=\linewidth, rowhead=1, colspec={ll|X[l]|}} % x ist Auto-Breite
% 	\hline \SetRow{c, font=\bfseries} Spalte 1 & Spalte 2 & Spalte 3 \\\hline
%     Text & Text & Text \\\hline
%     Text & Text & Text \\\hline
%     Text & Text & Text \\\hline
%     Text\TblrNote{a} & Text & Text \\\cline{2-3}
%     Text & Text & Text \\\hline
%     Text & Text & Text \\\hline
%     Text & Text & Text \\\hline
%     Text & Text & Text \\\hline
%     Text & Text & Text \\\hline
%     Text & Text & Text \\\hline
%     Text & Text & Text \\\hline
%     Text & Text & Text \\\hline
%     Text & Text & Text \\\hline
%     Text & Text & Text \\\hline
%     Text & Text & Text \\\hline
%     Text & Text & Text \\\hline
%     Text & Text & Text \\\hline
%     Text & Text & Text \\\hline
%     Text & Text & Text \\\hline
%     Text & Text & Text \\\hline
%     Text & Text & Text \\\hline
%     Text & Text & Text \\\hline
%     Text & Text & Text \\\hline
%     Text & Text & Text \\\hline
%     Text & Text & Text \\\hline
%     Text & Text & Text \\\hline
%     Text & Text & Text \\\hline
%     Text & Text & Text \\\hline
%     Text & Text & Text \\\hline
%     Text & Text & Text \\\hline
%     Text & Text & Text \\\hline
%     Text & Text & Text \\\hline
%     Text & Text & Text \\\hline
% \end{longtblr}

\begin{tabularx}{\textwidth}{L{2cm}L{2cm}X}
	% Erster Kopf
	\caption[Mehrseitige Tabelle]{Lange Überschrift für eine mehrseitige Tabelle} \label{spielwiese:langetabelle} \\ \hline
	\textbf{Spalte 1} & \textbf{Spalte 2} & \textbf{Spalte 3} \\ \hline \endfirsthead
	% Zweiter Kopf
	\multicolumn{3}{c}{\small{\tablename\ \thetable{} (Fortsetzung)}} \\ \hline
	\textbf{Spalte 1} & \textbf{Spalte 2} & \textbf{Spalte 3} \\ \hline \endhead
	% Beginn Tabelle
    Text Text Text Text Text Text Text & Text Text Text Text Text Text Text & Text Text Text Text Text Text Text Text Text Text Text \\\hline
    Text & Text & Text \\\hline
    Text & Text & Text \\\hline
    Text & Text & Text \\\cline{2-3}
    Text & Text & Text \\\cline{2-3}
    Text & Text & Text \\\hline
    Text & Text & Text \\\hline
    Text & Text & Text \\\hline
    Text & Text & Text \\\hline
    Text & Text & Text \\\hline
    Text & Text & Text \\\hline
    Text & Text & Text \\\hline
    Text & Text & Text \\\hline
\end{tabularx}

	
	\printbibliography[heading=bibintoc]
	
	\appendix
	% !TeX root = main.tex

\chapter{Appendix}
\section{Einige Teile des Quellcode}
\subsection{Backend Quellcode}

\begin{lstlisting}[caption={Node/Express Backend Quelltext}, label={lst:bakcend}, language=javascript]
const express = require('express');
const bodyParser = require('body-parser');
const { parse } = require('plantuml-parser');
const cors = require('cors');
const app = express();
const port = 3000;
app.use(cors());
app.use(bodyParser.json());
app.post('/convert', (req, res) => {
  const code = req.body.code;
  try {
    const parsedCode = parse(code);
    if(parsedCode.length === 0)
       throw new Error('Failed to parse PlantUML code')

    res.json(parsedCode);
  } catch (error) {
    res.status(500)
.json({ error: 'Failed to parse PlantUML code' });
  }
});
app.listen(port, () => {
  console.log(`Server is running on port: ${port}`);
});
\end{lstlisting}


\subsection{Frontend Quellcode}
\subsubsection{Rules Definition JSON}
\begin{lstlisting}[caption={Rules Definition JSON}, label={lst:rules_def}, language=javascript]
export default {
    RULE_TYPE: {
        // CLASS & INTERFACE
        'defined_class': 'defined_class_rule',
        'defined_enum': 'defined_enum_rule',

        // GENERALIZATION & SPECIALIZATION
        'generalization': 'generalization_rule',

        // RELATIONSHIPS
        'simple_association': 'simple_association_rule',
        'composition': 'composition_rule',
        'aggregation': 'aggregation_rule',

        // ASSOCIATION CLASS
        'association_class': 'association_class_rule',

        // OPTIONAL
        'nomination_consistency': 'nomination
_consistency_rule',
        'test_association': 'test_association_rule',
        'count_methods': 'count_methods_rule',
        'count_attributes' : 'count_attributes_rule'

    },
    RULE_TYPE_JSON: {
        // CLASS & INTERFACE
        'defined_class_rule' : {
            rule_type: 'defined_class_rule',
            rule_name: 'Class definition',
            feedback: '... no feedback yet',
            points: 0,
            existence: 'presence',
            rule_specific: {
                class_name: "Car",
                abstract: false,
                interface: false,
                methods: [],
                attributes: [],
            }
        },
        // ENUM
        'defined_enum_rule' : {
            rule_type: 'defined_enum_rule',
            rule_name: 'Enum definition',
            feedback: '... no feedback',
            points: 0,
            existence: 'presence',
            rule_specific: {
                enum_class_name: "Car",
                attributes: [],
            }
        },
        // GENERALIZATION & SPECIALIZATION
        'generalization_rule' : {
            rule_type: "generalization_rule",
            rule_name: "Generalization",
            existence: "presence",
            points: 0,
            feedback: '... no feedback',
            rule_specific: {
                class_child: "Child",
                class_parent: "Parent",
                type: "inheritance" // implementation
            }
        },
        // RELATIONSHIPS
        'simple_association_rule': {
            rule_type: "simple_association_rule",
            rule_name: "Simple Association",
            existence: "presence",
            points: 0,
            feedback: "... no feedback",
            rule_specific: {
                class_A: "Class A",
                class_B: "Class B",
                A_multiplicity: "1",
                B_multiplicity: "1"
            }
        },
        'composition_rule' : {
            rule_type: "composition_rule",
            rule_name: "Composition",
            existence: "presence",
            points: 0,
            feedback: "... no feedback",
            rule_specific: {
                class_composite: "Composite",
                class_element: "Element",
                composite_multiplicity: "1",
                element_multiplicity: "*",
            }
        },
        'aggregation_rule' : {
            rule_type: "aggregation_rule",
            rule_name: "Aggregation",
            existence: "presence",
            points: 0,
            feedback: "... no feedback",
            rule_specific: {
                class_aggregate: "Aggregate",
                class_element: "Element",
                aggregate_multiplicity: "1",
                element_multiplicity: "*",
            }
        },
        // ASSOCIATION CLASS
        'association_class_rule' : {
            rule_type: "association_class_rule",
            rule_name: "Association Class",
            existence: "presence",
            points: 0,
            feedback: "Es muss eine Asso...",
            rule_specific: {
                class_A: "Class A",
                class_B: "Class B",
                class_C: "Class C"
            }
        },
        // OPTIONAL
        'nomination_consistency_rule' : {
            rule_type: "nomination_consistency_rule",
            rule_name: "Nomination Consistency",
        },
        'count_methods_rule' : {
            rule_type: "count_methods_rule",
            rule_name: "Count Methods",
            existence: 'absence',
            points: 0,
            rule_specific: {
                methods: 0,
            }
        },
        'count_attributes_rule' : {
            rule_type: "count_attributes_rule",
            rule_name: "Count Attributes",
            existence: 'absence',
            points: 0,
            rule_specific: {
                attributes: 0,
            }
        },
        'test_association_rule' : {
            rule_type: "test_association_rule",
            rule_name: "Test Association",
            existence: "absence",
            points: 0,
            rule_specific: {
                class_A: "Class A",
                class_B: "Class B",
            }
        },
    },
    METHODS_TYPE: {
        name: "public_method_name",
        return_type: "void",
        visibility: "public",
        arguments: "",
        points: 0,
        feedback: '... no feedback',
        is_static: false
    },
    ATTRIBUTE_TYPE: {
        name: "attribute_name",
        type: "string",
        visibility: "public",
        points: 0,
        feedback: '... no feedback',
        is_static: false
    },
    ENUM_ATTRIBUTE_TYPE: {
        name: "ENUM_ATTR",
        points: 0,
        feedback: '... no feedback',
    },
    EXISTENCE_TYPE: {
        'presence': 'presence',
        'absence' : 'absence'
    },
    GENERALIZATION_TYPE: {
        'inheritance': 'inheritance',
        'implementation': 'implementation'
    }
}
\end{lstlisting}

\newpage

\subsubsection{GReQL Code Generierung}
\begin{lstlisting}[caption={Fall der Class Definition Rule}, label={lst:class_def_rule}, language=javascript]
generateDefineClassRule: function (rule) {
    const isInterface = rule.rule_specific.interface
    let code = ""
    if (isInterface) {
        code += "<!-- Interface Definition -->"
        code += `<rule type="${rule.existence}"
        points="${rule.points}">
        <query>from x : V{Interface}
               with
                  isDefined(x.name) and
                  stringLevenshteinDistance(x.name,
                  "${rule.rule_specific.class_name}")&lt;3
              report 1 end
        </query>
        <feedback>${rule.feedback}</feedback>
     </rule>`
    } else {
        const isAbstract = rule.rule_specific.abstract
        let abstractCode
        if (isAbstract)
            abstractCode = `and x.isAbstract`
        else
            abstractCode = `and (not x.isAbstract)`
        code += "<!-- Class Definition -->"
        code += `<rule type="${rule.existence}"
        points="${rule.points}">
        <query>from x : V{Class}
               with
                  isDefined(x.name) and
                  stringLevenshteinDistance(x.name,
                  "${rule.rule_specific.class_name}")&lt;3
                  ${abstractCode}
               report 1 end
        </query>
        <feedback>${rule.feedback}</feedback>
    </rule>`
    }

    if(rule.rule_specific.attributes.length !== 0){
    rule.rule_specific.attributes.forEach(attribute => {
    code += this.generateAttributeRule(rule, attribute)
    })}

    if(rule.rule_specific.methods.length !== 0){
    rule.rule_specific.methods.forEach(method => {
    code += this.generateMethodRule(rule, method)
    })}

    return code
},

generateAttributeRule: function (rule, attribute) {
    /***
     1- Only 3 primitive type are working
     Integer - Boolean - String
     2- GReQL Engine cannot handle this case from BOUML XMI
     */
    let code = ""
    code += "<!-- Attribute Rule -->"

    const visibility = this.getVisibility(attribute)
    const isStatic = this.isStatic(attribute)
    const primitiveType = this.getType(attribute.type)

    let vType = "from x: V{Class}, y : V{Property}"
    let vTypeText = ""
    if (primitiveType !== '!prim') {
        vType = "from x : V{Class},
        y : V{Property},
        z : V{PrimitiveType}"
        vTypeText = `and y --> z and
        isDefined(z.name) and
        z.name="${primitiveType}"`
    }

    code += `<rule type="presence"
    points="${attribute.points}">
    <query>${vType}
           with
              isDefined(x.name) and
              stringLevenshteinDistance(x.name,
              "${rule.rule_specific.class_name}")&lt;3 and
              x --> y and isDefined(y.name) and
              stringLevenshteinDistance(y.name,
              "${attribute.name}")&lt;3 and
              ${visibility}
              ${isStatic}
              ${vTypeText}
           report 1 end
    </query>
    <feedback>${attribute.feedback}</feedback>
</rule>`
    return code
},
generateMethodRule: function (rule, method) {
    let code = ""
    const visibility = this.getVisibility(method)
    const isStatic = this.isStatic(method)
    const retType = this.getType(method.return_type)
    /***
     1- Only 3 primitive type are working
     Integer - Boolean - String
     */
    let vType = "from x: V{Class}, y : V{Operation}"
    let vTypeText = ""
    if (retType !== '!prim') {
        vType = "from x : V{Class},
        y : V{Operation},
        ret: V{Parameter},
        retType: V{PrimitiveType}"
        vTypeText = ` and y --> ret and
        isDefined(ret.name) and
        ret.name="return"  and
        ret --> retType and
        isDefined(retType.name) and
        retType.name="${retType}"`
    }

    code += "<!-- Method Rule -->"
    code += `<rule type="presence"
    points="${method.points}">
    <query>${vType}
           with
              isDefined(x.name) and
              x.name="${rule.rule_specific.class_name}" and
              isDefined(y.name) and
              stringLevenshteinDistance(y.name,
              "${method.name}")&lt;3 and
              ${visibility}
              ${isStatic} and
              x --> y
              ${vTypeText}
           report 1 end
    </query>
    <feedback>${method.feedback}</feedback>
  </rule>`

    this.extractVariableNames(method.arguments)
        .forEach( arg => {
        code += "<!-- Method Param -->"
        code += `<rule type="presence" points="0">
        <query>from x: V{Class}, y : V{Operation},
        param: V{Parameter}
        with
        isDefined(x.name) and
        stringLevenshteinDistance(x.name,
        "${rule.rule_specific.class_name}")&lt;3 and
        isDefined(y.name) and
        stringLevenshteinDistance(y.name,
        "${method.name}")&lt;3 and
        x --> y and
        y --> param and isDefined(param.name) and
        param.name="${arg}"
       report 1 end
        </query>
        <feedback>Die Methode ${method.name} muss
        ein Attribut ${arg} haben.</feedback>
      </rule>`
    })
    return code
},

extractVariableNames: function (inputString) {
    inputString += ';'
    const variablePattern = /\b(\w+)\s*,?\s*(?=[,;])/g;
    const matches = inputString.match(variablePattern);
    if (matches)
        return  matches.map(match => match.trim());
    return [];
}
getVisibility: function (accessor) {
    switch (accessor.visibility) {
        case 'public':
            return 'y.visibility="public" and '
        case 'private':
            return 'y.visibility="private" and'
        case 'protected':
            return 'y.visibility="protected" and'
        default:
            return ""
    }
},
isStatic: function (accessor) {
    if (accessor.is_static)
        return 'y.isStatic=true '
    else
        return 'y.isStatic=false '
},
getType: function (type) {
    switch (type.toLowerCase()) {
        case 'int':
            return "Integer"
        case 'string':
            return "String"
        case 'bool':
        case 'boolean':
            return "Boolean"
        default:
            return "!prim"
    }
},
\end{lstlisting}
	
	\newpage
	\include{erklaerung}
 
\end{document}