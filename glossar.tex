\newglossaryentry{E-Assessment-Systemen}{
  name={E-Assessment-Systemen},
  description={auch als elektronische Bewertungssysteme bezeichnet, sind computergestützte Technologien und Plattformen, die entwickelt wurden, um den Prozess der Bewertung und Beurteilung von akademischen Leistungen, Prüfungen und Aufgaben zu unterstützen oder zu automatisieren}
}

\newglossaryentry{konzeptuellen Modellen}{
  name={konzeptuellen Modellen},
  description={sind abstrakte Darstellungen oder konzeptionelle Strukturen, die darauf abzielen, Ideen, Beziehungen, Konzepte oder Entitäten eines bestimmten Bereichs zu beschreiben, ohne in konkrete Details oder Implementierungsdetails einzugehen.}
}

\newglossaryentry{JACK}{
  name={JACK},
  description={ist ein e-Assessment System, das von der Universität Duisburg-Essen entwickelt wurde. Es handelt sich um eine webbasierte Plattform, die es Lehrkräften ermöglicht, Online-Bewertungen wie Quiz, Prüfungen und Umfragen zu erstellen und durchzuführen. Jack bietet eine Vielzahl von Funktionen, die den Lehrkräften bei der Verwaltung ihrer Prüfungen helfen, wie z. B. Benotung, Berichterstattung und Verfolgung des Lernfortschritts.
  \cite{jack}}
}

\newglossaryentry{summative Bewertung}{
  name={summative Bewertung},
  description={ist eine Art von Bewertung, die am Ende eines definierten Lernzeitraums durchgeführt wird, wie z.B. am Ende eines Kurses, eines Moduls oder eines Schuljahres.}
}

\newglossaryentry{formative Bewertung}{
  name={formative Bewertung},
  description={ist ein kontinuierlicher Prozess, der während des Lernens stattfindet. Ihr Hauptziel ist es, den Fortschritt der Schüler zu verfolgen, ihre Bedürfnisse zu identifizieren und ihnen bei der Verbesserung ihrer Leistung zu helfen.}
}

\newglossaryentry{Geschlossene Fragen}{
  name={Geschlossene Fragen},
  description={Diese Fragen zeichnen sich durch ihre Fähigkeit aus, eine
begrenzte Auswahl vorab definierter Antwort möglichkeiten anzubieten, aus
denen die Lernenden die adäquate Antwort auswählen müssen}
}

\newglossaryentry{Offene Fragen}{
  name={Offene Fragen},
  description={Im Kontrast zu geschlossenen Fragen präsentieren sie keine
vordefinierten Antworten und gestatten den Lernenden, ihre Gedanken
selbstständig zu artikulieren}
}