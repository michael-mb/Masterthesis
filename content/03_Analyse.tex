\chapter{Problemanalyse}\label{ch:problemanalyse}

In diesem Kapitel erfolgt eine gründliche Analyse der zugrunde liegenden Problematik. Die Identifikation und eingehende Diskussion der spezifischen Herausforderungen und Schwierigkeiten, die im Rahmen dieser Arbeit behandelt werden, stehen dabei im Fokus.

\section{Beschreibung des Bewertungsprozesses}\label{sec:bewertungsprozess}
% Description de l'environnement

Wie bereits in den vorherigen Kapiteln erwähnt, verwendet die Fakultät für Informatik der Universität Duisburg-Essen (UDE) ein elektronisches Bewertungssystem namens ``JACK'' \cite{jack}, um bestimmte Prüfungen und Übungen automatisch zu bewerten. Unter den verschiedenen Arten von Übungen konzentriert sich diese Masterarbeit auf Übungen vom Typ UML.

Eine Übung vom Typ UML-Klassendiagramm besteht darin, die statische Struktur eines Software-Systems mithilfe der UML-Modellierungssprache zu modellieren \cite{reggio2013used}. Den Studierenden wird in der Regel eine Beschreibung des Systems, der wichtigsten Entitäten und ihrer Beziehungen zur Verfügung gestellt, und sie müssen dann ein Klassendiagramm erstellen, das diese Elemente darstellt. In diesem Diagramm sind die Klassen die Hauptobjekte, mit ihren Attributen (Variablen) und Methoden (Funktionen), und die Beziehungen zwischen den Klassen werden durch Assoziationen, Aggregationen oder Kompositionen dargestellt. Die Studierenden müssen besonders auf die Genauigkeit der Namen, Multiplizitäten und Kardinalitäten achten, um die Struktur des Systems korrekt widerzuspiegeln \cite{reggio2013used}. Das Hauptziel dieser Übung ist es, das Verständnis der objektorientierten Modellierungskonzepte zu vertiefen und eine solide Grundlage für die Softwareentwicklung zu schaffen. In Bezug auf das JACK-System kann der Bewertungsprozess in mehrere verschiedene Phasen unterteilt werden:

\vspace{0.5cm}

\textbf{Phase 1: Erstellung der Übung}

In dieser ersten Phase erstellt der Lehrer die Übung, indem er eine schriftliche Beschreibung eines zu modellierenden Systems bereitstellt. Besonderes Augenmerk wird auf die Genauigkeit der Bezeichnungen der Entitäten und die Klarheit und Explizitheit der Beziehungen zwischen ihnen gelegt. Der Lehrer hat dann einen Bereich in der JACK-Anwendung, in den er diesen Text einfügen kann, der von den Studierenden eingesehen wird.


\textbf{Phase 2: Erstellung einer Musterlösung oder Anmerkung (optional)}

In dieser Phase kann der Lehrer entscheiden, ein UML-Diagramm als Musterlösung für die Übung zu erstellen. Dies erleichtert das Verständnis der Übung, ermöglicht die Überprüfung der Kohärenz und Durchführbarkeit des Systems. Alternativ kann der Lehrer die Übung lediglich annotieren, um die Schlüsselbegriffe und -elemente hervorzuheben, die in der späteren Phase nützlich sein werden.


\textbf{Phase 3: Entwicklung des GReQL-Codes}

In dieser Phase verwendet der Lehrer seine Anmerkungen und die Musterlösung, um den GReQL-Code zu entwickeln, der von JACK interpretiert wird, um die von den Studierenden eingereichten Lösungen zu bewerten. JACK verwendet den GReQL-Code sowie verschiedene in dieser Sprache definierte Regeln, um die UML-Diagramme zu bewerten. Wenn ein Student seine Lösung einreicht, wird eine grafische Darstellung dieser Lösung erstellt, und der GReQL-Code führt Abfragen auf dieser Darstellung aus, um eine Note für die Lösung des Studenten zu vergeben \cite{striewe2011automated}.

\textbf{Phase 4: Einreichung der Lösung durch den Studenten}

Nachdem der Student an der Lösung der Übung gearbeitet hat, lädt er ein XML-Dokument im XMI Format auf JACK hoch. Zur Generierung dieses kompatiblen XMI-Dokuments verwenden Studierende Tools wie BOUML \cite{bouml} oder Software Ideas Modeler \cite{sim}, mit denen sie XMI-Code aus einer zuvor erstellten grafischen UML Darstellung ableiten können. Dieses Dokument repräsentiert die von ihm entwickelte Lösung.

    
\textbf{Phase 5: Bewertung des Diagramms}

Das von den Studierenden eingereichte Diagramm wird anhand des von den Lehrern verfassten GReQL-Codes bewertet. Anschließend wird dem Studenten eine Note zugeteilt.


Diese Phasen veranschaulichen die grundlegende Funktionsweise der JACK-Plattform in Bezug auf die automatisierte Bewertung von Übungen zur Erstellung von UML-Klassendiagrammen.


\section{Untersuchung des Problems}

Im Verlauf der dritten Phase, wie im vorherigen Kapitel beschrieben, sehen sich Lehrende mit der Notwendigkeit konfrontiert, GReQL-Code zu verfassen, der vom JACK-System zur Bewertung der Einreichungen verschiedener Studierender verwendet wird. Diese Phase stellt jedoch bereits auf verschiedenen Ebenen eine Herausforderung für die Lehrenden dar, aus folgenden Gründen:

\textbf{Erforderliche Expertise für die Erstellung von GReQL-Code:} Das Verfassen von GReQL-Code erfordert eine gewisse Expertise. Auf den ersten Blick mag die Syntax von GReQL nicht schwer zu verstehen sein. Dennoch kann es anspruchsvoll sein, die Feinheiten der Code-Erstellung zu beherrschen. Dies erfordert von den Lehrenden mehrere Stunden, intensives Üben und umfangreiche Tests, um einen Code zu erstellen, der präzise bewertet, insbesondere in Fällen von komplexeren Beziehungen zwischen Entitäten.

\textbf{Hohe Möglichkeit von Fehlern im GReQL-Code:} Selbst bei Beherrschung der Feinheiten von GReQL sind Lehrende nicht vor möglichen Fehlern gefeit. Diese Fehler können zwar geringfügig erscheinen, haben jedoch das Potenzial, das Bewertungssystem erheblich zu beeinträchtigen.

\textbf{Zeitaufwand für die Erstellung von GReQL-Code:} Die Bewertung jedes Diagramms erfordert erheblichen Zeitaufwand, um einen GReQL-Bewertungscode zu erstellen, insbesondere für komplexere Beziehungen zwischen verschiedenen Entitäten. Die Zeit, um die erforderliche Expertise zu erlangen, kann stark variieren und erfordert zahlreiche Stunden intensiver Übung.

\textbf{Wartung und Anpassungsfähigkeit des GReQL-Codes:} Nachdem der GReQL-Code erstellt und für die Bewertung von UML-Diagrammen implementiert wurde, ist kontinuierliche Wartung erforderlich. Mit steigenden Anforderungen an die Bewertung muss der GReQL-Code regelmäßig aktualisiert und angepasst werden. Dies stellt für Lehrende eine fortwährende Herausforderung dar und erfordert Zeit und Aufwand, um sicherzustellen, dass das Bewertungssystem präzise und relevant bleibt.

\textbf{Bedarf an Ressourcen und technischer Unterstützung:} Lehrende benötigen möglicherweise Zugang zu technischen Ressourcen und angemessener Unterstützung, um die Kunst der effizienten GReQL-Code-Erstellung zu beherrschen. Dies kann Schulungen, Orientierungsdokumente, Diskussionsforen oder andere Formen technischer Unterstützung einschließen. Die Beschaffung dieser Ressourcen kann Zeit in Anspruch nehmen und eine institutionelle Koordination erfordern.

Dies sind die verschiedenen Herausforderungen, die bei der Erstellung von GReQL-Code für die Bewertung von UML-Diagrammen auf der JACK-Plattform auftreten können.


\section{Ableitung und Abgrenzung der Anforderungen}
% Description des choses à faire ainsi qu'une limitation

Das Ziel dieser Masterarbeit besteht nicht darin, Lehrkräfte von der Notwendigkeit zu befreien, GReQL-Code zu schreiben oder zu bearbeiten. Vielmehr geht es darum, den Prozess der Regeldefinition erheblich zu erleichtern. Um diesen Prozess zu vereinfachen, zielt diese Masterarbeit darauf ab, ein Verfahren zu entwickeln und in Form einer Softwareanwendung zu prototypisieren, mit dem (halb-)automatisch Bewertungsregeln aus annotierten Musterlösungen erstellt werden können. Dieses System soll in der Lage sein, folgende Ziele zu erreichen:

\begin{enumerate}
    \item \textbf{Verminderung der Einstiegshürde:} Es strebt danach, den Prozess der Erstellung von GReQL-Regeln zu erleichtern. Selbst jemand, der keine Vorkenntnisse in GReQL hat, sollte unser Tool verwenden können, um Regeln zu erstellen, die bereits eine Bewertung von einfachen Diagrammen und Beziehungen ermöglichen.
    
    \item \textbf{Verbesserung der Präzision und Zuverlässigkeit der GReQL-Regeln:}
    Durch die Nutzung von annotierten Musterlösungen zielt die Anwendung darauf ab, die Präzision und Zuverlässigkeit der generierten Regeln zu verbessern. Dies würde das Risiko von Fehlern und Inkonsistenzen in den Bewertungsregeln verringern und somit eine konsistentere Bewertung der studentischen Arbeiten gewährleisten.
    
    \item \textbf{Optimierung von Zeit und Aufwand der Lehrkräfte:}
    Durch Automatisierung eines Teils des Regelbildungsprozesses zielt das System darauf ab, Zeit und Aufwand der Lehrkräfte zu sparen. Dies würde es ihnen ermöglichen, sich mehr auf das Lehren und Betreuen der Studierenden zu konzentrieren, anstatt auf mühsame administrative Aufgaben.
    
    \item \textbf{Förderung der Skalierbarkeit und Wartbarkeit der Regeln:}
    Durch die Implementierung eines Mechanismus zur Pflege und Aktualisierung der generierten Regeln würde die Anwendung dazu beitragen sicherzustellen, dass die Regeln relevant und anpassungsfähig bleiben und sich den sich ändernden Anforderungen von Lehre und Bewertung anpassen.
    
    \item \textbf{Unterstützung einer breiten Palette von Bewertungsszenarien:}
    Die Anwendung sollte flexibel genug sein, um eine Vielzahl von Bewertungsszenarien zu unterstützen, einschließlich solcher mit komplexen Diagrammen und Beziehungen. Dadurch würde sie eine vielseitige Lösung für Lehrkräfte bieten.
    
\end{enumerate}

Das Ziel dieser Initiative ist es, den Prozess der Bewertung von UML-Diagrammen effizienter, zugänglicher und präziser zu gestalten, während die administrativen Arbeitslasten der Lehrkräfte reduziert werden. Dadurch soll die Qualität von Lehre und Bewertung im Bereich der Softwaremodellierung verbessert werden.
