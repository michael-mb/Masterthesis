\chapter{Zusammenfassung und Ausblick}

Dieses Kapitel markiert den Abschluss dieser Masterarbeit. In erster Linie wird eine kurze Zusammenfassung der
erreichten Ergebnisse in dieser Arbeit präsentiert. Anschließend wird die Diskussion über potenzielle Ansätze und
Alternativen eröffnet, die zur Lösung der in diesem Kontext aufgeworfenen Problematik erkundet werden könnten.

\section{Zusammenfassung}

In dieser Masterarbeit wurde die Generierung von Feedback-Regeln für UML-Modelle im Kontext von E-Assessment-Systemen
untersucht. Die Untersuchung umfasste die Analyse der Herausforderungen, die im Zusammenhang mit der manuellen
Erstellung von GReQL-Code für die Bewertung von UML-Modellen auftreten können, sowie die Vorstellung des GReQL
Converters als Lösungsansatz. Im dritten Kapitel erfolgte eine detaillierte Analyse der Problematiken, die im
Rahmen der manuellen Erstellung von GReQL-Code auftreten können. Diese Probleme umfassen die notwendige Expertise für
die Erstellung von GReQL-Code, die Komplexität bei der Formulierung von Regelwerken sowie den zeitlichen Aufwand,
der für die manuelle Erstellung von Feedback-Regeln erforderlich ist. Die erwähnten Schwierigkeiten können zu einer
zeitaufwändigen und fehleranfälligen  manuellen Erstellung von Feedback-Regeln führen.

Der GReQL Converter präsentiert sich als Lösung für diese Problematiken. Durch den Einsatz dieses Konverters können
Feedback-Regeln automatisch generiert werden, ohne dass umfassende Kenntnisse im Bereich GReQL-Code erforderlich sind.
Zusätzlich dazu ist der Konverter in der Lage, komplexe Regeln zu formulieren und den zeitlichen Aufwand für die
manuelle Erstellung von Feedback-Regeln zu reduzieren. Die Ergebnisse dieser Untersuchung legen nahe, dass der GReQL
Converter als ein nützliches Instrument für die automatisierte Bewertung von UML-Modellen betrachtet werden kann. Er
trägt dazu bei, den Prozess der Generierung von Feedback-Regeln zu vereinfachen.

In der Gesamtschau der vorliegenden Masterarbeit wird aufgezeigt, dass die Nutzung des GReQL Converters eine wirksame
Lösung für die Herausforderungen bei der manuellen Erstellung von Feedback-Regeln für UML-Modelle darstellt. Die hier
erzielten Erkenntnisse haben das Potenzial, den Einsatz von E-Assessment-Systemen zu optimieren und den Lernprozess für
Lehrende und Lernende zu verbessern.

\section{Ausblick}

Für die Entwicklung des GReQL Converters wurde ursprünglich PlantText aufgrund seiner Fähigkeit gewählt, mithilfe des
PlantUML Parser Beziehungen und Entitäten aus einem UML-Diagramm zu extrahieren und sie in einem leicht verwertbaren
JSON-Format zu exportieren. Es gibt jedoch mehrere andere, komplexere Strategien, die in Erwägung gezogen werden können.
Aktuell erfordert die Verwendung des GReQL Converters die Anwendung von PlantText, während für die Erstellung des zu
bewertenden Diagramms auf der JACK-Seite die Generierung einer XMI-Datei mit BOUML erforderlich ist. Dies führt zu einer
gewissen Heterogenität im Arbeitsablauf. Die Erkundung der Möglichkeit, Regeln direkt aus einer XMI-Datei zu generieren,
könnte ein interessanter Entwicklungsbereich für den GReQL Converter sein. Dies wäre jedoch aufgrund der Entwicklung
eines angepassten Parsers, die bereits eine recht komplexe Aufgabe darstellt, anspruchsvoll. Dennoch würde dies
zweifellos einen Mehrwert für den Entwicklungsprozess bieten und es ermöglichen, aus einer Lösung, die beispielsweise
in BOUML erstellt wurde, nicht nur GReQL-Regeln zu generieren, sondern sie auch direkt zu überprüfen. Der Benutzer
müsste keine zwei unterschiedlichen Technologien erlernen, um diese Aufgabe zu erfüllen.

Ein weiterer Entwicklungsbereich, der von Anfang an aufgegeben wurde, aber dennoch von Interesse sein könnte, ist die
Integration eines KI-Algorithmus. Ein solcher Algorithmus könnte dazu befähigt sein, ein zuvor annotiertes
UML-Übungsszenario zu lesen und daraus abzuleiten, welche Regeln generiert werden sollten. Eine ähnliche Aufgabe wurde
bereits in einem anderen Kontext von Mohammed Amraouy et al.\cite{amraouy2023sentiment} realisiert, bei dem mithilfe
von Studentenkommentaren auf einer E-Assessment-Plattform eine Analyse und Klassifizierung der Emotionen durchgeführt
wurde, um das Engagement der Studenten in einem bestimmten Kurs zu bewerten. In einem Kontext, der dem dieser
Masterarbeit ähnlicher ist, ist auch die Arbeit von Aggarwal et al.\cite{aggarwal2018machine} relevant, die verschiedene
Algorithmen des maschinellen Lernens verwenden, um verschiedene Entitäten auf der Grundlage von Text in Kategorien zu
klassifizieren.

Da das Tool nicht vollständig automatisiert werden kann, ist nach wie vor das Eingreifen eines Benutzers erforderlich,
um die Richtigkeit und Konsistenz der generierten Regeln zu überprüfen.