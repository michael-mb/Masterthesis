\chapter{Implementierung}

Das vorliegende Kapitel widmet sich der umfassenden Dokumentation des Implementierungsprozesses des zuvor beschriebenen
Konzepts. Es bietet eine detaillierte Aufarbeitung der technischen Umsetzung und des Entwicklungsprozesses, der im
Rahmen dieser Masterarbeit durchgeführt wurde. Beginnend mit einer umfassenden Beschreibung der verwendeten Technologien
und Werkzeuge sowie einer ausführlichen Begründung für die Wahl dieser spezifischen Technologien, wird dieses Kapitel
einen tiefen Einblick in die präzise Umsetzung des Konzepts gewähren. Die Implementierung ist ein entscheidender Schritt
zur Realisierung des in den vorherigen Kapiteln skizzierten Ansatzes zur Bewertung von UML-Diagrammen. Durch die
Dokumentation dieses Schrittes wird das Verständnis für die technischen Aspekte des Projekts vertieft und ermöglicht
eine transparente Darstellung des Entwicklungsprozesses.

\section{Überblick über angewendete Werkzeuge und Technologien}
Dieses Kapitel konzentriert sich auf die Vorstellung der Werkzeuge, die für die Umsetzung des im vorherigen Kapitel
vorgestellten Konzepts verwendet werden. Es wird jedes Werkzeug vorgestellt und die Auswahl begründet.

\subsection{PlantText}

Der erste Schritt des im vorherigen Kapitel vorgestellten Konzepts besteht in der Modellierung eines UML-Diagramms.
Diese Modellierungsphase verlangt die Anwendung einer Software-Anwendung, welche in der Lage ist, das entworfene
Diagramm in ein Format zu überführen, welches für die anschließende Extraktion der enthaltenen Regelsätze dienlich
ist. Eine facettenreiche Auswahl an digitalen Werkzeugen steht zur Verfügung, um diese spezifische Aufgabe zu
bewältigen, wobei PlanText exemplarisch zu nennen ist.

PlantText~\cite{planttext} ist ein webbasiertes Instrument zur Diagrammmodellierung, das insbesondere in den Domänen
der UML und anderer Modellierungssprachen einen renommierten Status innehat. Dieses Instrument wurde entwickelt,
um die bequeme Erstellung, Bearbeitung und gemeinsame Nutzung von Diagrammen in einer kollaborativen Umgebung zu
ermöglichen. Seine Auszeichnungen resultieren aus der Benutzerfreundlichkeit, der Flexibilität und der
Leistungsfähigkeit, wodurch es zu einer favorisierten Wahl für Softwareentwickler, Systemarchitekten und Projektmanager
avanciert~\cite{planttext}.

PlantText basiert auf einer schlichten, aber wirkungsvollen Konzeption, nämlich der Generierung von Diagrammen durch
Verwendung von textuellen Notationen. Benutzer können Diagramme unter Einsatz von natürlicher Sprache und vordefinierten
Schlüsselwörtern kreieren, wodurch der gesamte Prozess simplifiziert wird. Eine prototypische Darstellung einer Klasse
in einem UML-Klassendiagramm kann etwa wie folgt aussehen:


\begin{lstlisting}[caption={[Codebeispiel] PlantText code Example}, label={lst:Planttext}, float=!ht, language=javascript]
class A {
   + attribute1: Typ1
   - attribute2: Typ2
   # operation1(): void
}
\end{lstlisting}

In diesem Illustrationsfall repräsentieren einfache Textnotationen die Klasse ``A'', ihre Attribute und Methoden.
Die Verwendung von ``+'' für öffentliche Attribute, ``-'' für private Attribute und ``#'' für Methoden gestaltet sich
intuitionsgetreu und erleichtert die Entwicklung von UML-Diagrammen erheblich. PlantText beinhaltet eine leistungsstarke
Rendering-Engine, die diese textlichen Notationen automatisiert in visuell ansprechende Diagramme konvertiert. Benutzer
sind in der Lage, die Diagramme in Echtzeit zu visualisieren und zu editieren, ohne sich mit den komplexen Details der
grafischen Gestaltung auseinandersetzen zu müssen. Diese textorientierte Herangehensweise führt zu einer höchst
effizienten und adaptierbaren Gestaltung und Veränderung von Diagrammen. Die Vorzüge der Anwendung von PlantText
manifestieren sich unter anderem in:

\begin{enumerate}
    \item \textbf{Benutzerfreundlichkeit:} PlantText ist für Einsteiger und erfahrene Modellierer gleichermaßen zugänglich.
Die Verwendung von textuellen Notationen vereinfacht den Einstieg und reduziert die Lernkurve, da sie natürlicher und
verständlicher sind als grafische Schnittstellen~\cite{mazanec2012general}.
    \item \textbf{Kollaboration und gemeinsame Nutzung:} PlantText bietet eine eingebettete Kollaborationsplattform, auf
der mehrere Benutzer simultan an Diagrammen arbeiten können. Dies fördert die Teamarbeit und erlaubt die
Echtzeit-Erstellung und Überarbeitung von Modellen~\cite{madanayake2017transforming}.
    \item \textbf{Plattformunabhängigkeit:} Da PlantText webbasiert ist, ist es plattformneutral. Benutzer können von
jedem Gerät mit Internetzugang auf ihre Modelle zugreifen und sie editieren, ohne Softwareinstallationen durchführen
zu müssen~\cite{planttext}.
    \item \textbf{Erweiterbarkeit:} PlantText unterstützt nicht ausschließlich UML, sondern auch diverse andere
Modellierungssprachen und Diagrammtypen. Infolgedessen entwickelt sich PlantText zu einem vielseitigen Werkzeug für eine
Vielzahl von Anwendungsszenarien~\cite{planttext}.
\end{enumerate}

\begin{figure}
    \centering
    \includegraphics[width=15cm]{images/plantText}
    \caption{Grafische Benutzeroberfläche von plantText}
    \label{fig:plant-text}
\end{figure}

Vor der Wahl von PlantText als Instrument für die Entwurfsphase der Anwendung wurden mehrere Modellierungswerkzeuge
einer eingehenden Prüfung unterzogen. Diese Werkzeuge schlossen namhafte Anwendungen wie Enterprise Architect~\cite{enterarch},
Astah UML ~\cite{astah}, MagicDraw ~\cite{magic}, Visual Paradigm ~\cite{visual}, Umbrello ~\cite{umbrello} und Draw.io~\cite{draw} ein.
Eine gemeinsame Eigenschaft dieser Werkzeuge besteht darin, dass sie sich für die rasche Erstellung von Diagrammen
eignen, was auf ihre intuitive Benutzeroberfläche, leichte Verständlichkeit und Nutzerfreundlichkeit zurückzuführen ist.
Bedauerlicherweise wiesen sie jedoch einen bedeutenden Mangel auf, der ihre Anwendbarkeit in Bezug auf unsere
spezifischen Anforderungen einschränkte. Keines dieser Programme bot die Möglichkeit, die erstellten Diagramme in eine
leicht interpretierbare textuelle Form zu überführen.

Die meisten dieser Tools gestatten zwar das Exportieren der erstellten Diagramme im XML-Format, doch dieses Format
präsentiert lediglich eine räumliche Repräsentation der verschiedenen Objekte in einer Ebene, ohne eine semantische
Tiefenstruktur. Ein weiteres Hindernis bestand darin, dass die Informationen bezüglich der Verbindungen zwischen den
diversen UML-Objekten nur mit großem Aufwand und erheblichen Schwierigkeiten aus dem generierten XML extrahiert werden
konnten. Dies wäre in der Praxis äußerst zeitaufwendig und würde die Entwicklung eines eigenen Parsers erfordern, um die
relevanten Informationen zu extrahieren und in eine verwertbare Form zu überführen.

Die Nutzung von PlantText hingegen bietet einen klaren Vorteil in dieser Hinsicht. Dies resultiert aus der bereits
implementierten PlantUML-Engine, die über einen eingebauten Parser verfügt. Diese Funktionalität ermöglicht es,
den erstellten Code direkt in einem Format zu erhalten, das für die Weiterverarbeitung und Interpretation äußerst zugänglich ist.
Diese grundlegende Unterscheidung führt dazu, dass PlantText in unserem Anwendungsfall als überlegen angesehen wird.
Durch die Fähigkeit zur Bereitstellung des Modells in einer textuellen Form ermöglicht es eine tiefere und
bedeutungsvollere Analyse der erstellten Diagramme. Dies fördert die Genauigkeit bei der Modellierung und stellt sicher,
dass die erstellten Diagramme nicht nur als visuelle Darstellungen betrachtet werden, sondern auch als Quellen
semantischer Informationen dienen können.

\subsection{PlantUML Parser + Nodejs}

Im vorangegangenen Unterabschnitt wurde die Anwendung PlantText als Instrument zur Erstellung von Diagrammen
vorgestellt. Es ist jedoch essenziell zu betonen, dass PlantText lediglich die grafische Benutzeroberfläche darstellt,
da die zugrundeliegende Engine, auf der PlantText basiert, ``Plant-UML''~\cite{plantUML} ist. PlantUML ist ein Open-Source-Tool,
das von Arnaud Roques entwickelt wurde und erstmals im Jahr 2009 veröffentlicht wurde~\cite{plantUML}. Als Java-Anwendung
ermöglicht PlantUML, ähnlich wie PlantText, die Modellierung von Diagrammen, wobei die Verwendung durch die Entwicklung
von PlantText vereinfacht wurde. Alle im vorherigen Abschnitt hervorgehobenen Vorteile der Nutzung von PlantText sind in
Wirklichkeit Funktionen von PlantUML.

Eine besonders bemerkenswerte Funktion, die in diesem Abschnitt präsentiert und im Folgenden verwendet wird, ist jedoch
der PlantUML-Parser~\cite{plantUMLParser}. PlantUML ist im Wesentlichen eine Backend-Anwendung, die auf einem Server
ausgeführt wird. Wenn ein Benutzer ein Modell erstellen und Code in PlantText eingeben möchte, wird dieser Code an einen Server
übertragen, der ihn analysiert, interpretiert und ein Diagramm mithilfe von Graphviz~\cite{graphViz} generiert. Mit
Hilfe des PlantUML-Parsers besteht die Möglichkeit, das erzeugte Diagramm in ein Format zu exportieren, das für
programmatische Anwendungen geeignet ist. Es ist genau diese Funktion, die im weiteren Verlauf dieser Abhandlung
ausführlich behandelt wird.

Der PlantUML Parser~\cite{plantUMLParser} ist ein Open-Source-Tool,
mit dem der PlantUML-Code in ein JSON-Format geparst werden kann. Dieses Format kann zur Modellierung verschiedener
Regelobjekte (wie im Konzept beschrieben~\ref{sec:konzept}) verwendet werden. Da der Parser ausschließlich in einer
Serverumgebung funktioniert und somit in verschiedenen Java-, Node.js-Umgebungen verfügbar ist, kann er
ausschließlich in solchen Umgebungen eingesetzt werden.

Node.js~\cite{Node} wird verwendet, um diese Serverumgebung zu erstellen und eine gewisse Kohärenz zwischen dem
Frontend und dem Backend sicherzustellen, indem die Verwendung mehrerer Programmiersprachen in einem Projekt vermieden
wird. Dies trägt zur Effizienz und Integration des Gesamtprojekts bei.

\subsection{Vue.js}

Vue.js, oft auch einfach als Vue bezeichnet, ist  ein leistungsfähiges Open-Source-JavaScript-Framework, konzipiert und
entwickelt von Evan You~\cite{vue}. Der Ursprung von Vue.js entsprang der Vision, eine zeitgemäße, wandelbare und leicht
handhabbare Lösung zur Gestaltung von Benutzeroberflächen in Webanwendungen zu schaffen. In seiner Erstveröffentlichung
im Jahr 2014 initiiert, hat Vue.js eine bemerkenswerte Entfaltung erfahren, die es zu einem herausragenden Akteur unter
den JavaScript-Frameworks in der Sphäre der Webentwicklung gemacht hat.

In den Zielen und Vorzügen von Vue.js kristallisiert sich eine Antwort auf die Herausforderungen bei der Generierung
interaktiver Webanwendungen und die Optimierung des Entwicklungsprozesses in ein ergötzliches Narrativ. Wesentliche
Prämissen und Gewinnpunkte von Vue.js offenbaren sich in folgender Weise:

\begin{enumerate}
    \item \textbf{Nahtlose Integration:} Vue.js kann ohne Mühe in laufende Projekte integriert werden, unabhängig davon,
ob es als das Hauptframework oder als eine ergänzende Komponente in Kombination mit anderen Technologien fungiert.
Hierdurch ergibt sich ein gestaffelter Übergangsprozess und eine vorherrschende Flexibilität in der Architektur von
Anwendungen.
    \item \textbf{Komponentenbasierte Architektur:} Vue begünstigt die Einsetzung wiederverwendbarer Komponenten, die
nicht nur die Strukturierung und Organisation des Quellcodes erleichtern, sondern auch eine präzise Abgrenzung von
Aufgaben und eine verbesserte Wartbarkeit ermöglichen.
    \item \textbf{Reaktive Datenbindung:} Vue bietet eine reaktive Datenbindung, die die automatische Synchronisierung
von Daten und Benutzeroberfläche gestattet. Hierbei erfolgt die Anpassung von Daten an die Benutzeroberfläche und
umgekehrt ohne das Hinzufügen von Zusatzcode.
    \item \textbf{Deklarative Rendering:} Die Einbindung deklarativer Syntax in Vue.js vereinfacht die Implementierung
von Benutzeroberflächenelementen erheblich. Entwickler können die gewünschte Darstellung der Benutzeroberfläche
beschreiben, während Vue für die entsprechende Logikumsetzung sorgt.
    \item \textbf{Gemeinschaftsunterstützung:} Vue.js profitiert von einer florierenden und stetig wachsenden
Entwicklergemeinschaft, die eine Vielzahl von Ressourcen, Bibliotheken und Erweiterungen zur Verfügung stellt. Dies
vergrößert die Möglichkeiten zur Erweiterung und Anpassung von Vue-Projekten erheblich.
\end{enumerate}

Die Entscheidung zur Verwendung von Vue.js in einem Projekt kann vielschichtige Vorteile entfalten. Primär ermöglicht
die komponentenbasierte Architektur eine effiziente Code-Entwicklung, indem wiederverwendbare Komponenten zur
Strukturierung komplexer Benutzeroberflächen genutzt werden. Dies resultiert in einer verbesserten Wartbarkeit des
Quellcodes und einer beschleunigten Entwicklungszeit~\cite{wohlgethan2018supportingweb}.

Die reaktive Datenbindung von Vue.js bewirkt eine harmonische Synchronisation von Daten und Benutzeroberfläche, was die
Schöpfung interaktiver und ansprechender Webanwendungen begünstigt. Die deklarative Syntax von Vue reduziert
gleichzeitig den Boilerplate-Code und erleichtert die Nachvollziehbarkeit des Quellcodes. Vue.js ist zudem für seine
aktive Entwicklergemeinschaft und die Verfügbarkeit einer Vielzahl von Erweiterungen und
Plugins bekannt. Dies erlaubt den Zugriff auf bewährte Lösungen und bewährte Praktiken, was die Effizienz und Qualität
eines Projekts immens steigern kann~\cite{wohlgethan2018supportingweb}.

Schließlich bietet Vue.js eine attraktive Option für jene, die ein flexibles und gut dokumentiertes Framework suchen,
das sich reibungslos in bestehende Projekte einfügt. Vue kann schrittweise übernommen und je nach Projektanforderungen
sowohl als Hauptframework als auch für spezifische Aufgaben verwendet werden ~\cite{wohlgethan2018supportingweb}.

Zusammengefasst gewährleistet Vue.js eine stabile Grundlage für die Gestaltung zeitgemäßer, interaktiver Webanwendungen
und trägt entscheidend dazu bei, die Effizienz und Qualität von Projekten zu erhöhen. Vor diesem Hintergrund erweist
sich die Verwendung von Vue.js in diesem Projekt als empfohlen, um die Vorzüge dieses robusten Frameworks voll
auszuschöpfen. Für die Umsetzung des Konzepts wurden verschiedene andere Werkzeuge und Bibliotheken verwendet, jedoch
wurden in diesem Abschnitt nur die wichtigsten vorgestellt.

\section{Darlegung des Workflow-Prozesses}

In diesem Abschnitt wird der umfassende Prozess zur Umsetzung des im vorherigen Kapitels skizzierten Konzepts~\ref{sec:konzept}
detailliert erörtert. Ein zentrales Element dieser Umsetzung ist das derzeit in der Entwicklungsphase befindliche
Werkzeug, welches unter dem Namen \textbf{``GReQL Converter''} firmiert. Dieses Instrument dient dazu, das zuvor skizzierte
Konzept in die praktische Anwendung zu überführen. Der GReQL Converter ist eine Webanwendung, die in der Lage ist,
GReQL Code aus PlantText zu generieren, um damit die Evaluation von UML-Klassendiagrammen zu ermöglichen. Die
technischen Einzelheiten der Implementierung dieses Tools werden im ausführlichen Abschnitt 5.3 eingehend beleuchtet.
Der Workflow-Prozess gliedert sich in vier Phasen:

\begin{table}[h!]
    \centering
    \caption{Workflowphasen}\label{tab:phasen}
    \begin{tabular}{ll}
        \toprule
        \textbf{Darstellung der Workflowphasen} \\
        \midrule
        Phase 1: Modellierung einer Musterlösung mit PlantText \\
        Phase 2: Verarbeitung der auf dem GReQL Converter erzeugten Regeln \\
        Phase 3: Generierung des GReQL-Codes \\
        Phase 4: Übertragung des GReQL-Codes auf JACK \\
        \bottomrule
    \end{tabular}
\end{table}

\begin{figure}
    \centering
    \includegraphics[width=15cm]{images/workflow}
    \caption{Repräsentatives Bild des Workflows des GReQL-Converters.}
    \label{fig:workflow}
\end{figure}

\subsubsection{Phase 1: Modellierung einer Musterlösung mit PlantText}

Nachdem der Lehrer eine Übung in Textform erstellt hat, um die Studierenden zu bewerten, muss er die Musterlösung
mithilfe von PlantText modellieren. Dabei müssen bestimmte Kriterien beachtet werden, die in der Dokumentation zur 
Verwendung des GReQL Converters ausführlich beschrieben sind. Diese Kriterien umfassen beispielsweise:

\begin{itemize}
    \item Enums immer mit $<<enum>>$ zu kennzeichnen.
    \item Interfaces immer mit $<<interface>>$ zu annotieren.
    \item Eine spezielle Syntax, die für bestimmte Assoziationen einzuhalten ist.
\end{itemize}

Nachdem der Code für die Musterlösung generiert wurde, kann der Lehrer zur Phase 2 übergehen.


\subsubsection{Phase 2: Verarbeitung der auf dem GReQL Converter erzeugten Regeln}

In der Phase der Regelbearbeitung, die auf die Codegenerierung folgt, hat die Lehrkraft die Aufgabe, den generierten
Code in den GReQL Converter zu übertragen. In dieser Phase entstehen eine Vielzahl von anpassbaren Regeln. Die Anpassung
erfolgt gemäß den spezifischen Bewertungskriterien, die die Lehrkraft verfolgt. Es ist hierbei möglich, Änderungen an
den Regeln vorzunehmen, je nachdem, welche Aspekte der Übung bewertet werden sollen. Weiterhin besteht die Option,
Regeln hinzuzufügen oder zu entfernen. In der Bearbeitung der Regeln ergibt sich auch die Möglichkeit, individuelles
Feedback für jede einzelne Regel hinzuzufügen und die Punktzahl für jede Regel erneut zu justieren.


\subsubsection{Phase 3: Generierung des GReQL-Codes}

Sobald die Phase der Regelbearbeitung (Phase 2) abgeschlossen ist, kann die Lehrkraft den GReQL-Code auf der Plattform
generieren. Dabei orientiert sich die Generierung an den zuvor definierten Regeln. Der generierte GReQL-Code kann,
falls erforderlich, weiterhin angepasst werden.

Dieser Bearbeitungsschritt ist nur dann sinnvoll, wenn der Lehrer bereits über Kenntnisse in GReQL zur Generierung von
Regeln für Klassendiagramme verfügt. Dies ermöglicht es ihm, einige Details des Codes zu verfeinern. Der Zweck des
GReQL Converters ist es, den GReQL-Code zu generieren, der am besten auf die Erwartungen der Lehrkraft eingeht, so dass
dieser Änderungsschritt einfach übersprungen wird. Es ist jedoch möglich, dass dieser Schritt zu Beginn noch notwendig
ist.

\subsubsection{Phase 4: Übertragung des GReQL-Codes auf JACK}

Nach Beendigung der Phase 3 kann die Lehrkraft den generierten GReQL-Code auf die JACK-Plattform übertragen. Dies
erfolgt, nachdem die Übung auf der Plattform erstellt wurde. Anschließend haben die Benutzer die Möglichkeit, ihre
Übungen einzureichen, woraufhin eine Bewertung mit entsprechendem Feedback erfolgt.


Die hier skizzierten Phasen repräsentieren einen essenziellen Schritt in der Umsetzung des im vorherigen Kapitel Ansatzes
zur Bewertung von UML-Diagrammen. Die Dokumentation dieses Prozesses gewährt einen
umfassenden Einblick in die technischen Aspekte des Projekts und erlaubt eine transparente Darstellung des
Implementierungsprozesses.

\section{Einrichtung und Entwicklung des ``\gls{GReQL Converter}s''}\label{sec:greql-converter}

Der \gls{GReQL Converter} stellt eine Webanwendung dar, die es ermöglicht, GReQL-Regeln aus einem UML-Klassendiagramm zu
extrahieren, das mithilfe von PlantText erstellt wurde. Diese extrahierten Regeln dienen anschließend der Bewertung
von UML-Klassendiagrammen auf der JACK-Plattform. Die Realisierung dieser Anwendung erfolgte mit dem Vue.js-Framework
im Frontend und Node.js im Backend, wie in fig-~\ref{fig:infrastructure} veranschaulicht. In diesem Kapitel wird der
Implementierungsprozess der Plattform sowie ihrer verschiedenen Komponenten ausführlich erläutert.

\begin{figure}[h]
    \centering
    \includegraphics[width=15cm]{images/infrastucture}
    \caption{GReQL-Converter global infrastructure}
    \label{fig:infrastructure}
\end{figure}

\subsection{Einrichtung des PlantUML-Parsers}\label{subsec:einrichtung-des-plantuml-parsers}

Die Entwicklungsphase der Anwendung startete mit der Konfiguration des PlantUML-Parsers. Das ursprüngliche Ziel bestand
darin, eine Anwendung zu entwickeln, die in der Lage ist, PlantText-Code als Eingabe zu akzeptieren und als Ausgabe ein
JSON zu generieren, welches später zur Modellierung von Regelobjekten verwendet werden könnte. Ursprünglich war geplant,
den Parser direkt im Frontend einzusetzen. Es stellte sich jedoch heraus, dass der Parser nur in einer Serverumgebung
effektiv funktioniert. An dieser Stelle gab es zwei Optionen zur Auswahl: die Verwendung eines Tomcat-Servers mit Java
oder eines Node.js-Servers mit JavaScript. Die zweite Option erwies sich als die geeignetere Wahl, da sie es ermöglichte,
dieselbe Programmiersprache für das gesamte Projekt beizubehalten und die Gesamtarchitektur des Systems, einschließlich
seiner zukünftigen Bereitstellung, zu vereinfachen. Node.js eignet sich besonders gut für kleinere Projekte wie dieses.

Im Großen und Ganzen wurde Node.js ausschließlich für die Bereitstellung des Parsers genutzt, was den Backend-Code
erheblich vereinfachte~\ref{lst:bakcend}. Dies ermöglichte einen reibungslosen Übergang zum nächsten Schritt, nämlich
der Erstellung des Grunddesigns der Anwendung.

\subsection{Erstellung des Grunddesigns der Anwendung}\label{subsec:erstellung-des-grunddesigns-der-anwendung}

Um die Ziele des \gls{GReQL Converter}s zu erreichen, ist es von entscheidender Bedeutung, den Benutzern eine elegante,
intuitive und benutzerfreundliche grafische Benutzeroberfläche zur Verfügung zu stellen. Die möglichen Aktionen müssen
auf den ersten Blick leicht erkennbar sein, um das Verständnis zu fördern und dem Benutzer eine schnelle und effiziente
Arbeit zu ermöglichen~\cite{guntupalli2008user}. Die Seite ``Class Converter'' stellt die Benutzeroberfläche dar, auf
der der Benutzer seinen PlantText-Code in GReQL-Code umwandeln kann. Diese Seite ist in drei Hauptabschnitte unterteilt:

\begin{figure}
    \centering
    \includegraphics[width=16cm]{images/board}
    \caption{Dashboard im Überblick}
    \label{fig:dashboard}
\end{figure}

\subsubsection{Teil 1: Code editor}

Dieser Abschnitt der grafischen Benutzeroberfläche ist für die Eingabe des PlantUML-Codes durch den Benutzer vorgesehen.
Nachdem der Benutzer die Musterlösung in PlantText modelliert hat, muss er den generierten Code kopieren und in den
Editor einfügen, der durch die Nummer 1 (siehe~\ref{fig:dashboard}) gekennzeichnet ist. Darüber hinaus hat der Benutzer die
Möglichkeit, Beispielscodes auszuwählen, wie dies in PlantText der Fall ist und in der Abbildung unter Nummer 2 (siehe~\ref{fig:dashboard})
dargestellt ist. Da der geschriebene Code jedoch der in der Dokumentation definierten Struktur entsprechen muss, sind
diese Beispielscodes für den Benutzer nützlich, da sie funktionierende Codebeispiele liefern, von denen er sich
inspirieren lassen und seinen eigenen Code schreiben kann, um die Plattform zu testen. Schließlich ermöglicht der
Button ``Parse Code'' (Nummer 3 - siehe~\ref{fig:dashboard}) das Extrahieren der Regelobjekte aus dem Code, was uns zum
Teil 2 führt.

\subsubsection{Teil 2: Derived Rules}

Nachdem der Benutzer auf den Button ``Parse Code'' geklickt hat, wird eine Reihe von Regelobjekten
(Nummer 4 - siehe~\ref{fig:dashboard}) aus dem im Code-Editor vorhandenen Code generiert. Diese Regeln werden durch ein
Symbol und einen Titel dargestellt, was es ermöglicht, sie schnell zu unterscheiden und zu identifizieren. Ganz rechts
neben dem Regelobjekt befindet sich ein Ausrufezeichen, das tatsächlich ein Button ist (Nummer 6 - siehe~\ref{fig:dashboard}).
Wenn auf diesen Button geklickt wird, öffnet sich eine Dokumentation, die erklärt, was die Regel ist und wie sie
verwendet wird. Wenn der Benutzer auf die Regel klickt (Nummer 5 - siehe ~\ref{fig:dashboard}), öffnet sich ein Menü,
das es ihm ermöglicht, die in der Regel vorhandenen Informationen nach Belieben zu ändern (siehe~\ref{fig:rule_exemple}),
wodurch der \gls{GReQL Converter} zu einem halbautomatischen Werkzeug wird. Diese Informationen variieren je nach Art der Regel
und den für die Generierung des GReQL-Codes erforderlichen Informationen.

\begin{figure}[h]
    \centering
    \includegraphics[width=16cm]{images/derived_rule}
    \caption{Beispiel eines Regelobjekts}
    \label{fig:rule_exemple}
\end{figure}

Dann gibt es der Button ``Add Rule'' (Nummer 7 - siehe~\ref{fig:dashboard}), die es dem Benutzer ermöglicht,
zusätzliche Regeln hinzuzufügen. Nachdem die Konfiguration und die Änderungen abgeschlossen sind, kann der Benutzer den
GReQL-Code generieren, indem er auf den Button ``Generate GReQL Code'' klickt (Nummer 8 - siehe~\ref{fig:dashboard}).
Der GReQL-Code wird aus den zuvor vom Benutzer konfigurierten Regelobjekten generiert.

\subsubsection{Teil 3: GReQL Editor}

Genau wie der erste Abschnitt ist dieser Bereich auch ein Code-Editor, jedoch für XML, da der GReQL-Code im XMI-Format
vorliegt (das auf XML basiert). Dieser Editor (Nummer 9 - siehe~\ref{fig:dashboard}) ermöglicht es erfahrenen
Benutzern, die bereits Erfahrung mit GReQL haben, detaillierte und fortgeschrittene Änderungen am generierten Code
vorzunehmen. Auf diese Weise können sie die Abfragen verbessern und spezifizieren, wenn sie sie zu allgemein finden,
oder sie einfach anpassen, je nachdem, was sie im UML-Diagramm bewerten möchten.

Diese drei verschiedenen Teile bilden das grundlegende Design des \gls{GReQL Converter}s. Es gibt auch andere Seiten wie die
Dokumentation oder die Startseite, die nicht erwähnt wurden, da sie in der weiteren Entwicklung keine wesentliche Rolle
spielen. Dieses Design bietet jedoch die Möglichkeit zur Erweiterung, um die Hinzufügung weiterer Seiten und sogar
weiterer Parser zu erleichtern. Der nächste Abschnitt konzentriert sich auf den Prozess der Extraktion von Regelobjekten
aus dem zuvor geparsten Code.

\subsection{Regel-Extraktionsprozess}\label{subsec:regel-extraktionsprozess}


Um den Extraktionsprozess, der in diesem Kapitel beschrieben wird, einzuführen, wird mit einem Beispiel aus der
Modellierung in PlantText angefangen. Die Schritte bis zur Erstellung der Regeln werden verfolgt. Hierzu wird der
folgende PlantText-Code verwendet, der ein UML-Diagramm modelliert (siehe~\ref{fig:code-uml}).

\begin{figure}[h]
    \centering
    \includegraphics[width=16cm]{images/code-uml}
    \caption{PlantText-Code mit der grafischen Darstellung}
    \label{fig:code-uml}
\end{figure}

Das Diagramm enthält vier Klassen: ClassA, ClassB, ClassC und ClassD\@. ClassA hat drei Elemente: ein privates Attribut
namens ``Windows'' vom Typ Integer (Int), ein öffentliches Attribut namens ``Time'' vom Typ Date und eine öffentliche
Methode namens ``Lock'' ohne Parameter und mit dem Rückgabetyp ``void''. ClassB und ClassC sind leere Klassen ohne
definierte Attribute oder Methoden. ClassD ist ebenfalls eine leere Klasse. Im Diagramm sind verschiedene Beziehungen
zwischen diesen Klassen dargestellt:

\begin{enumerate}
    \item Es gibt eine Assoziationsbeziehung zwischen ClassA und ClassB mit einer Multiplizität von ``1..2'' auf der
Seite von ClassA und einer Multiplizität von ``*'' auf der Seite von ClassB, was bedeutet, dass eine Instanz von ClassA
mit 1 bis 2 Instanzen von ClassB verknüpft ist.
    \item Es besteht eine Kompositionsbeziehung zwischen ClassA und ClassC, die durch ein Rautensymbol auf der Seite von
ClassA gekennzeichnet ist, mit einer Multiplizität von ``*'' auf der Seite von ClassC. Diese Komposition zeigt an, dass
ClassA eine Sammlung von Instanzen von ClassC besitzt und dass diese Instanzen von ClassC von den Instanzen von ClassA
verwaltet werden.
    \item Eine Vererbungsbeziehung (oder Generalisierung) ist zwischen ClassB und ClassC etabliert, gekennzeichnet durch
einen Pfeil, der von ClassC auf ClassB zeigt, mit einer gestrichelten Linie. Dies bedeutet, dass ClassC eine Unterklasse
von ClassB ist, was auf eine Vererbungsbeziehung hinweist.
    \item Schließlich gibt es eine einfache Assoziationsbeziehung zwischen ClassC und ClassD, dargestellt durch eine
durchgehende Linie, was darauf hinweist, dass es eine Verbindung zwischen den beiden Klassen gibt, obwohl die Details
dieser Assoziation im Diagramm nicht spezifiziert sind.
\end{enumerate}


Dies ist eine Darstellung des PlantText-Codes. Sobald dieser Code im \gls{GReQL Converter} vorliegt und der Benutzer auf
``Parse Code'' klickt (Nummer 5 siehe~\ref{fig:dashboard}), wird dieser Code über das HTTP-Protokoll an den
Node.js-Server gesendet. Der Server ist dafür verantwortlich, den Code zu analysieren, parsen und ein JSON-Objekt
zurückzugeben, ähnlich dem unten dargestellten:

\begin{lstlisting}[caption={Snippet des JSON-Codes, der nach dem Parsen des PlantText-Codes durch den Server erhalten
wurde.}, label={lst:parsed-json}, language=text]
[
  {
    "elements": [
      {
        "name": "ClassA",
        "title": "ClassA",
        "isAbstract": false,
        "members": [
          {
            "name": "Windows",
            "isStatic": false,
            "accessor": "-",
            "type": "Int"
          },
          {
            "name": "Time",
            "isStatic": false,
            "accessor": "+",
            "type": "Date"
          },
          {
            "name": "Lock",
            "isStatic": false,
            "accessor": "+",
            "returnType": "void",
            "_arguments": ""
          }
        ],
        ...
      },
      ...
      {
        "left": "ClassA",
        "right": "ClassB",
        "leftType": "Unknown",
        "rightType": "Unknown",
        "leftArrowHead": "",
        "rightArrowHead": "",
        "leftArrowBody": "-",
        "rightArrowBody": "-",
        "leftCardinality": "1..2",
        "rightCardinality": "*",
        "label": "",
        "hidden": false
      },
      ...
      {
        "left": "ClassC",
        "right": "ClassD",
        "leftType": "Unknown",
        "rightType": "Unknown",
        "leftArrowHead": "",
        "rightArrowHead": "",
        "leftArrowBody": "-",
        "rightArrowBody": "-",
        "leftCardinality": "",
        "rightCardinality": "",
        "label": "",
        "hidden": false
      }
    ]
  }
]
\end{lstlisting}

Nach Erhalt des JSON wird der Regelgenerierungsprozess gestartet. Basierend auf dem JSON kann für jedes Element eine
zugehörige Regel identifiziert werden. Zum Beispiel, um eine Generalisierung (Vererbung) zwischen zwei Klassen zu
erkennen, reicht es aus, ein Objekt aus dem JSON (hier als ``elem'' bezeichnet) zu betrachten und zu überprüfen, ob
folgende Bedingung erfüllt ist:

\begin{lstlisting}[caption={Evaluationsbedingung für eine Generalisierung}, label={lst:condition-check}, language=javascript]
elem.leftArrowHead.includes("<|") &&
elem.leftArrowBody.includes("-") &&
elem.rightArrowBody.includes("-")
\end{lstlisting}

Dies gilt auch für alle anderen Regeln, die anhand des generierten JSON identifiziert werden können. Einige
UML-Feinheiten werden jedoch nicht vom Parser erkannt, was in einem späteren Abschnitt erläutert wird. Bei der
Entwicklung des GReQL Converters wurden eine Reihe von Regeln definiert, darunter:

\subsubsection{Class definition}
Diese Regel hat zum Ziel, die Anwesenheit oder Abwesenheit einer Klasse in einem UML-Diagramm zu bestimmen. Darüber
hinaus ermöglicht sie die Extraktion verschiedener Merkmale dieser Klasse. Mit Hilfe dieser Regel können Informationen
wie die Abstraktion der Klasse, ihre Interface-Natur, die Methoden (einschließlich der Parameter und des Rückgabetyps),
die Attribute und ihre Typen extrahiert werden (siehe~\ref{lst:rules_def}). Nach Extraktion dieser Informationen wird
ein entsprechendes JSON-Objekt erstellt und einer Liste von Objekten hinzugefügt.

\subsubsection{Enum definition}
Das Ziel dieser Regel besteht darin, Enumerationen in einem UML-Diagramm zu identifizieren. Nachdem eine Enumeration
ordnungsgemäß identifiziert wurde, geht es darum, alle verschiedenen Attribute zu identifizieren, die sie zusammensetzen,
was ebenfalls in dieser Regel durchgeführt wird (siehe~\ref{lst:rules_def}). Nach dieser Identifizierung wird ein
entsprechendes JSON-Objekt erstellt und der Liste von Objekten hinzugefügt.

\subsubsection{Generalization}
Diese Regel dient dazu, die verschiedenen Vererbungsbeziehungen zwischen Klassen und Schnittstellen zu definieren.
Sie wird erstellt, wenn eine Klasse von einer anderen erbt oder eine Schnittstelle implementiert. Die Details dieser
Beziehung werden erkannt, und ein JSON-Objekt wird generiert und zu einer Objektliste hinzugefügt (siehe~\ref{lst:rules_def}).

\subsubsection{Simple Association}
Diese Regel zielt darauf ab, die Assoziationsbeziehungen zwischen Klassen mit ihren jeweiligen Vielfachen zu definieren.
Wenn eine Assoziation erkannt wird, werden die Klassennamen und die Vielfachen in einem JSON-Objekt gespeichert, das
anschließend zu einer Objektliste hinzugefügt wird (siehe~\ref{lst:rules_def}).

\subsubsection{Composition und Aggregation}
Diese beiden Regeln dienen dazu, die verschiedenen Compositions- und Aggregationsbeziehungen im UML-Diagramm zu
identifizieren und zu definieren. Das generierte Objekt enthält auch Multiplizitäten, die bei der Generierung von
GReQL-Code wichtig sind. Dieses Objekt wird dann zu einer Liste von Objekten hinzugefügt (siehe~\ref{lst:rules_def}).

\subsubsection{Association Class}
Diese Regel ermöglicht es, die Beziehung von Assoziationsklassen zu identifizieren und zu definieren. Obwohl sie in den
meisten UML-Diagrammen während der Modellierung selten verwendet wird, spielt sie dennoch bei der Bewertung eine
wichtige Rolle. Das generierte JSON-Objekt enthält im Wesentlichen Informationen zur zugehörigen Klasse und nicht direkt
zur Beziehung (siehe~\ref{lst:rules_def}).

\subsubsection{Nomination Consistency (optional)}
Diese Regel dient dazu, festzustellen, ob das zu bewertende Diagramm eine Konsistenz bei der Benennung der verschiedenen
Attribute aufweist. Zum Beispiel gilt es als schlechte Praxis, Attribute gleichzeitig mit Groß- und Kleinbuchstaben zu
benennen~\cite{albert2003implementing}. Ein JSON-Objekt wird generiert und zu einer Liste von Objekten hinzugefügt.
Diese Regel ist optional, was bedeutet, dass sie zu den Regeln gehört, die vom Lehrer selbst hinzugefügt werden müssen.
Sie wird also nicht automatisch generiert (siehe~\ref{lst:rules_def}).

\subsubsection{Count Methods (optional)}
Diese Regel ermöglicht es, die Anzahl der im UML-Diagramm vorhandenen Methoden zu definieren. Der Lehrer muss als
Parameter die genaue Anzahl der Methoden angeben, die im UML-Diagramm des Schülers vorhanden sein müssen.
Ein JSON-Objekt wird generiert und zu einer Liste von Objekten hinzugefügt. Diese Regel ist optional, was bedeutet,
dass sie zu den Regeln gehört, die vom Lehrer selbst hinzugefügt werden müssen. Sie wird also nicht automatisch
generiert (siehe~\ref{lst:rules_def}).

\subsubsection{Count Attributes (optional)}
Diese Regel ermöglicht es, die Anzahl der im UML-Diagramm vorhandenen Attribute zu definieren. Der Lehrer muss als
Parameter die genaue Anzahl der Attribute angeben, die im UML-Diagramm des Schülers vorhanden sein müssen. Diese Regel
ähnelt der Regel Count Methods (siehe~\ref{lst:rules_def}).

\subsubsection{Test Association}
Diese Regel dient einfach dazu, festzustellen, ob es eine Beziehung zwischen zwei Klassen gibt. Ein JSON-Objekt wird
generiert und zu einer Liste von Objekten hinzugefügt (siehe~\ref{lst:rules_def}).
\\~\\
Die zuvor erwähnte Objektliste enthält nun alle Regeln, die aus dem vom Node.js-Server bereitgestellten JSON extrahiert
wurden. Jede dieser Regeln wird auf der Frontend-Seite angezeigt und ist anpassbar (siehe~\ref{fig:rule_exemple}).
Der Benutzer hat somit die Möglichkeit, jede Regel nach seinen Wünschen anzupassen, indem er beispielsweise Kommentare,
Punkte (Points) sowie jeden mit der jeweiligen Regel verknüpften Attribut ändert. Auf diese Weise werden die Regeln aus dem JSON
extrahiert und angepasst. Anschließend folgt der nächste Schritt, bei dem diese Regeln in GReQL-Code umgewandelt werden.

\begin{figure}[!h]
    \centering
    \includegraphics[width=16cm]{images/extracted_rules}
    \caption{Regeln, die dem geparsten PlantText-Code entsprechen}
    \label{fig:extracted_rules}
\end{figure}

\subsection{Umwandlung der Regel-Objekte in GReQL-Code}\label{subsec:umwandlung-der-regel-objekte-in-greql-code}
Nach dem Extrahieren und Anpassen der Regeln ist es nun möglich, GReQL-Code aus der Liste zu generieren, die die Objekte
darstellt, welche diese Regeln repräsentieren. Tatsächlich hat jede Regel eine entsprechende Vorlage. Obwohl es
verschiedene Möglichkeiten gibt, GReQL-Code zu schreiben, wurde im Verlauf der Entwicklung nach einer optimierten
Variante gesucht, die das schnellste und effizienteste Ergebnis mit minimalem Code ermöglicht. Konkret geht es darum,
den GReQL-Code so anzupassen, dass er perfekt zur zu konvertierenden Regel passt, und ihn dann als Vorlage zu verwenden.
Zum Beispiel für die Regel, die das Vorhandensein einer Klasse definiert (siehe~\ref{lst:class_def_xml}):
\\~\\
\begin{lstlisting}[!h, caption={Class Definition Template \(erster Teil\)}, label={lst:class_def_xml}, language=xml]
<rule type="${rule.existence}" points="${rule.points}">
    <query>
        from x : V{Class} with isDefined(x.name) and
           stringLevenshteinDistance(x.name,
           "${rule..}")&lt;3
           ${abstractCode}
           report 1 end
    </query>
    <feedback>
        ${rule.feedback}
    </feedback>
</rule>
\end{lstlisting}

Nachdem die Vorlage erhalten wurde, werden die Parameter durch diejenigen aus dem Regelobjekt ersetzt. Auf diese Weise
wird für jedes Objekt eine entsprechende Regel erstellt. Dieser Prozess kann je nach den verschiedenen Regeln und
Objekten variieren, bleibt jedoch im Wesentlichen für die meisten Regeln gleich. Am Ende dieses Prozesses wird einen
gültigen GReQL-Code erhalten, der auf der JACK-Plattform verwendet werden kann.
\\~\\
\begin{lstlisting}[!h, caption={Ausschnitt aus dem GReQL-Code, der nach der Konvertierung erhalten wurde}, label={lst:final_xml_convert}, language=xml]
<checkerrules>
    <!-- Class Definition -->
    <rule type="presence" points="0">
        <query>
            from x : V{Class} with isDefined(x.name) and
               stringLevenshteinDistance(x.name,
               "ClassA")&lt;3
               and (not x.isAbstract)
               report 1 end
        </query>
        <feedback>
            Es soll eine Klasse mit der Name
            ClassA bereitgestellt werden.
        </feedback>
    </rule>
    <!-- Class Definition -->
    <rule type="presence" points="0">
        <query>
            from x : V{Class} with isDefined(x.name) and
               stringLevenshteinDistance(x.name,
               "ClassB")&lt;3
               and (not x.isAbstract)
               report 1 end
        </query>
        <feedback>
            Es soll eine Klasse mit der Name
            ClassB bereitgestellt werden.
        </feedback>
    </rule>
    ...
    <!-- Generalization rule -->
    <rule type="presence" points="0">
        <query>
            from a,b : V{Class}
                   with
                      isDefined(a.name) and
                      a.name="ClassC" and
                      isDefined(b.name) and
                      b.name="ClassB" and
                      a --> V{Generalization} --> b
                   report 1 end
        </query>
        <feedback>
            Das Diagramm sollte eine Klasse
            ClassC enthalten, die von einer
            Oberklasse ClassB erbt.
        </feedback>
    </rule>
    <!-- Test Association Rule -->
    <rule type="presence" points="0">
        <query>
            from x,y : V{Class}
                           with
                              isDefined(x.name) and
                              x.name="ClassC" and
                              isDefined(y.name) and
                              y.name="ClassD" and
                              x --> V{Property}
                              --> V{Association}
                              &lt;-- V{Property}
                              &lt;-- y
                           report 1 end
        </query>
        <feedback prefix="Hinweis">
            Im Diagramm gibt es keine directe Association
            zwischen die Klasse "ClassD" und
            die Klasse "ClassC". Das kann durch eine
            bessere Modellierung vermieden werden.
        </feedback>
    </rule>
</checkerrules>
\end{lstlisting}

\section{Entwicklung eines Annotationssystems}
In der in Abschnitt~\ref{sec:darlegung-des-workflow-prozesses} beschriebenen ersten Phase (die das Schreiben des
plantText-Codes in den GReQL Converter umfasst), folgt die zweite Phase, in der die generierten Regelobjekte modifiziert
werden. Es ist jedoch wichtig zu beachten, dass dieser Prozess zur Modifikation der Regelobjekte als äußerst
arbeitsintensiv angesehen werden kann. Um dies zu verdeutlichen, kann das Beispiel der Anpassung der Anzahl der Punkte (Points)
für jede Regel in einem Diagramm mit mehr als zwanzig Regeln dienen. Diese Aufgabe erweist sich schnell als zeitaufwändig,
da jede Regel individuell bearbeitet werden muss, um die erforderlichen Änderungen vorzunehmen. Ebenso stellt sich die
gleiche Problematik ein, wenn es um die Eigenschaft ``Exact match'' (siehe~\ref{fig:code-uml}) geht (die festlegt, ob
der Name eines Attributs, einer Methode oder einer Klasse genau mit dem im Diagramm angegebenen Namen übereinstimmen
muss oder ob eine gewisse Fehlermarge akzeptiert wird).

Um diesen Prozess zu erleichtern, wurde im GReQL Converter ein Annotationssystem implementiert. Dieses Vorhaben zielt
zunächst darauf ab, einige Attribute des PlantUML-Parsers zu nutzen, die in Bezug auf UML-Diagramme vergleichsweise
selten verwendet werden, insbesondere Generika, sowie das Labeling-System von PlantText. Hierfür wurde eine spezifische
Syntax entwickelt, deren Einzelheiten in der Dokumentation des GReQL Converters erläutert sind. Das folgende Beispiel~\ref{fig:annotation}
zeigt einen Anwendungsfall dieser Syntax für die Annotation einer Klasse:

\begin{figure}[h]
    \centering
    \includegraphics[width=16cm]{images/annotation}
    \caption{Beispiel für die Verwendung des Annotationssystems}
    \label{fig:annotation}
\end{figure}

\begin{itemize}[label={}]
    \item \textbf{!class}: Die Verwendung der Direktive ``!class'' dient dazu, die Funktion der exakten Übereinstimmung
anhand von Klassennamen zu aktivieren.
    \item \textbf{!attr(0,1,3)}: Die Direktive ``!attr(0,1,3)'' wird verwendet, um die exakte Übereinstimmung in den
    ersten, zweiten und vierten Attributen, nämlich ``windows'', ``x'' und ``time'' zu aktivieren.
    \item \textbf{!attr(*)}: Wenn beabsichtigt wird, die exakte Übereinstimmung in allen Attributen zu aktivieren,
sollte die Direktive ``!attr(*)'' verwendet werden.
    \item \textbf{!method(1,2)}: Die Direktive ``!method(1,2)'' wird verwendet, um die exakte Übereinstimmung in den
zweiten und dritten Methoden, nämlich ``unlock'' und ``block'' zu aktivieren.
    \item \textbf{!method(*)}: Sollte der Wunsch bestehen, die exakte Übereinstimmung in allen Methoden zu
aktivieren, sollte die Direktive ``!method(*)'' verwendet werden.
    \item \textbf{p}: Die Variable ``p'' dient dazu, das Ausmaß zu bestimmen, in dem die Punkte (Points) der
Klassendefinitionsregel zuzuordnen sind.
    \item \textbf{ad-p}: In ähnlicher Weise bestimmt ``ad-p'' die Punkte (Points), die den
Attributregeln zugeordnet sind.
    \item \textbf{md-p}: Gleichzeitig dient ``md-p'' als Indikator für die Punkte (Points), die den Methodenregeln
zugeordnet sind.
\end{itemize}

Dieses Annotierungssystem beschleunigt und optimiert signifikant den Prozess der Generierung von GReQL-Code. Sobald
die Beherrschung dieses Annotierungssystems erreicht ist, werden Lehrende nicht länger auf die Zwischenrepräsentation
von Regelobjekten angewiesen sein, da sie in der Lage sein werden, GReQL-Code direkt mit der PlantText-Code und der
Annotation zu generieren.

\section{Erreichte Ergebnisse}

Der GReQL Converter ist ein Instrument zur Generierung von GReQL-Code aus zuvor bereitgestelltem PlantText-Code. Der
vorherige Abschnitt hat im Detail den Prozess der Extraktion von Regeln aus PlantText-Code bis hin zur Generierung
von GReQL-Code beschrieben. Es ist jedoch von entscheidender Bedeutung, die Frage zu beantworten, ob dieses Werkzeug
effektiv funktioniert. Selbst wenn es funktioniert, ist seine Nützlichkeit von Interesse. Im folgenden Kapitel wird
die Frage der Evaluation erörtert. Dabei werden die verschiedenen Prozesse ausführlich beschrieben, die zur Prüfung
des Tools verwendet wurden. Darüber hinaus wird eine gründliche Untersuchung durchgeführt, um die Relevanz dieses
Tools für verschiedene Lehrkräfte nachzuweisen.
