\chapter{Diskussion}

Dieses Kapitel wird in mehrere Abschnitte unterteilt sein. Zunächst wird eine Interpretation der Ergebnisse der
Evaluation vorgenommen, um fundierte Schlussfolgerungen zu ziehen. Anschließend wird eine eingehende Diskussion über
die verschiedenen Herausforderungen geführt, die im Laufe des Entwicklungsprozesses des Tools bewältigt wurden.
Abschließend wird eine umfassende Untersuchung durchgeführt, um die verschiedenen Möglichkeiten zur Verbesserung des
GReQL Converters zu erörtern und sein fortwährendes Potenzial zu bewerten. Diese thematische Struktur zielt darauf ab,
eine ganzheitliche und gründliche Analyse der Leistung, der Herausforderungen und der Verbesserungsaussichten im
Zusammenhang mit diesem Tool bereitzustellen.

\section{Interpretation der Evaluationsergebnisse}


Die Interviews ergaben überwiegend positive Ergebnisse. Der GReQL Converter stellt ein innovatives Tool dar, das als
erstes seiner Art die Generierung von GReQL-Code ermöglicht, um Klassendiagramme zu bewerten. Alle Befragten äußerten
sich äußerst positiv über die Nützlichkeit des GReQL Converters und betonten dessen Potenzial, Lehrern erheblich Zeit
beim Verfassen von GReQL-Code zu sparen. Sie sind außerdem bereit, das Tool ihren Kollegen zu empfehlen, die bereits
Erfahrung mit GReQL-Code für die Klassendiagrammbewertung haben.

Die grafische Benutzeroberfläche und die Nutzererfahrung stießen bei den meisten Befragten auf Zustimmung.
Dennoch wurde auf verschiedene Bereiche hingewiesen, die noch verbessert werden könnten, um das Tool weiter zu
optimieren. Obwohl der GReQL Converter die Generierung generischer Regeln stark erleichtert, stellten die Interviews
fest, dass bei zunehmender Komplexität manuelle Eingriffe erforderlich sind. Es wurde betont, dass fortlaufende
Verbesserungen an dem Tool dazu beitragen könnten, den Bedarf an solchen manuellen Interventionen im Laufe der Zeit zu
verringern. Es ist jedoch wichtig zu beachten, dass der GReQL Converter nicht darauf abzielt, die menschliche
Intervention zu ersetzen. Stattdessen sollte er als unterstützendes Instrument für Lehrkräfte betrachtet werden.


\section{Herausforderungen während des Entwicklungsprozesses}

Im Verlauf des Entwicklungsprozesses manifestierten sich verschiedene herausfordernde Sachverhalte, die eine gezielte
Entwicklungsdynamik bedingten. Dies wiederum zwang die Notwendigkeit zur Implementierung spezifischer Beschränkungen
oder die Abkehr von bestimmten funktionalen Aspekten.

\subsubsection{PlantUML Parser}

Bezüglich des PlantUML Parsers sind gewisse Limitationen zu konstatieren. Er ist nicht in der Lage, statische
Attribute, statische Methoden und statische Klassen zu erfassen. Indessen fand rasch eine Lösung in Form eines
Kompromisses Anklang, indem dem Anwender ermöglicht wird, diese Modifikationen manuell im Rahmen des Regel-Editors
vorzunehmen.

\subsubsection{GReQL Engine Optimizer}

Der GReQL Engine Optimizer  verfügt über einen Algorithmus zur Optimierung von Abfragen, um deren Ausführung zu
erleichtern und mögliche Probleme wie die Verwendung undefinierter Variablen, welche eine Abfrage fehlerhaft machen
könnten, zu umgehen. Nichtsdestoweniger kann dieser Optimierer zuweilen Unklarheit in der Abfrageausführung stiften.
Es besteht die Möglichkeit, dass eine Abfrage verfasst wird, die auf den ersten Blick in vollkommen korrektem Einklang
erscheint, jedoch bei der Ausführung vom Optimierer in einer Art und Weise modifiziert wird, welche die Abfrage invalide
werden lässt (Wie es bei einigen Regeln im WIKI der Fall ist~\cite{GReQL-wiki}). Daraus resultiert, dass die GReQL Engine
Fehlermeldungen retourniert. Zur Bewältigung dieser Thematik waren eigens maßgeschneiderte Abfragen erforderlich, welche
verschiedene Prüfungen vor der Ausführung durchführen. In dieser Hinsicht erweist sich die Verwendung des GReQL
Converters als vorteilhaft, indem er ausschließlich valide Abfragen zur Optimierung generiert und dem Nutzer die
Frustration erspart, scheinbar korrekte, aber nicht funktionierende Abfragen manuell zu konzipieren.


\subsubsection{Beschränkung auf BOUML}

Der Kern des GReQL Converters liegt in der Erstellung und Definition von Vorlagen, die für jede Regel festgelegt wurden.
Zur Herstellung dieser Vorlagen war es erforderlich, zunächst ein Diagramm, welches die jeweilige Regel in Anspruch
nimmt, mittels der Software BOUML zu modellieren. Anschließend erfolgte die grafische Darstellung mithilfe des GReQL
Engine, um abschließend die Regel aus der grafischen Darstellung abzuleiten. Diese Vorgehensweise impliziert, dass die
Mehrzahl der in Gebrauch genommenen Regeln ihren Ursprung in einer bildlichen Repräsentation eines Diagramms haben,
welches mithilfe von BOUML erstellt wurde. Dies stellt ein substantielles Problem dar, da die XMI-Repräsentationen der
Diagramme abhängig vom verwendeten Tool variieren. Als Beispiel generiert der Enterprise Architect offensichtlich eine
XMI-Datei, die sich von derjenigen generiert durch BOUML zu unterscheiden scheint. Dies hätte zur Konsequenz, dass die
Mehrzahl der durch den GReQL Converter generierten Regeln ungültig würde, sofern das zu beurteilende Diagramm mittels
eines alternativen Tools geschaffen wurde. Das bedeutet, dass die Auswahl des Tools, das für die Generierung der
Lösungen zur Beurteilung eingesetzt wird, von entscheidender Relevanz ist, was wiederum den GReQL Converter auf eine
spezifische Werkzeugauswahl oder auf die Nutzung von BOUML für die Gestaltung der zu beurteilenden Diagramme beschränkt.

\subsubsection{Primitive Datentypen}

Im Zusammenhang mit den von BOUML generierten Diagrammen ist zu berücksichtigen, dass sie einem ausgewiesenen Standard
entsprechen, nämlich dem UML-Standard 2.3~\cite{OMG_UML_23_Infrastructure}, der von BOUML in Gebrauch genommen wird.
Dieser Standard erkennt jedoch lediglich vier primitive Datentypen, nämlich int, bool, string und
UnlimitedNatural~\cite{OMG_UML_23_Infrastructure}. Diese Beschränkung führt dazu, dass Typen, die im Grundsatz als
primitiv erachtet werden könnten, wie double, float, char und dergleichen, schlichtweg nicht berücksichtigt werden.
Dieses Problem hat zur Folge, dass Typen in GReQL-Abfragen nicht überprüft werden können, sofern sie nicht den Kriterien
des UML 2.3-Standards genügen. In praktischer Konsequenz mussten gewisse Funktionen aufgegeben werden, etwa die
Überprüfung des Rückgabetyps einer Methode oder des Typs einer Variablen (sofern diese nicht gemäß UML 2.3 als primitiv
gelten), da die Repräsentation, die durch die GReQL Engine erzeugt wird (basierend auf dem XMI von BOUML), diese Typen
nicht erkennt und daher nicht darstellen kann. Infolgedessen können derlei Abfragen nicht ausgeführt werden.
\\~\\
Diese genannten Beschränkungen stellen zweifelsohne vielversprechende Ansatzpunkte für eine substantielle Verbesserung
des GReQL Converters dar. Daher wird in dem folgenden Abschnitt eine Diskussion darüber eingeleitet, wie der GReQL
Converter möglicherweise verbessert werden kann, um einige dieser inhärenten Einschränkungen zu überwinden.


\section{Potenziale für Weiterentwicklungen}

Der GReQL Converter, obwohl er vielversprechend ist, verwehrt sich der Illusion der Vollkommenheit. In diversen Domänen
sind signifikante Verbesserungen realisierbar, um seine Effektivität bei der Bewältigung spezifischer Herausforderungen
zu optimieren.

\subsection{Hinzufügen neuer Regeln}

Eine solche Möglichkeit zur Verbesserung manifestiert sich in der Erweiterung des Regelkatalogs. Obwohl die Entwicklung
des GReQL Converters bereits eine umfassende Berücksichtigung der Regeln, die der Modellierung von UML-Klassendiagrammen
zugrunde liegen, einschloss, bleiben einige subtile Nuancen unvollständig berücksichtigt. Zum Beispiel wurden keine
Regeln für Assoziationen mit spezifischer Richtung oder für nicht-ausgerichtete Beziehungen integriert. Während die
Assoziation zwischen zwei Klassen betrachtet wird, sofern eine Beziehung zwischen ihnen besteht, erfolgt keine explizite
Erfassung der Ausrichtung dieser Assoziation. Ebenso bleiben die mit Assoziationen verknüpften Rollennamen
unberücksichtigt. Diese und andere Feinheiten könnten zukünftige Erweiterungen des GReQL Converters sein, um das Tool
in Bezug auf die Generierung präziserer GReQL-Regeln zu bereichern.


\subsection{Erweiterung bestehender Regeln}

Eine Vielzahl von Regeln könnte von Verbesserungen profitieren, um eine größere Bandbreite von Designvarianten zu
unterstützen und kompatibler zu werden. Um dieses Ziel zu erreichen, ist es unerlässlich, den GReQL Converter an einer
beträchtlichen Anzahl von Diagrammen zu testen, um seine Grenzen und Einschränkungen zu identifizieren und umfassend
anzugehen. Durch eine breite Testbasis können potenzielle Schwachstellen erkannt und verbessert werden, um eine
robustere und flexiblere Anwendung des Tools zu gewährleisten.


\subsection{Erweiterung der Kompatibilität des GReQL Converter}

Wie zuvor erwähnt, ist der von GReQL Converter generierte GReQL-Code derzeit zu 100\% kompatibel mit Lösungsdiagrammen,
die mit BOUML erstellt wurden. Der GReQL Converter wurde jedoch mit Blick auf die Erweiterbarkeit zu anderen
Technologien entwickelt, die die Modellierung von UML-Diagrammen und die Generierung von XMI-Dateien ermöglichen. Die
Erweiterung auf andere Diagramm-Modellierungstools sollte hauptsächlich das Hinzufügen von regelbasierten Vorlagen für
diese Tools. Eine Möglichkeit, die Kompatibilität mit Enterprise Architect hinzuzufügen, wurde im Abschnitt über
die Erweiterbarkeit des GReQL-Converters gezeigt (siehe Abschnitt~\ref{sec:erweiterbarkeit-des-greql-converters}).


\subsection{Erweiterung des PlantUML-Syntaxparsers zur Erkennung von Syntaxfehlern}

Während des Interviews mit Dr. Michael Striewe wurde ein Verhalten bemerkt. Als er seinen PlantText-Code auf die
Plattform kopierte, konnten nur die Klassendefinitionen Regeln (``Class Definition'') generiert werden, und die
Assoziationsregeln wurden nicht erkannt. Dies lag daran, dass die Klassennamen in den Assoziationen in
Anführungszeichen standen, was der Parser nicht erkennen konnte. Dieses Problem lässt sich dadurch erklären, dass der
Parser solche Ausdrücke nicht als Fehler erkennt. Es gibt mehrere Arten von gültigem PlantText-Code, der jedoch nicht
zwangsläufig vom Parser verarbeitet werden kann. Aus diesem Grund sollte man sich an die Dokumentation halten, um die
Regeln zu schreiben, und auf der Entwicklungsseite einen Weg finden, diese Fehler zu identifizieren und insbesondere den
Benutzer darauf aufmerksam zu machen, dass etwas nicht wie erwartet funktioniert. Das gilt auch für den Fehler, den er
bezüglich des Enums erwähnt hat, der eine spezifische Syntax im GReQL Converter erfordert.


Der PlantUML-Parser ist ein Open-Source-Projekt, was bedeutet, dass der Quellcode für jeden zugänglich,
modifizierbar und verteilbar ist. Es ist also möglich, direkt auf der Parser-Ebene Änderungen vorzunehmen, um Funktionen
hinzuzufügen und damit seine Fähigkeiten zu erweitern.


\subsection{Erweiterung auf andere UML-Diagrammtypen}

Im Rahmen dieser Masterarbeit wurde der GReQL Converter auf die Generierung von GReQL-Code für UML-Klassendiagramme
spezialisiert. Während der Entwicklung wurde jedoch berücksichtigt, dass das Tool erweitert werden kann, um auch andere
Arten von Diagrammen wie Aktivitäts-, Sequenz- oder sogar Use-Case-Diagramme zu unterstützen (siehe
Abschnitt~\ref{sec:erweiterbarkeit-des-greql-converters}). Es ist bereits möglich, mithilfe von GReQL Code zu schreiben,
um diese genannten Diagrammtypen zu evaluieren. Zusätzlich sind diese Diagramme mithilfe von PlantUML modellierbar.
Folglich könnte der GReQL Converter tatsächlich erweitert werden, um Code für verschiedene Arten von Diagrammen zu
generieren. Es wäre also ein Ansatz zur Verbesserung des GReQL Converters, nicht nur auf Klassendiagramme beschränkt zu
sein, sondern auch andere Diagrammtypen zu unterstützen. Dies würde das Tool vielseitiger machen und seine
Anwendungsfälle deutlich erweitern.


\section{Zusammenfassung der Diskussion}

Die Entwicklung des GReQL Converters basierte auf der grundlegenden Überlegung, ein Tool zu schaffen, das sich
kontinuierlich weiterentwickeln kann. Dieser Aspekt fand sich in der Gestaltung des Codes wider, der gemäß etablierter
Designprinzipien und bewährter Praktiken verfasst wurde. Der Fokus lag darauf, kommenden Entwicklern, die an diesem Tool
arbeiten, eine klare Orientierung zu bieten und ihre Entwicklererfahrung erheblich zu verbessern, wie auch McConnell es
erwähnt~\cite{mcconnell2006software}. Diese strategische Herangehensweise zielt darauf ab, eine solide Grundlage zu
schaffen, die zukünftige Erweiterungen und Anpassungen erleichtert und einen nahtlosen Entwicklungsprozess für kommende
Versionen des GReQL Converters ermöglicht.
