\chapter{Spielwiese TODO wieder entfernen!} \label{spielwiese}

Zwölf Boxkämpfer jagen Viktor quer über den großen Sylter ``Deich''.

Lorem  ipsum dolor sit amet, consectetur adipisici elit, sed eiusmod tempor incidunt ut labore et dolore magna aliqua. Ut enim ad minim veniam, quis nostrud exercitation ullamco laboris nisi ut aliquid ex ea commodi consequat. Quis aute iure reprehenderit in voluptate velit esse cillum dolore eu fugiat nulla pariatur. Excepteur sint obcaecat cupiditat non proident\footnote{this is an url \url{https://wikipedia.de}}, sunt in culpa qui officia deserunt mollit anim id est laborum. 

\section{Abschnitt Eins - Gleichungen}

\begin{equation} \label{eq:relativity}
e = m \cdot c^{2}
\end{equation}

Lorem ipsum dolor sit amet, \autoref{eq:relativity} consectetur adipisici elit, sed eiusmod tempor incidunt ut labore et dolore magna aliqua. Ut enim ad minim veniam, quis nostrud exercitation ullamco $m=\frac{1 \cdot n}{2k!}$ laboris nisi ut aliquid ex ea commodi consequat. Quis aute iure reprehenderit in voluptate velit esse cillum dolore eu fugiat nulla pariatur. Excepteur sint obcaecat cupiditat non proident, sunt in culpa qui officia deserunt mollit anim id est laborum.

$$ e=\frac{a}{b}$$

\subsection{Unterabschnitt Eins}
Lorem ipsum dolor sit amet, consectetur adipisici elit, sed eiusmod tempor incidunt ut labore et dolore magna aliqua. Ut enim ad minim veniam, quis nostrud exercitation ullamco laboris nisi ut aliquid ex ea commodi consequat. Quis aute iure reprehenderit in voluptate velit esse cillum dolore eu fugiat nulla pariatur. Excepteur sint obcaecat cupiditat non proident, sunt in culpa qui officia deserunt mollit anim id est laborum.

\section{Abschnitt Zwei}

\begin{figure}
	\centering
	\includegraphics[width=8cm]{Plot}
	\caption[Kurzbeschreibung]{Die Graphik zeigt gedöns.}
	\label{fig:labelForPlot}
\end{figure}


Lorem ipsum dolor sit amet, consectetur adipisici elit, sed eiusmod tempor incidunt ut labore et dolore magna aliqua. Ut enim ad minim veniam, quis nostrud exercitation ullamco laboris nisi ut aliquid ex ea commodi consequat. Quis aute iure reprehenderit in voluptate velit esse cillum dolore eu fugiat nulla pariatur. Excepteur sint obcaecat cupiditat non proident, sunt in culpa qui officia deserunt mollit anim id est laborum. \autoref{fig:subPlot1} 

Lorem ipsum dolor sit amet, consectetur adipisici elit, sed eiusmod tempor incidunt ut labore et dolore magna aliqua. Ut enim ad minim veniam, quis nostrud exercitation ullamco laboris nisi ut aliquid ex ea commodi consequat. Quis aute iure reprehenderit in voluptate velit esse cillum dolore eu fugiat nulla pariatur. Excepteur sint obcaecat cupiditat non proident, sunt in culpa qui officia deserunt mollit anim id est laborum. \autoref{fig:labelForPlot}.

\subsection{Unterabschnitt Eins}

Lorem ipsum dolor sit amet, consectetur adipisici elit, sed eiusmod tempor incidunt ut labore et dolore magna aliqua. Ut enim ad minim veniam, quis nostrud exercitation ullamco laboris nisi ut aliquid ex ea commodi consequat. Quis aute iure reprehenderit in voluptate velit esse cillum dolore eu fugiat nulla pariatur. Excepteur sint obcaecat cupiditat non proident, sunt in culpa qui officia deserunt mollit anim id est laborum. \autoref{fig:subPlot2}

Lorem ipsum dolor sit amet, consectetur adipisici elit, sed eiusmod tempor incidunt ut labore et dolore magna aliqua. Ut enim ad minim veniam, quis nostrud exercitation ullamco laboris nisi ut aliquid ex ea commodi consequat. Quis aute iure reprehenderit in voluptate velit esse cillum dolore eu fugiat nulla pariatur. Excepteur sint obcaecat cupiditat non proident, sunt in culpa qui officia deserunt mollit anim id est laborum. \autoref{fig:subPlotAll}

\begin{figure}
	\centering
	\begin{subfigure}{.49\textwidth}
		\includegraphics[width=\textwidth]{Plot}
		\caption[Kurzbeschreibung]{Die Graphik zeigt gedöns a.}
		\label{fig:subPlot1}
	\end{subfigure}
	\hfill
	\begin{subfigure}{.49\textwidth}
		\includegraphics[width=\textwidth]{Plot}
		\caption[Kurzbeschreibung]{Die Graphik zeigt gedöns b.}
		\label{fig:subPlot2}
	\end{subfigure}
	\caption[Kurzbeschreibung Gedöns All]{Die Graphik zeigt gedöns insgesammt.}
	\label{fig:subPlotAll}
\end{figure}

\section[Kurztitel 3]{Abschnitt Drei - Abkürzungen}
Lorem  ipsum dolor sit amet, consectetur adipisici elit, sed eiusmod  tempor incidunt ut labore et dolore magna aliqua. Ut enim ad minim veniam, quis nostrud exercitation ullamco laboris nisi ut aliquid ex ea commodi consequat. Quis aute iure reprehenderit in voluptate velit esse cillum dolore eu fugiat nulla pariatur. Excepteur sint obcaecat cupiditat non proident,  sunt in culpa qui officia deserunt mollit anim id est laborum. 


Dieser Teil kann genutzt werden, um Bild-, Tabellen- und Code-Elemente herauszukopieren.
Generell für bessere Schreib/ Debug Übersicht immer nur einen Satz pro angefangene Zeile schreiben (\mono{Enter} drücken nach jedem Punkt).
\LaTeX~rendert das trotzdem als Fließtext. \todo{Das ist eine ToDo-Nachricht}

\vspace{3cm}

% Normalen Absatz einfach mit leeren Absätzen trennen.
Lorem ipsum dolor sit amet, consetetur sadipscing elitr, sed diam nonumy eirmod tempor invidunt ut labore et dolore magna aliquyam erat, sed diam voluptua. 

At vero eos et accusam et justo duo dolores et ea rebum. Stet clita kasd gubergren, no sea takimata sanctus est Lorem ipsum dolor sit amet.


Abschnitte referenzieren können wir wie hier: \autoref{spielwiese} oder mit dem Namen \nameref{spielwiese}.
\enquote{Anführungszeichen}.
Direktes Zitat: \enquote{The toaster is the greatest invention since sliced bread} %\parencite[14]{einstein}.
Blocktext:

\blockquote{
Lorem ipsum dolor sit amet, consetetur sadipscing elitr, sed diam nonumy eirmod tempor invidunt ut labore et dolore magna aliquyam erat, sed diam voluptua.
At vero eos et accusam et justo duo dolores et ea rebum. Stet clita kasd gubergren, no sea takimata sanctus est Lorem ipsum dolor sit amet.
}

Wenn man Abkürzungen wie verwendet, werden diese automatisch im Verzeichnis gelistet, verlinkt und beim zweiten Mal wird nur noch kurz geschrieben. 
Fußnote\footnote{Hier steht Text. \url{https://www.google.com/}, aufgerufen am \today.}.
Url ohne Extras als Fußnote\footurl{https://example.com}.

Liste:
\begin{enumerate}
    \item Item
    \item Item
\end{enumerate}

Unnummerierte Liste:
\begin{itemize}
    \item Item
    \item Item
\end{itemize}

\section{Quellcode}

Im Fließtext \mono{wie hier}.
Als Block wie in \autoref{spielwiese:code}.

% https://de.overleaf.com/learn/latex/Code_listing
\begin{lstlisting}[caption={[Codebeispiel]Codebeispiel mit Hello World in Java}, label=spielwiese:code, float=!ht, language=java]
public class HelloWorld {
    public static void main (String[] args) {
        System.out.println("Hello World!");
    }
}
\end{lstlisting}

\section{Formeln}

Dies ist eine inline-Formel: $a^2+b^2=c^2$ es geht aber auch eine Formel in der ganzen Zeile:
%
$$a^2+b^2=c^2$$
%
Alternativ kann auch die \mono{align}-Umgebung verwendet werden.
Hier wird am Gleichheitszeichen ausgerichtet: % https://de.overleaf.com/learn/latex/Aligning_equations_with_amsmath
%
\begin{align*} % ohne Stern falls nummeriert
2x - 5y &=  8 \\ 
3x + 9y + z &=  -12 + 14
\end{align*}
%
Oder mit Nummerierung und unausgerichtet wie in \ref{spielwiese:formel-gather2}:
%
\begin{gather} % man beachte, dass gather nicht ausrichtet. gather* nutzen für unnummeriert
2x - 5y = 8 \label{spielwiese:formel-gather1} \\ 
3x^2 + 9y = 3a + c \notag \\ % \notag führt dazu, dass hier keine Zahl steht
8x = 8 \label{spielwiese:formel-gather2}
\end{gather}
%
Es geht auch mit Sub-Nummerierung wie folgt:
\begin{subequations} \label{spielwiese:formel}
	\begin{gather}
		a^2+b^2=c^2 \label{spielwiese:formel:a} \\
		e=mc^2 \label{spielwiese:formel:b}
	\end{gather}
\end{subequations}
%
Referenzieren kann man diese einfach normal: \autoref{spielwiese:formel} oder \ref{spielwiese:formel:b}.

\section{Bilder}

Wie wir hier sehen, können wir Abbildungen referenzieren (\scvgl \autoref{spielwiese:testbild}).
Wir können auch nur die Nummer angeben, \sczb \ref{spielwiese:testbild}.
Es ist auch möglich, Grafiken nebeneinanderzustellen, wie \sczb in \autoref{spielwiese:fragment}, die in \autoref{spielwiese:fragment1} und \autoref{spielwiese:fragment2} aufgeteilt ist.
Nur die \ref{spielwiese:fragment} erscheint dabei im Verzeichnis.
Zwei gleichwertige Grafiken nebeneinander sind in \autoref{spielwiese:vollwertig1} und \autoref{spielwiese:vollwertig2} zu finden.
Beide Grafiken erscheinen dann als \ref{spielwiese:vollwertig1} und \ref{spielwiese:vollwertig2} im Verzeichnis.

% https://de.overleaf.com/learn/latex/Positioning_of_Figures
% h = here, t = top, b = bottom, ! = überschreibe Parameter für "guten Stil"
\begin{figure}[!ht]
	\centering
    % z.B. "width=0.5\textwidth" für halbe Seitenbreite, "width=\textwidth" für gesamte Seitenbreite
    % "height=3cm" wäre z.B. ebenso möglich
	\includegraphics[width=\textwidth]{images/test.jpg}  % 
    % in eckigen Klammern der Text, wie er im Verzeichnis erscheint, in geschweiften Klammern der Text, wie er in der Unterschrift erscheint
	\caption[Testbild]{Dies ist ein Testbild.}
	\label{spielwiese:testbild}
\end{figure}

% https://texfragen.de/bilder_nebeneinander
% h = here, t = top, b = bottom, ! = überschreibe Parameter für "guten Stil"
\begin{figure}[!ht]
	\centering
    % Insgesamt sollten die addierten Breiten 0.95 nicht überschreiten.
	\begin{subfigure}{.45\textwidth} % Breite 1
		\includegraphics[width=\textwidth]{images/test.jpg} % hier volle Größe angeben, wird durch subfigure-Befehl gesteuert
		\caption{Erstes Fragment}
		\label{spielwiese:fragment1}
	\end{subfigure}
	\hspace{.05\textwidth} % kann zusätzlich eingefügt werden, um Platz aufzufüllen
	\begin{subfigure}{.45\textwidth} % Breite 2
		\includegraphics[width=\textwidth]{images/test.jpg} % hier volle Größe angeben, wird durch subfigure-Befehl gesteuert
		\caption{Zweites Fragment}
		\label{spielwiese:fragment2}
	\end{subfigure}
	\caption[Bild mit Fragmenten]{Dies ist eine Grafik mit zwei Fragmenten.}
	\label{spielwiese:fragment}
\end{figure}

% Erklärung siehe oben
\begin{figure}[!ht]
    \centering
    \begin{minipage}[b]{.45\textwidth} % [b] => Ausrichtung an \caption
        \includegraphics[width=\textwidth]{images/test.jpg}
        \caption[Vollwertige Grafiken (1)]{Erste Grafik}
        \label{spielwiese:vollwertig1}
    \end{minipage}
    \hspace{.05\textwidth}
    \begin{minipage}[b]{.45\textwidth} % [b] => Ausrichtung an \caption
        \includegraphics[width=\textwidth]{images/test.jpg}
        \caption[Vollwertige Grafiken (2)]{Zweite Grafik}
        \label{spielwiese:vollwertig2}
    \end{minipage}
\end{figure}

\clearpage

\section{Tabellen}

Im Folgenden einfache Tabellen, wie \sczb \autoref{spielwiese:tabelle1}:

\begin{table}[!ht]
	\centering
	\caption{Kleine Tabelle}
	\label{spielwiese:tabelle1}
	\begin{tabular}{|l|c|l|} % | = Linie, l = linksbündig, c = zentriert, r = rechtsbündig
		\hline Text & Text & Text Text \\\hline
		Text & Text Text & Text \\\hline
	\end{tabular}
\end{table}

\begin{table}[!ht]
	\centering
	\caption{Kleine Tabelle mit fester Spaltenbreite}
	\label{spielwiese:tabelle2}
	\begin{tabular}{|l|C{3cm}|R{5cm}|} % L, C und R für linksbündig, zentriert, rechtsbündig mit fester Spaltenbreite
		\hline Text & Text & Text Text \\\hline
		Text & Text Text & Text \\\hline
	\end{tabular}
\end{table}

\begin{table}[!ht]
	\centering
	\caption{Kleine Tabelle mit Blocksatz}
	\label{spielwiese:tabelle3}
	\begin{tabular}{|l|L{3cm}|P{5cm}|} % P = Blocksatz, Text wird umgebrochen
		\hline Text & Text & Text Text \\\hline
		Text Text & Text Text & Dieser lange Text wird umgebrochen. Dieser lange Text wird umgebrochen. \\\hline
	\end{tabular}
\end{table}

\begin{tabularx}{\textwidth}{|l|X|} % X = Spalte bis zum Rand auffüllen
	\caption{Tabelle mit aufgefüllter Spalte} \label{spielwiese:tabelle4} \\\hline \endhead
    \hline Text & Text \\ \hline
    Text Text & Dieser lange Text wird umgebrochen und aufgefüllt.
    Dieser lange Text wird umgebrochen und aufgefüllt.
    Dieser lange Text wird umgebrochen und aufgefüllt. \\ \hline
\end{tabularx}

\begin{table}[!t]
    \centering
    \caption{Tabelle mit Multi-Columns}
    \label{spielwiese:multicoltabelle}
    \begin{tabular}{p{1cm}llllcp{2cm}}
        \toprule%
        \multirow{2}{*}{$P_{ID}$} & \multicolumn{4}{c}{Order of Application}\\%
        \cmidrule{2-5}%
        {} & 1 & 2 & 3 & 4\\%
        \midrule%
        \addlinespace
        0 & a & b & c & d\\%
        1 & b & c & d & a\\%
        2 & c & d & a & b\\%
        3 & $x_{1}$ & $x_{2}$ & $x_{3}$ & $x_{4}$\\%
        \bottomrule%
    \end{tabular}
\end{table}


\autoref{spielwiese:langetabelle} erstreckt sich sogar auf mehrere Seiten:

% \begin{longtblr}[
% 	caption= {Lange Überschrift für eine mehrseitige Tabelle. Lange Überschrift für eine mehrseitige Tabelle. Lange Überschrift für eine mehrseitige Tabelle.},
%     entry = {Mehrseitige Tabelle},
% 	label = {spielwiese:langetabelle},
% 	note{a} = {Fußnote, hier geht die Linie nicht ganz durch}
% ]{width=\linewidth, rowhead=1, colspec={ll|X[l]|}} % x ist Auto-Breite
% 	\hline \SetRow{c, font=\bfseries} Spalte 1 & Spalte 2 & Spalte 3 \\\hline
%     Text & Text & Text \\\hline
%     Text & Text & Text \\\hline
%     Text & Text & Text \\\hline
%     Text\TblrNote{a} & Text & Text \\\cline{2-3}
%     Text & Text & Text \\\hline
%     Text & Text & Text \\\hline
%     Text & Text & Text \\\hline
%     Text & Text & Text \\\hline
%     Text & Text & Text \\\hline
%     Text & Text & Text \\\hline
%     Text & Text & Text \\\hline
%     Text & Text & Text \\\hline
%     Text & Text & Text \\\hline
%     Text & Text & Text \\\hline
%     Text & Text & Text \\\hline
%     Text & Text & Text \\\hline
%     Text & Text & Text \\\hline
%     Text & Text & Text \\\hline
%     Text & Text & Text \\\hline
%     Text & Text & Text \\\hline
%     Text & Text & Text \\\hline
%     Text & Text & Text \\\hline
%     Text & Text & Text \\\hline
%     Text & Text & Text \\\hline
%     Text & Text & Text \\\hline
%     Text & Text & Text \\\hline
%     Text & Text & Text \\\hline
%     Text & Text & Text \\\hline
%     Text & Text & Text \\\hline
%     Text & Text & Text \\\hline
%     Text & Text & Text \\\hline
%     Text & Text & Text \\\hline
%     Text & Text & Text \\\hline
% \end{longtblr}

\begin{tabularx}{\textwidth}{L{2cm}L{2cm}X}
	% Erster Kopf
	\caption[Mehrseitige Tabelle]{Lange Überschrift für eine mehrseitige Tabelle} \label{spielwiese:langetabelle} \\ \hline
	\textbf{Spalte 1} & \textbf{Spalte 2} & \textbf{Spalte 3} \\ \hline \endfirsthead
	% Zweiter Kopf
	\multicolumn{3}{c}{\small{\tablename\ \thetable{} (Fortsetzung)}} \\ \hline
	\textbf{Spalte 1} & \textbf{Spalte 2} & \textbf{Spalte 3} \\ \hline \endhead
	% Beginn Tabelle
    Text Text Text Text Text Text Text & Text Text Text Text Text Text Text & Text Text Text Text Text Text Text Text Text Text Text \\\hline
    Text & Text & Text \\\hline
    Text & Text & Text \\\hline
    Text & Text & Text \\\cline{2-3}
    Text & Text & Text \\\cline{2-3}
    Text & Text & Text \\\hline
    Text & Text & Text \\\hline
    Text & Text & Text \\\hline
    Text & Text & Text \\\hline
    Text & Text & Text \\\hline
    Text & Text & Text \\\hline
    Text & Text & Text \\\hline
    Text & Text & Text \\\hline
\end{tabularx}
