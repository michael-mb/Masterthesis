\chapter{Evaluation}

Dieses Kapitel widmet sich einer umfassenden Analyse der Relevanz und Effektivität des GReQL Converters als innovatives
Werkzeug. Spezifisch zielt diese Abschnitt darauf ab, die grundlegende Frage zu beantworten, ob dieses Instrument
tatsächlich im akademischen und pädagogischen Kontext nützlich ist. Um dieses Ziel zu erreichen, ist es von größter
Wichtigkeit, eine systematische Herangehensweise zu verfolgen, die die Anwendung verschiedener Bewertungsmethoden
beinhaltet, um die Vorzüge und Effektivität des GReQL Converters nachzuweisen. Dieses Kapitel behandelt ausführlich die
angewandten Ansätze zur Prüfung des GReQL Converters, die Datensammlungsmethoden und die durchgeführten Analysen zur
Messung seiner Nützlichkeit. Letztendlich geht es darum, empirisch festzustellen, ob dieses Werkzeug konkrete Vorteile
und einen signifikanten Mehrwert für Lehrende und Lernende bietet.

\section{Erreichte Ziele}
todo - write something here

\section{Umfrage zur Bewertung des GReQL Converters}
todo - write something here