\chapter*{Abstract - Englisch}

This master's thesis focuses on the development of the GReQL Converter to simplify the process of generating feedback
rules for UML models in the context of E-Assessment systems. The tool is designed to provide (semi-)automated support to
teachers in creating GReQL code required for evaluating student submissions related to UML class diagram tasks on the
JACK platform. The GReQL Converter is designed to address the challenges that teachers face when creating GReQL code,
such as the need for expertise in GReQL syntax and the time-consuming nature of the process. The tool aims to streamline
the process of generating feedback rules for UML models, thereby improving the efficiency and effectiveness of
E-Assessment systems. The results of this work demonstrate the feasibility and effectiveness of the GReQL Converter
in simplifying the process of generating feedback rules for UML models.


\chapter*{Abstract - Deutsch}
Diese Masterarbeit konzentriert sich auf die Entwicklung des GReQL Converters, um den Prozess der Generierung
von Feedback-Regeln für UML-Modelle im Kontext von E-Assessment-Systemen zu vereinfachen. Das Tool soll Lehrern
(halb-)automatisierte Unterstützung bei der Erstellung von GReQL-Code bieten, der für die Bewertung von
Studenteneinreichungen bezüglich UML-Klassendiagramm-Aufgabe auf der JACK-Plattform erforderlich ist.
Der GReQL Converter ist darauf ausgelegt, die Herausforderungen zu bewältigen, mit denen Lehrer bei der
Erstellung von GReQL-Code konfrontiert sind, wie z.B. die Notwendigkeit von Expertise in GReQL-Syntax und der
zeitaufwändige Charakter des Prozesses. Das Tool soll den Prozess der Generierung von Feedback-Regeln für
UML-Modelle vereinfachen und damit die Effizienz und Effektivität von E-Assessment-Systemen verbessern. Die Ergebnisse
dieser Arbeit zeigen die Machbarkeit und Wirksamkeit des GReQL Converter bei der Vereinfachung des Prozesses der
Generierung von Feedback-Regeln für UML-Modelle.

